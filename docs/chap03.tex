\chapter{缺陷格子可能存在的位置与相对应的平铺方案}

\section{骨牌大小k = 3时的情况}
由\ref*{basic-theorem-1}有,此时棋盘的大小只可能为$(3x_m + 1) \times (3x_n + 2)$,不妨假设缺陷格子的坐标分别为$(i_1, j_1), (i_2, j_2)$。

在这里我们给出一个定理,来辅助我们之后的证明

\begin{theorem}
	\label{basic-theorem-2}
	对于$m \times n$的广义棋盘和$1 \times k$的骨牌,且$m \equiv 1 (mod k), n \equiv 2 (mod k)$,
	记某格子的坐标为$(i, j)$,意味着某格子处于第i行第j列,则如果存在$i \times j$的缺陷C,$i \equiv 1 (mod k), j \equiv 2 (mod k), i \le m, j \le n$,
	且棋盘的左上角和右下角的格子坐标分别为$(3k_{11} + 1, 3k_{12} + 1)$,$(3k_{21} + 1, 3k_{22} + 2)$,
	则该棋盘一定能被平铺
\end{theorem}
\begin{proof}
	当存在这样的缺陷时,我们可以把棋盘分成如\ref*{fig:nine-separate}的9个区域,分别如图所示,由\ref*{basic-lemma-1},可得除了缺陷C外的8个区域均可被$1 \times k$的骨牌平铺,
	而中央的缺陷C不需要被平铺,故此时整个棋盘已经可以被平铺,证毕。

	\begin{table}[h]
		\centering
		\caption{对棋盘的9划分}
		\begin{tabular}{|c|c|c|}

			\hline
			$3k_{12} \times 3k_{11}$               & $(3k_{22} - 3k_{12} + 2)\times 3k_{11}$  & $3k_{22} \times 3k_{11}$                   \\
			\hline
			$3k_{12} \times (3k_{21}-3k_{11} + 1)$ & $C: i \times j$                          & $3k_{22} \times (3k_{21}-3k_{11} + 1 + 1)$ \\
			\hline
			$3k_{12} \times 3k_{21}$               & $3 (k_{22} - k_{12}) + 2 \times 3k_{21}$ & $3k_{22} \times 3k_{21}$                   \\
			\hline
		\end{tabular}
		\label{fig:nine-separate}
	\end{table}
\end{proof}

因此根据\ref*{basic-theorem-2},我们可以得把问题转化为:
对于某个残缺两格的棋盘,如果他的两个缺陷可以组合成一个满足\ref*{basic-theorem-2}中条件的$i \times j$大小的缺陷,那么该棋盘可以被平铺。

在图\ref*{fig:k-order-staining-1}的基础上,我们对棋盘进行两种染色,即

\begin{table}[h]
	\caption{3阶染色}
	\label{fig:3-order-staining}
	\begin{subtable}{.5\linewidth}
		\centering
		\caption{3阶染色-1}
		\begin{tabular}{|c|c|c|c|c|c|}
			\hline
			1        & 2        & 3        & 1        & 2        & $\cdots$ \\
			\hline
			2        & 3        & 1        & 2        & 3        & $\cdots$ \\
			\hline
			3        & 1        & 2        & 3        & 1        & $\cdots$ \\
			\hline
			1        & 2        & 3        & 1        & 2        & $\cdots$ \\
			\hline
			2        & 3        & 1        & 2        & 3        & $\cdots$ \\
			\hline
			$\vdots$ & $\vdots$ & $\vdots$ & $\vdots$ & $\vdots$ & $\ddots$ \\
			\hline
		\end{tabular}
		\label{fig:3-order-staining-1}
	\end{subtable}%
	\begin{subtable}{.5\linewidth}
		\centering
		\caption{3阶染色-2}
		\begin{tabular}{|c|c|c|c|c|c|}
			\hline
			1        & 2        & 3        & 1        & 2        & $\cdots$ \\
			\hline
			3        & 1        & 2        & 3        & 1        & $\cdots$ \\
			\hline
			2        & 3        & 1        & 2        & 3        & $\cdots$ \\
			\hline
			1        & 2        & 3        & 1        & 2        & $\cdots$ \\
			\hline
			3        & 1        & 2        & 3        & 1        & $\cdots$ \\
			\hline
			$\vdots$ & $\vdots$ & $\vdots$ & $\vdots$ & $\vdots$ & $\ddots$ \\
			\hline
		\end{tabular}
		\label{fig:3-order-staining-2}
	\end{subtable}
\end{table}

不妨设棋盘行数$m \equiv 1 (mod 3)$,列数$n \equiv 2 (mod 3)$,记$f(x)$为染色为x的格子的数目,显然有$f(1) = f(2) = f(3) + 1$。

记$P(i, j) = \left\{a, b\right\}$, 意味着在坐标为$(i, j)$在\ref*{fig:3-order-staining-1}和\ref*{fig:3-order-staining-2}分别染色为$a, b$,
并据此按照以下规则重新染色得到\ref*{fig:3-order-staining-last}。

\begin{itemize}
	\item $P(i, j) = \left\{2, 2\right\}$染色为1, $P(i, j) = \left\{2, 2\right\}$染色为2
	\item $P(i, j) = \left\{1, 2\right\}$染色为4, $P(i, j) = \left\{2, 1\right\}$染色为5
	\item $P(i, j) = \left\{3, any\right\}$或$P(i, j) = \left\{any, 3\right\}$染色为0
\end{itemize}

\begin{table}[h]
	\centering
	\caption{3阶重染色}
	\begin{tabular}{|c|c|c|c|c|c|c|}
		\hline
		1        & 2        & 0        & 1        & 2        & 0        & $\cdots$ \\
		\hline
		0        & 0        & 4        & 0        & 0        & 4        & $\cdots$ \\
		\hline
		0        & 0        & 5        & 0        & 0        & 5        & $\cdots$ \\
		\hline
		1        & 2        & 0        & 1        & 2        & 0        & $\cdots$ \\
		\hline
		0        & 0        & 4        & 0        & 0        & 4        & $\cdots$ \\
		\hline
		0        & 0        & 5        & 0        & 0        & 5        & $\cdots$ \\
		\hline
		$\vdots$ & $\vdots$ & $\vdots$ & $\vdots$ & $\vdots$ & $\vdots$ & $\ddots$ \\
		\hline
	\end{tabular}
	\label{fig:3-order-staining-last}
\end{table}

而该棋盘可被平铺的充分条件为缺陷的两个格子分别落在染色1和染色2的位置,意味着如果棋盘可以被平铺,则只有以下两种情况

\begin{itemize}
	\item $P(i_1, j_1) = \left\{2, 2\right\}, P(i_2, j_2) = \left\{2, 2\right\}$,即在\ref*{fig:3-order-staining-last}中分别位于染色1和染色2处
	\item $P(i_1, j_1) = \left\{1, 2\right\}, P(i_2, j_2) = \left\{2, 1\right\}$,即在\ref*{fig:3-order-staining-last}中分别位于染色4和染色5处
\end{itemize}

\subsection{在染色中分别处于4和5}

由图,我们可以简单的写出4和5所对应的格子的坐标,

\begin{itemize}
	\item 4的坐标: $(3k_{41} + 2, 3k_{42})$
	\item 5的坐标: $(3k_{51} + 3, 3k_{52})$
\end{itemize}

由于棋盘具有对称性,在这里我们不妨假设$k_{51} \ge k_{41}$,,因此会有以下2种情况:



\begin{table}[ht]
	\centering
	\caption{挖去4和5的染色-1}
	\begin{tabular}{|cc|c|cc|c|cc|}
		\hline
		\multicolumn{6}{|c|}{$1 \times (3k_{51} - 3k_{41} + 3)$}                & \multicolumn{2}{|c|}{\multirow{4}*{$(3k_{52} - 3k_{42} + 3) \times 2$}}                                                                                        \\
		\cline{1-6}
		\multicolumn{2}{|c|}{\multirow{4}*{$(3k_{52} - 3k_{42} + 3) \times 2$}} & 4                                                                       & \multicolumn{3}{|c|}{$1 \times (3k_{51} - 3k_{41})$}                       &   &     \\
		\cline{3-6}
		                                                                        &                                                                         & \multicolumn{4}{|c|}{$(3k_{52} - 3k_{42}) \times (3k_{51} - 3k_{41} + 1)$} &   &     \\
		\cline{3-6}
		                                                                        &                                                                         & \multicolumn{3}{|c|}{$1 \times (3k_{51} - 3k_{41})$}                       & 5 &   & \\
		\cline{3-8}
		                                                                        &                                                                         & \multicolumn{6}{|c|}{$1 \times (3k_{51} - 3k_{41} + 3)$}                             \\
		\hline
	\end{tabular}
	\label{fig:4-5-painting}
\end{table}

\begin{table}[ht]
	\centering

	\caption{挖去4和5的染色-2}
	\begin{tabular}{|ccccc|cc|}
		\hline
		\multicolumn{5}{|c|}{\multirow{2}*{$2 \times (3k_{51} - 3k_{41} + 3)$}} & \multicolumn{2}{|c|}{\multirow{3}*{$3 \times 2$}}                                                                                                                                           \\
		                                                                        &                                                   &                                                                            &   &                                                   &  & \\
		\cline{1-5}
		\multicolumn{2}{|c|}{\multirow{3}*{$2 \times (3k_{42} - 3k_{52} + 3)$}} & 5                                                 & \multicolumn{2}{|c|}{$1 \times (3k_{51} - 3k_{41})$}                       &   &                                                        \\
		\cline{3-7}
		                                                                        &                                                   & \multicolumn{5}{|c|}{$(3k_{42} - 3k_{52}) \times (3k_{51} - 3k_{41} + 3)$}                                                              \\
		\cline{3-7}
		                                                                        &                                                   & \multicolumn{2}{|c|}{$1 \times (3k_{51} - 3k_{41})$}                       & 4 & \multicolumn{2}{|c|}{\multirow{3}*{$3 \times 2$}}      \\
		\cline{1-5}
		\multicolumn{5}{|c|}{\multirow{2}*{$2 \times (3k_{51} - 3k_{41} + 3)$}} &                                                   &                                                                                                                                         \\
		                                                                        &                                                   &                                                                            &   &                                                   &  & \\
		\hline
	\end{tabular}
	\label{fig:5-4-painting}
\end{table}

\begin{itemize}
	\item 当$k_{52} \ge k_{42}$时,有如\ref*{fig:4-5-painting}一样的划分,
	      此时左上角坐标为$(3k_{41} + 1, 3k_{42} - 2)$,右下角坐标为$(3k_{51} + 4, 3k_{52} + 2)$。
	\item 当$k_{52} < k_{42}$时,有如\ref*{fig:5-4-painting}一样的划分,
	      此时左上角坐标为$(3k_{51} + 1, 3k_{52} - 2)$,右下角坐标为$(3k_{41} + 4, 3k_{42} + 2)$。
\end{itemize}

由\ref*{basic-lemma-1}可得,无论如何,所有区块均可被$1 \times 3$的骨牌平铺,也就等价于,这两个缺陷可以组合成一个大小为$(3k_{52} - 3k_{42} + 4) \times (3k_{51} - 3k_{41} + 5)$大小的缺陷,
并且角落坐标满足\ref*{basic-theorem-2}中的条件。因此由\ref*{basic-theorem-2},当缺陷分别处在染色4和染色5的位置时,无论缺陷怎么分布,该缺陷棋盘都可被平铺。

\subsection{在染色中分别处于1和2}

同在染色中处于4和5的情况一样,我们可以写出颜色1和颜色2所在的方格坐标。
\begin{itemize}
	\item 1的坐标: $(3k_{11} + 1, 3k_{12} + 1)$
	\item 2的坐标: $(3k_{21} + 1, 3k_{22} + 2)$
\end{itemize}


由于对称性,在这里我们不妨假设$k_{21} \ge k_{11}$,则同样有以下两种情况:

\begin{table}[b]
	\centering
	\caption{挖去1和2的染色}
	\begin{tabular}{|ccc|c|}
		\hline
		1                                                                                    & \multicolumn{2}{|c|}{$1 \times (3k_{22} -3k_{12})$} & \multirow{2}*{$(3k_{21} - 3k_{11}) \times 1$}     \\
		\cline{1-3}
		\multicolumn{3}{|c|}{\multirow{2}*{$(3k_{21} - 3k_{11}) \times (3k_{22} -3k_{12})$}} &                                                                                                         \\
		\cline{4-4}
		                                                                                     &                                                     &                                               & 2 \\
		\hline
	\end{tabular}
	\label{fig:1-2-painting}
\end{table}

\begin{table}[t]
	\centering
	\caption{挖去2和1的染色-1}
	\begin{tabular}{|cc|ccc|cc|}
		\hline
		\multicolumn{2}{|c|}{\multirow{3}{*}{$(3k_{21} - 3k_{11}) \times 1$}} & 2 & \multicolumn{4}{|c|}{$1 \times  (3k_{12} - 3k_{22})$}                                                                                 \\
		\cline{3-7}
		                                                                      &   & \multicolumn{3}{c}{\multirow{2}{*}{...}}              & \multicolumn{2}{|c|}{\multirow{3}{*}{$(3k_{21} - 3k_{11}) \times 1$}}         \\
		                                                                      &   &                                                       &                                                                       &  &  & \\
		\cline{1-5}
		\multicolumn{4}{|c|}{$1 \times  (3k_{12} - 3k_{22})$}                 & 1 &                                                       &                                                                               \\
		\hline
		\label{fig:2-1-painting-1}
	\end{tabular}
\end{table}

\begin{itemize}
	\item 当$k_{22} \ge k_{12}$时,有如\ref*{fig:1-2-painting}的区域划分,此时左上角坐标为$(3k_{11} + 1, 3k_{12} + 1)$,右下角坐标为$(3k_{21} + 1, 3k_{22} + 2)$。
	\item 当$k_{22} < k_{12}$时,将再次有以下分类
	      \begin{itemize}
		      \item 当缺陷不处于同一行时,即$k_{21} > k_{11}$,有如\ref*{fig:2-1-painting-1}的区域划分,左上角坐标为$(3k_{21} + 1, 3k_{22} + 1)$,右下角坐标为$(3k_{11} + 1, 3k_{22} + 2)$
		      \item 当缺陷处于同一行且不在边界上时,有$m \ge 7$,当$n = 5$时,穷举得到无解,当$n \ge 8$,
		            则当$k_{12} - k_{22} = 1$时,存在区域划分,左上角坐标为$(3k_{21}- 2, 3k_{22} + 1)$,右下角坐标为$(3k_{11} + 1, 3k_{22} + 2)$;
		            当$k_{12} - k_{22} > 1$时,存在区域划分,左上角坐标为$(3k_{21} - 2, 3k_{22} + 1)$,右下角坐标为$(3k_{11} + 4, 3k_{22} + 5)$。
		            两种情况的区域划分我都放在了附录中。
		      \item 当缺陷都处在边界上时,我们称为棘手情况,把它放在下一节讨论。
	      \end{itemize}
\end{itemize}



\section{骨牌大小k > 3时的情况}

\begin{table}[h]
	\caption{k阶染色}
	\label{fig:k-order-staining}
	\begin{subtable}{.5\linewidth}
		\centering
		\caption{k阶染色-1}
		\begin{tabular}{|c|c|c|c|c|c|c|}
			\hline
			1        & 2        & 3        & $\cdots$ & k        & 1        & $\cdots$ \\
			\hline
			2        & 3        & 4        & $\cdots$ & 1        & 2        & $\cdots$ \\
			\hline
			3        & 4        & 5        & $\cdots$ & 2        & 3        & $\cdots$ \\
			\hline
			$\vdots$ & $\vdots$ & $\vdots$ & $\ddots$ & $\ddots$ & $\ddots$ & $\cdots$ \\
			\hline
			k        & 1        & 2        & $\cdots$ & k-1      & k        & $\cdots$ \\
			\hline
			1        & 2        & 3        & $\cdots$ & k        & 1        & $\cdots$ \\
			\hline
			$\vdots$ & $\vdots$ & $\vdots$ & $\ddots$ & $\vdots$ & $\vdots$ & $\ddots$ \\
			\hline
		\end{tabular}
		\label{fig:k-order-staining-1}
	\end{subtable}%
	\begin{subtable}{.5\linewidth}
		\centering
		\caption{k阶染色-2}
		\begin{tabular}{|c|c|c|c|c|c|c|}
			\hline
			1        & 2        & 3        & $\cdots$ & k        & 1        & $\cdots$ \\
			\hline
			k        & 1        & 2        & $\cdots$ & k-1      & k        & $\cdots$ \\
			\hline
			k-1      & k        & 1        & $\cdots$ & k-2      & k-1      & $\cdots$ \\
			\hline
			$\vdots$ & $\vdots$ & $\vdots$ & $\ddots$ & $\ddots$ & $\ddots$ & $\cdots$ \\
			\hline
			2        & 3        & 4        & $\cdots$ & 1        & 2        & $\cdots$ \\
			\hline
			1        & 2        & 3        & $\cdots$ & k        & 1        & $\cdots$ \\
			\hline
			$\vdots$ & $\vdots$ & $\vdots$ & $\ddots$ & $\vdots$ & $\vdots$ & $\ddots$ \\
			\hline
		\end{tabular}
		\label{fig:k-order-staining-2}
	\end{subtable}
\end{table}


在$k \ge 4$的情况下,我们仍然假设$m \equiv 1 (mod k), n \equiv 2 (mod k)$,我们可以给出以下引理。
\begin{theorem}
	\label{basic-theorem-3}
	当$m \times n$的缺陷两格广义棋盘可被$1 \times k$大小的骨牌平铺时,若$k \ge 4$,则缺陷格子分别只可能在坐标$(kx_1 + 1, ky_1 + 1)$和$(kx_2 + 1, ky_2 + 2)$处。
\end{theorem}
\begin{proof}
	我们不妨令坐标为$(1, j)$的格子染色为颜色$j \% k$,记$Q_p(i, j) = a$, 意味着在坐标为$(i, j)$在染色方案p下被染色为$a,$,$f(x)$为染色为x的格子数目,
	我们能够很轻易的得到$f(1) = f(2) = f(x), x \neq 1, x \neq 2$,
	因此我们只能挖去在所有染色组合中,均被染色为1或染色2的格子,即$Q_p(i, j) = 1 or 2, \forall p$,
	而$Q_p(kx_1 + 1, ky_1 + 1) = 1, \forall p$,$Q_p(kx_2 + 1, ky_2 + 2) = 1, \forall p$,
	因此缺陷格子分别落在坐标$(kx_1 + 1, ky_1 + 1)$和$(kx_2 + 1, ky_2 + 2)$处的格子时,棋盘可能可以被平铺。

	接下来我们证明,当$(i, j) \neq (kx_1 + 1, ky_1 + 1), (i, j) \neq (kx_2 + 1, ky_2 + 2)$时,$\exists p, s.t. Q_p(i, j) \neq 1 or 2$。

	挑选k阶染色中最经典的两种染色\ref{fig:k-order-staining-1}和\ref{fig:k-order-staining-2},考虑最左上的$k \times k$大小的区域,做出以下区域划分:

	\begin{table}[ht]
		\centering
		\begin{tabular}{|c|c|}
			\hline
			A: $1 \times 2$       & B: $1  \times (k - 2)$     \\
			\hline
			C: $(k - 1) \times 2$ & D: $(k - 1)\times (k - 2)$ \\
			\hline
		\end{tabular}
	\end{table}

	记$Q(i, j) = {a, b}$,意味着在\ref{fig:k-order-staining-1}和\ref{fig:k-order-staining-2}分别染色为$a, b$,除去模块A,分别考虑剩下3个模块:
	\begin{itemize}
		\item[模块B] 由于模块B不会染色为1或2,所以缺陷格子不可能落在模块B中。
		\item[模块C] $Q(2, 1) = \left\{2, k\right\}, Q(2, 2) = \left\{3, 1\right\}, Q(2, k) = \left\{k, 1\right\}, Q(k,2) = \left\{1, 3\right\}$
			所以对于模块C中的所有格子,均不可能满足$Q(i, j) = \left\{1 or 2, 1 or 2\right\}$。
		\item[模块D] 假设$\exists i, j, s.t. Q_1(i, j) = 1$,在第$i$行,第$2$列到第$j$列的元素中,排除染色为1或者2的元素,还剩下$k - 1 - 2 = k - 3\ge 1$个元素,
			故$\exists j^{'}, s.t. Q_1(i, j^{'}) = t, t \neq 1 or 2$,此时交换第$j$行和第$j^{'}$行,仍然是一个满足条件的染色,但是在这个染色中,$Q_p(i, j) = Q_1(i, j^{'}) = t \neq 1 or 2$,矛盾。
	\end{itemize}
	因此缺陷格子只能落在模块A中,证毕。
\end{proof}

因此类似$k=3$的情况,我们只需考虑像这样的染色

\begin{table}[h]
	\centering
	\caption{k阶重染色}
	\begin{tabular}{|c|c|c|c|c|c|c|c|}
		\hline
		1        & 2        & 0        & $\cdots$ & 0        & 1        & 2        & $\cdots$ \\
		\hline
		0        & 0        & 0        & $\cdots$ & 0        & 0        & 0        & $\cdots$ \\
		\hline
		0        & 0        & 0        & $\cdots$ & 0        & 0        & 0        & $\cdots$ \\
		\hline
		$\vdots$ & $\vdots$ & $\vdots$ & $\ddots$ & $\vdots$ & $\vdots$ & $\vdots$ & $\ddots$ \\
		\hline
		0        & 0        & 0        & $\cdots$ & 0        & 0        & 0        & $\cdots$ \\
		\hline
		1        & 2        & 0        & $\cdots$ & 0        & 1        & 2        & $\cdots$ \\
		\hline
		0        & 0        & 0        & $\cdots$ & 0        & 0        & 0        & $\cdots$ \\
		\hline
		$\vdots$ & $\vdots$ & $\vdots$ & $\ddots$ & $\vdots$ & $\vdots$ & $\vdots$ & $\ddots$ \\
		\hline
	\end{tabular}
	\label{fig:k-order-staining-last}
\end{table}

同$k=3$的情况,我们只需要考虑染色分别处于1和2的情况,此时可以完全的套用$k=3$的情况,且棘手情况也一样,为$k>3$时的边界情况,我们在下一节一起讨论。

\clearpage