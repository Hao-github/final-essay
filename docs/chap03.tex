\chapter{缺陷格子可能存在的位置与相对应的平铺方案}

\section{骨牌大小k = 3时的情况}
由\ref*{basic-theorem-1}有,此时棋盘的大小只可能为$(3a + 1) \times (3b + 2)$。

在图\ref*{fig:k-order-staining-1}的基础上,我们对棋盘进行两种染色,即

\begin{table}[h]
	\caption{3阶染色}
	\label{fig:3-order-staining}
	\begin{subtable}{.5\linewidth}
		\centering
		\caption{3阶染色-1}
		\begin{tabular}{|c|c|c|c|c|c|}
			\hline
			1        & 2        & 3        & 1        & 2        & $\cdots$ \\
			\hline
			2        & 3        & 1        & 2        & 3        & $\cdots$ \\
			\hline
			3        & 1        & 2        & 3        & 1        & $\cdots$ \\
			\hline
			1        & 2        & 3        & 1        & 2        & $\cdots$ \\
			\hline
			2        & 3        & 1        & 2        & 3        & $\cdots$ \\
			\hline
			$\vdots$ & $\vdots$ & $\vdots$ & $\vdots$ & $\vdots$ & $\ddots$ \\
			\hline
		\end{tabular}
		\label{fig:3-order-staining-1}
	\end{subtable}%
	\begin{subtable}{.5\linewidth}
		\centering
		\caption{3阶染色-2}
		\begin{tabular}{|c|c|c|c|c|c|}
			\hline
			1        & 2        & 3        & 1        & 2        & $\cdots$ \\
			\hline
			3        & 1        & 2        & 3        & 1        & $\cdots$ \\
			\hline
			2        & 3        & 1        & 2        & 3        & $\cdots$ \\
			\hline
			1        & 2        & 3        & 1        & 2        & $\cdots$ \\
			\hline
			3        & 1        & 2        & 3        & 1        & $\cdots$ \\
			\hline
			$\vdots$ & $\vdots$ & $\vdots$ & $\vdots$ & $\vdots$ & $\ddots$ \\
			\hline
		\end{tabular}
		\label{fig:3-order-staining-2}
	\end{subtable}
\end{table}

不妨设棋盘行数$m \equiv 1 (mod k)$,列数$n \equiv 2 (mod k)$,记$f(x)$为染色为x的格子的数目,显然有$f(1) = f(2) = f(3) + 1$。

因此该棋盘可被平铺的充分条件为缺陷的两个格子分别落在染色1和染色2的位置,此时我们将两张图合并,并以以下规则重新染色:
\begin{itemize}
	\item 在\ref*{fig:3-order-staining-1}和\ref*{fig:3-order-staining-2}均染色为1的格子,在新图染色为1
	\item 在\ref*{fig:3-order-staining-1}和\ref*{fig:3-order-staining-2}均染色为2的格子,在新图染色为2
	\item 在\ref*{fig:3-order-staining-1}染色为1,\ref*{fig:3-order-staining-2}染色为2的格子,在新图染色为4
	\item 在\ref*{fig:3-order-staining-1}染色为2,\ref*{fig:3-order-staining-2}染色为1的格子,在新图染色为5
	\item 在\ref*{fig:3-order-staining-1}或\ref*{fig:3-order-staining-2}染色为3的格子,在新图染色为0
\end{itemize}

在此条件下重新染色后,棋盘为

\begin{table}[h]
	\centering
	\caption{3阶重染色}
	\begin{tabular}{|c|c|c|c|c|c|c|}
		\hline
		1        & 2        & 0        & 1        & 2        & 0        & $\cdots$ \\
		\hline
		0        & 0        & 4        & 0        & 0        & 4        & $\cdots$ \\
		\hline
		0        & 0        & 5        & 0        & 0        & 5        & $\cdots$ \\
		\hline
		1        & 2        & 0        & 1        & 2        & 0        & $\cdots$ \\
		\hline
		0        & 0        & 4        & 0        & 0        & 4        & $\cdots$ \\
		\hline
		0        & 0        & 5        & 0        & 0        & 5        & $\cdots$ \\
		\hline
		$\vdots$ & $\vdots$ & $\vdots$ & $\vdots$ & $\vdots$ & $\vdots$ & $\ddots$ \\
		\hline
	\end{tabular}
	\label{fig:3-order-staining-last}
\end{table}

由于在所有染色方案中,缺陷的格子都需要落在染色1和染色2上,因此我们可以进一步排除缺陷格子的可能所在的位置
\begin{itemize}
	\item 当一个缺陷格子在\ref*{fig:3-order-staining-1}和\ref*{fig:3-order-staining-2}都染色为1,
	      另一个缺陷格子在\ref*{fig:3-order-staining-1}和\ref*{fig:3-order-staining-2}都染色为2时,他们将分别在\ref*{fig:3-order-staining-last}中染色为1和染色为2,因此如果
	      该棋盘可以被平铺,则缺陷格子必然分别落在\ref*{fig:3-order-staining-last}的染色1和染色2处。

	\item 当一个缺陷格子在\ref*{fig:3-order-staining-1}染色为1,在\ref*{fig:3-order-staining-2}染色为2,
	      另一个缺陷格子在\ref*{fig:3-order-staining-1}染色为2,\ref*{fig:3-order-staining-2}染色为1时,他们将分别在\ref*{fig:3-order-staining-last}中染色为4和染色为5,因此如果
	      该棋盘可以被平铺,则缺陷格子必然分别落在\ref*{fig:3-order-staining-last}的染色4和染色5处。
\end{itemize}
如此我们便只需要考虑两种情况。

在这里我们给出一个定理,来辅助我们之后的证明
\begin{theorem}
	\label{basic-theorem-2}
	对于$m \times n$的广义棋盘和$1 \times k$的骨牌,且$m \equiv 1 (mod k), n \equiv 2 (mod k)$,
	记某格子的坐标为$(i, j)$,意味着某格子处于第i行第j列,则如果存在$i \times j$的缺陷C,$i \equiv 1 (mod k), j \equiv 2 (mod k), i \le m, j \le n$,
	且棋盘的左上角和右下角的格子坐标分别为$(3k_{11} + 1, 3k_{12} + 1)$,$(3k_{21} + 1, 3k_{22} + 2)$,
	则该棋盘一定能被平铺
\end{theorem}
\begin{proof}
	当存在这样的缺陷时,我们可以把棋盘分成如\ref*{fig:nine-separate}的9个区域,分别如图所示,由\ref*{basic-lemma-1},可得除了缺陷C外的8个区域均可被$1 \times k$的骨牌平铺,
	而中央的缺陷C不需要被平铺,故此时整个棋盘已经可以被平铺,证毕。

	\begin{table}[h]
		\centering
		\caption{对棋盘的9划分}
		\begin{tabular}{|c|c|c|}

			\hline
			$3k_{12} \times 3k_{11}$               & $(3k_{22} - 3k_{12} + 2)\times 3k_{11}$  & $3k_{22} \times 3k_{11}$                   \\
			\hline
			$3k_{12} \times (3k_{21}-3k_{11} + 1)$ & $C: i \times j$                          & $3k_{22} \times (3k_{21}-3k_{11} + 1 + 1)$ \\
			\hline
			$3k_{12} \times 3k_{21}$               & $3 (k_{22} - k_{12}) + 2 \times 3k_{21}$ & $3k_{22} \times 3k_{21}$                   \\
			\hline
		\end{tabular}
		\label{fig:nine-separate}
	\end{table}
\end{proof}



因此根据\ref*{basic-theorem-2},我们可以得把问题转化为:对于某个棋盘,他的两个缺陷可以组合成一个满足\ref*{basic-theorem-2}中条件的$i \times j$缺陷,那么该棋盘可以被平铺。
\subsection{在染色中分别处于4和5}

由图,我们可以简单的写出4和5所对应的格子的坐标,

\begin{itemize}
	\item 4的坐标: $(3k_{41} + 2, 3k_{42})$
	\item 5的坐标: $(3k_{51} + 3, 3k_{52})$
\end{itemize}

在这里我们不妨假设$k_{51} \ge k_{41}$,(否则我们只需要左右翻转棋盘即可),则有以下2种情况:



\begin{table}[ht]
	\centering
	\caption{挖去4和5的染色-1}
	\begin{tabular}{|cc|c|cc|c|cc|}
		\hline
		\multicolumn{6}{|c|}{$1 \times (3k_{51} - 3k_{41} + 3)$}                & \multicolumn{2}{|c|}{\multirow{4}*{$(3k_{52} - 3k_{42} + 3) \times 2$}}                                                                                        \\
		\cline{1-6}
		\multicolumn{2}{|c|}{\multirow{4}*{$(3k_{52} - 3k_{42} + 3) \times 2$}} & 4                                                                       & \multicolumn{3}{|c|}{$1 \times (3k_{51} - 3k_{41})$}                       &   &     \\
		\cline{3-6}
		                                                                        &                                                                         & \multicolumn{4}{|c|}{$(3k_{52} - 3k_{42}) \times (3k_{51} - 3k_{41} + 1)$} &   &     \\
		\cline{3-6}
		                                                                        &                                                                         & \multicolumn{3}{|c|}{$1 \times (3k_{51} - 3k_{41})$}                       & 5 &   & \\
		\cline{3-8}
		                                                                        &                                                                         & \multicolumn{6}{|c|}{$1 \times (3k_{51} - 3k_{41} + 3)$}                             \\
		\hline
	\end{tabular}
	\label{fig:4-5-painting}
\end{table}

\begin{table}[ht]
	\centering

	\caption{挖去4和5的染色-2}
	\begin{tabular}{|ccccc|cc|}
		\hline
		\multicolumn{5}{|c|}{\multirow{2}*{$2 \times (3k_{51} - 3k_{41} + 3)$}} & \multicolumn{2}{|c|}{\multirow{3}*{$3 \times 2$}}                                                                                                                                           \\
		                                                                        &                                                   &                                                                            &   &                                                   &  & \\
		\cline{1-5}
		\multicolumn{2}{|c|}{\multirow{3}*{$2 \times (3k_{42} - 3k_{52} + 3)$}} & 5                                                 & \multicolumn{2}{|c|}{$1 \times (3k_{51} - 3k_{41})$}                       &   &                                                        \\
		\cline{3-7}
		                                                                        &                                                   & \multicolumn{5}{|c|}{$(3k_{42} - 3k_{52}) \times (3k_{51} - 3k_{41} + 3)$}                                                              \\
		\cline{3-7}
		                                                                        &                                                   & \multicolumn{2}{|c|}{$1 \times (3k_{51} - 3k_{41})$}                       & 4 & \multicolumn{2}{|c|}{\multirow{3}*{$3 \times 2$}}      \\
		\cline{1-5}
		\multicolumn{5}{|c|}{\multirow{2}*{$2 \times (3k_{51} - 3k_{41} + 3)$}} &                                                   &                                                                                                                                         \\
		                                                                        &                                                   &                                                                            &   &                                                   &  & \\
		\hline
	\end{tabular}
	\label{fig:5-4-painting}
\end{table}

\begin{itemize}

	\item 当$k_{52} \ge k_{42}$时,有如\ref*{fig:4-5-painting}一样的划分,
	      此时左上角坐标为$(3k_{41} + 1, 3k_{42} - 2)$,右下角坐标为$(3k_{51} + 4, 3k_{52} + 2)$。
	\item 当$k_{52} < k_{42}$时,有如\ref*{fig:5-4-painting}一样的划分,
	      此时左上角坐标为$(3k_{51} + 1, 3k_{52} - 2)$,右下角坐标为$(3k_{41} + 4, 3k_{42} + 2)$。
\end{itemize}

由\ref*{basic-lemma-1}可得,无论如何,所有区块均可被$1 \times 3$的骨牌平铺,也就等价于,这两个缺陷可以组合成一个大小为$(3k{52} - 3k_{42} + 4) \times (3k_{51} - 3k_{41} + 5)$大小的缺陷,
由\ref*{basic-theorem-2},当缺陷分别处在染色4和染色5的位置时,该棋盘可被平铺。

\subsection{在染色中分别处于1和2}
同上述情况,我们可以写出颜色1和颜色2所在的方格坐标。
\begin{itemize}
	\item 1的坐标: $(3k_{11} + 1, 3k_{12} + 1)$
	\item 2的坐标: $(3k_{21} + 1, 3k_{22} + 2)$
\end{itemize}


在这里我们不妨假设$k_{21} \ge k_{11}$(否则只需要简单的上下翻转棋盘即可),则同样有以下两种情况:

\begin{table}[b]
	\centering
	\caption{挖去1和2的染色}
	\begin{tabular}{|ccc|c|}
		\hline
		1                                                                                    & \multicolumn{2}{|c|}{$1 \times (3k_{22} -3k_{12})$} & \multirow{2}*{$(3k_{21} - 3k_{11}) \times 1$}     \\
		\cline{1-3}
		\multicolumn{3}{|c|}{\multirow{2}*{$(3k_{21} - 3k_{11}) \times (3k_{22} -3k_{12})$}} &                                                                                                         \\
		\cline{4-4}
		                                                                                     &                                                     &                                               & 2 \\
		\hline
	\end{tabular}
	\label{fig:1-2-painting}
\end{table}

\begin{table}[t]
	\centering
	\caption{挖去2和1的染色-1}
	\begin{tabular}{|cc|ccc|cc|}
		\hline
		\multicolumn{2}{|c|}{\multirow{3}{*}{$(3k_{21} - 3k_{11}) \times 1$}} & 2 & \multicolumn{4}{|c|}{$1 \times  (3k_{12} - 3k_{22})$}                                                                                 \\
		\cline{3-7}
																			  &   & \multicolumn{3}{|c|}{\multirow{2}{*}{...}}            & \multicolumn{2}{|c|}{\multirow{3}{*}{$(3k_{21} - 3k_{11}) \times 1$}}         \\
																			  &   &                                                       &                                                                       &  &  & \\
		\cline{1-5}
		\multicolumn{4}{|c|}{$1 \times  (3k_{12} - 3k_{22})$}                 & 1 &                                                       &                                                                               \\
		\hline
	\label{fig:2-1-painting-1}
	\end{tabular}
\end{table}

\begin{itemize}
	\item 当$k_{22} \ge k_{12}$时,有如\ref*{fig:1-2-painting}的区域划分,此时左上角坐标为$(3k_{11} + 1, 3k_{12} + 1)$,右下角坐标为$(3k_{21} + 1, 3k_{22} + 2)$。
	\item 当$k_{22} < k_{12}$时,将再次有以下分类
	      \begin{itemize}
		      \item 当不处于同一行时,即$k_{21} > k_{11}$,有如\ref*{fig:2-1-painting-1}的区域划分,左上角坐标为$(3k_{21} + 1, 3k_{22} + 1)$,右下角坐标为$(3k_{11} + 1, 3k_{22} + 2)$
		      \item 当处于同一行且不在边界上时,有以下区域划分,由\ref*{basic-lemma-1}得,此时棋盘可被平铺
		      \item 当都处在边界上时,我们称为棘手情况,把它放在下一节讨论
	      \end{itemize}
\end{itemize}



\section{骨牌大小k > 3时的情况}

\begin{table}[h]
	\caption{k阶染色}
	\label{fig:k-order-staining}
	\begin{subtable}{.5\linewidth}
		\centering
		\caption{k阶正染色}
		\begin{tabular}{|c|c|c|c|c|c|c|}
			\hline
			1        & 2        & 3        & $\cdots$ & k        & 1        & $\cdots$ \\
			\hline
			2        & 3        & 4        & $\cdots$ & 1        & 2        & $\cdots$ \\
			\hline
			3        & 4        & 5        & $\cdots$ & 2        & 3        & $\cdots$ \\
			\hline
			$\vdots$ & $\vdots$ & $\vdots$ & $\ddots$ & $\ddots$ & $\ddots$ & $\cdots$ \\
			\hline
			k        & 1        & 2        & $\cdots$ & k-1      & k        & $\cdots$ \\
			\hline
			1        & 2        & 3        & $\cdots$ & k        & 1        & $\cdots$ \\
			\hline
			$\vdots$ & $\vdots$ & $\vdots$ & $\ddots$ & $\vdots$ & $\vdots$ & $\ddots$ \\
			\hline
		\end{tabular}
		\label{fig:k-order-staining-1}
	\end{subtable}%
	\begin{subtable}{.5\linewidth}
		\centering
		\caption{k阶染色-逆}
		\begin{tabular}{|c|c|c|c|c|c|c|}
			\hline
			1        & 2        & 3        & $\cdots$ & k        & 1        & $\cdots$ \\
			\hline
			k        & 1        & 2        & $\cdots$ & k-1      & k        & $\cdots$ \\
			\hline
			k-1      & k        & 1        & $\cdots$ & k-2      & k-1      & $\cdots$ \\
			\hline
			$\vdots$ & $\vdots$ & $\vdots$ & $\ddots$ & $\ddots$ & $\ddots$ & $\cdots$ \\
			\hline
			2        & 3        & 4        & $\cdots$ & 1        & 2        & $\cdots$ \\
			\hline
			1        & 2        & 3        & $\cdots$ & k        & 1        & $\cdots$ \\
			\hline
			$\vdots$ & $\vdots$ & $\vdots$ & $\ddots$ & $\vdots$ & $\vdots$ & $\ddots$ \\
			\hline
		\end{tabular}
		\label{fig:k-order-staining-2}
	\end{subtable}
\end{table}


在k>4的情况下,我们仍然假设$m \equiv 1 (mod k), n \equiv 2 (mod k)$,我们可以给出以下引理。
\begin{theorem}
	\label{basic-theorem-3}
	当$m \times n$的缺陷两格广义棋盘可被$1 \times k$大小的骨牌平铺时,若$k \ge 4$,则缺陷格子分别只可能在坐标$(kx_1 + 1, ky_1 + 1)$和$(kx_2 + 1, ky_2 + 2)$处。
\end{theorem}
\begin{proof}
	我们不妨令坐标为$(1, 1)$和坐标为$(1, 2)$的格子染色为颜色1和颜色2,由引理?我们可以得知,我们只能挖去在所有染色组合中,均被染色为1或染色2的格子,再由定理?得知,坐标$(kx_1 + 1, ky_1 + 1)$和$(kx_2 + 1, ky_2 + 2)$处的格子在所有染色情况下,均会被染色成颜色1和颜色2,缺陷格子可以坐标$(kx_1 + 1, ky_1 + 1)$和$(kx_2 + 1, ky_2 + 2)$处。

	接下来我们证明,当坐标不是$(kx_1 + 1, ky_1 + 1)$和$(kx_2 + 1, ky_2 + 2)$时,均存在某种染色方案,使得该坐标染色既不是1也不是2,则定理得证。



	基于\ref{fig:k-order-staining-1},不妨把$(1, j)$染色为j, $(1, j), j \neq 1 or 2$时,该格子已经不是染色为1或2,故第一行的已排除。
	对于第1列和第2列且不为第一行时,当格子在\ref{fig:k-order-staining-1}中为2时,在\ref{fig:k-order-staining-2}为k,当格子在\ref{fig:k-order-staining-1}为1时,在\ref{fig:k-order-staining-2}为3,故第一列和第二列也已排除。

	对于$(i, j), i > 1, j > 2$,当格子在\ref{fig:k-order-staining-1}中为1时,由于$k \ge 4$,我们考虑第j列的元素,故$[2, k]$共计k-1个元素中,排除染色为1或者2的元素,还剩下$k - 1 - 2 = k-3$个元素,由于$k \ge 4$,故$k-3 \ge 1$,至少存在一个元素满足条件,染色为非1或2,且不在第一行,记为第k个元素。交换第i行和第j行后,这种染色也是满足条件的,但是在新染色中,$(i,j)$染色既不为1也不为2,故$(i, j)$也不能是格子。

	证毕。

\end{proof}


因此类似k=3的情况,我们只需考虑像这样的染色

\begin{table}[h]
	\centering
	\caption{k阶重染色}
	\begin{tabular}{|c|c|c|c|c|c|c|c|}
		\hline
		1        & 2        & 0        & $\cdots$ & 0        & 1        & 2        & $\cdots$ \\
		\hline
		0        & 0        & 0        & $\cdots$ & 0        & 0        & 0        & $\cdots$ \\
		\hline
		0        & 0        & 0        & $\cdots$ & 0        & 0        & 0        & $\cdots$ \\
		\hline
		$\vdots$ & $\vdots$ & $\vdots$ & $\ddots$ & $\vdots$ & $\vdots$ & $\vdots$ & $\ddots$ \\
		\hline
		0        & 0        & 0        & $\cdots$ & 0        & 0        & 0        & $\cdots$ \\
		\hline
		1        & 2        & 0        & $\cdots$ & 0        & 1        & 2        & $\cdots$ \\
		\hline
		0        & 0        & 0        & $\cdots$ & 0        & 0        & 0        & $\cdots$ \\
		\hline
		$\vdots$ & $\vdots$ & $\vdots$ & $\ddots$ & $\vdots$ & $\vdots$ & $\vdots$ & $\ddots$ \\
		\hline
	\end{tabular}
	\label{fig:k-order-staining-last}
\end{table}

同k=3的情况,我们只需要考虑染色分别处于1和2的情况,此时可以完全的套用k=3的情况,且棘手情况也一样,为$k>3$时的边界情况,我们在下一节一起讨论。

\clearpage