\chapter{缺陷格子可能存在的位置与相对应的平铺方案}

\section{骨牌大小k = 3时的情况}
由\ref*{basic-theorem-1}有,此时棋盘的大小只可能为$(3a + 1) \times (3b + 2)$。

在图\ref*{fig:k-order-staining-1}的基础上,我们对棋盘进行两种染色,即

\begin{table}[h]
	\caption{3阶染色}
	\label{fig:3-order-staining}
	\begin{subtable}{.5\linewidth}
		\centering
		\caption{3阶染色-1}
		\begin{tabular}{|c|c|c|c|c|c|}
			\hline
			1        & 2        & 3        & 1        & 2        & $\cdots$ \\
			\hline
			2        & 3        & 1        & 2        & 3        & $\cdots$ \\
			\hline
			3        & 1        & 2        & 3        & 1        & $\cdots$ \\
			\hline
			1        & 2        & 3        & 1        & 2        & $\cdots$ \\
			\hline
			2        & 3        & 1        & 2        & 3        & $\cdots$ \\
			\hline
			$\vdots$ & $\vdots$ & $\vdots$ & $\vdots$ & $\vdots$ & $\ddots$ \\
			\hline
		\end{tabular}
		\label{fig:3-order-staining-1}
	\end{subtable}%
	\begin{subtable}{.5\linewidth}
		\centering
		\caption{3阶染色-2}
		\begin{tabular}{|c|c|c|c|c|c|}
			\hline
			1        & 2        & 3        & 1        & 2        & $\cdots$ \\
			\hline
			3        & 1        & 2        & 3        & 1        & $\cdots$ \\
			\hline
			2        & 3        & 1        & 2        & 3        & $\cdots$ \\
			\hline
			1        & 2        & 3        & 1        & 2        & $\cdots$ \\
			\hline
			3        & 1        & 2        & 3        & 1        & $\cdots$ \\
			\hline
			$\vdots$ & $\vdots$ & $\vdots$ & $\vdots$ & $\vdots$ & $\ddots$ \\
			\hline
		\end{tabular}
		\label{fig:3-order-staining-2}
	\end{subtable}
\end{table}

不妨设棋盘行数$m \equiv 1 (mod k)$,列数$n \equiv 2 (mod k)$,显然有$f(1) = f(2) = f(3) + 1$。
故缺陷格子有且只有可能分别染色为1和染色为2,此时我们将两张图合并,并以以下规则重新染色:
\begin{itemize}
	\item 在\ref*{fig:3-order-staining-1}和\ref*{fig:3-order-staining-2}均染色为1的格子,在新图染色为1
	\item 在\ref*{fig:3-order-staining-1}和\ref*{fig:3-order-staining-2}均染色为2的格子,在新图染色为2
	\item 在\ref*{fig:3-order-staining-1}染色为1,\ref*{fig:3-order-staining-2}染色为2的格子,在新图染色为4
	\item 在\ref*{fig:3-order-staining-1}染色为2,\ref*{fig:3-order-staining-2}染色为1的格子,在新图染色为5
	\item 在\ref*{fig:3-order-staining-1}或\ref*{fig:3-order-staining-2}染色为3的格子,在新图染色为0
\end{itemize}

在此条件下重新染色后,棋盘为

\begin{table}[h]
	\centering
	\caption{3阶重染色}
	\begin{tabular}{|c|c|c|c|c|c|c|}
		\hline
		1        & 2        & 0        & 1        & 2        & 0        & $\cdots$ \\
		\hline
		0        & 0        & 4        & 0        & 0        & 4        & $\cdots$ \\
		\hline
		0        & 0        & 5        & 0        & 0        & 5        & $\cdots$ \\
		\hline
		1        & 2        & 0        & 1        & 2        & 0        & $\cdots$ \\
		\hline
		0        & 0        & 4        & 0        & 0        & 4        & $\cdots$ \\
		\hline
		0        & 0        & 5        & 0        & 0        & 5        & $\cdots$ \\
		\hline
		$\vdots$ & $\vdots$ & $\vdots$ & $\vdots$ & $\vdots$ & $\vdots$ & $\ddots$ \\
		\hline
	\end{tabular}
	\label{fig:3-order-staining-last}
\end{table}

由于缺陷格子分别只能染色1和染色2,故在两种染色下缺陷格子都能需要处在颜色1或颜色2中,且必须分别在颜色1和颜色2,
因此缺陷格子的可能性就缩小为以下2种情况之一
\begin{itemize}
	\item 分别处在\ref*{fig:k-order-staining-last}的1和2
	\item 分别处在\ref*{fig:k-order-staining-last}的4和5
\end{itemize}

在这里我们给出一个定理,来辅助我们之后的证明
\begin{theorem}
	\label{basic-theorem-2}
	对于$m \times n$的广义棋盘和$1 \times k$的骨牌,且$m \equiv 1 (mod k), n \equiv 2 (mod k)$,
	如果存在$i \times j$的缺陷,$i \equiv 1 (mod k), j \equiv 2 (mod k), i \le m, j \le n$,
	则该棋盘一定能被平铺
\end{theorem}
(待补充证明)

由此定理,我们可以得到,当可以把两个缺陷组合成一个$i \times j$的缺陷,$i \equiv 1 (mod k), j \equiv 2 (mod k)$,那么$\forall m > i, \forall n > j$,$m \times n$的广义棋盘
都可以被平铺。

\subsection{在染色中分别处于4和5}

由图,我们可以简单的写出4和5所对应的格子的坐标,

\begin{itemize}
	\item 4的坐标: $(3k_{41}, 3k_{42} + 2)$
	\item 5的坐标: $(3k_{51}, 3k_{52} + 3)$
\end{itemize}

在这里我们直接给出$k_{41}=k_{51}, k_{42} = k_{52}$和$k_{41}=k_{51}, k_{42} = k_{52} + 1$时的平铺方案

\begin{table}[h]
	\caption{4-5染色}
	\label{fig:3-order-staining-123}
	\begin{subtable}{.5\linewidth}
		\centering
		\caption{4-5染色}
		\centering
		\begin{tabular}{|c|c|c|c|c|}
			\hline
			\multicolumn{3}{|c|}{} & \multirow{3}*{} & \multirow{3}*{}             \\
			\cline{1-3}
			\multirow{3}*          & \multirow{3}*   & 4                      &  & \\
			\cline{3-3}
			                       &                 & 5                      &  & \\
			\cline{3-5}
			                       &                 & \multicolumn{3}{|c|}{}      \\
			\hline
		\end{tabular}
		\label{fig:3-order-staining-12}
	\end{subtable}%
	\begin{subtable}{.5\linewidth}
		\centering
		\caption{5-4染色}
		\begin{tabular}{|c|c|c|c|c|}
			\hline
			\multicolumn{3}{|c|}{} & \multirow{3}*{} & \multirow{3}*{}                                            \\
			\cline{1-3}
			\multicolumn{3}{|c|}{} &                 &                                                            \\
			\cline{1-3}
			\multirow{3}*{}        & \multirow{3}*{} & 5                      &                 &                 \\
			\cline{3-5}
			                       &                 & \multicolumn{3}{|c|}{}                                     \\
			\cline{3-5}
			                       &                 & 4                      & \multirow{3}*{} & \multirow{3}*{} \\
			\cline{1-3}
			\multicolumn{3}{|c|}{} &                 &                                                            \\
			\cline{1-3}
			\multicolumn{3}{|c|}{} &                 &                                                            \\
			\hline
		\end{tabular}
		\label{fig:3-order-staining-22}
	\end{subtable}
\end{table}

\section{骨牌大小k > 3时的情况}

\begin{table}[h]
	\caption{k阶染色}
	\label{fig:k-order-staining}
	\begin{subtable}{.5\linewidth}
		\centering
		\caption{k阶正染色}
		\begin{tabular}{|c|c|c|c|c|c|c|}
			\hline
			1        & 2        & 3        & $\cdots$ & k        & 1        & $\cdots$ \\
			\hline
			2        & 3        & 4        & $\cdots$ & 1        & 2        & $\cdots$ \\
			\hline
			3        & 4        & 5        & $\cdots$ & 2        & 3        & $\cdots$ \\
			\hline
			$\vdots$ & $\vdots$ & $\vdots$ & $\ddots$ & $\ddots$ & $\ddots$ & $\cdots$ \\
			\hline
			k        & 1        & 2        & $\cdots$ & k-1      & k        & $\cdots$ \\
			\hline
			1        & 2        & 3        & $\cdots$ & k        & 1        & $\cdots$ \\
			\hline
			$\vdots$ & $\vdots$ & $\vdots$ & $\ddots$ & $\vdots$ & $\vdots$ & $\ddots$ \\
			\hline
		\end{tabular}
		\label{fig:k-order-staining-1}
	\end{subtable}%
	\begin{subtable}{.5\linewidth}
		\centering
		\caption{k阶染色-逆}
		\begin{tabular}{|c|c|c|c|c|c|c|}
			\hline
			1        & 2        & 3        & $\cdots$ & k        & 1        & $\cdots$ \\
			\hline
			k        & 1        & 2        & $\cdots$ & k-1      & k        & $\cdots$ \\
			\hline
			k-1      & k        & 1        & $\cdots$ & k-2      & k-1      & $\cdots$ \\
			\hline
			$\vdots$ & $\vdots$ & $\vdots$ & $\ddots$ & $\ddots$ & $\ddots$ & $\cdots$ \\
			\hline
			2        & 3        & 4        & $\cdots$ & 1        & 2        & $\cdots$ \\
			\hline
			1        & 2        & 3        & $\cdots$ & k        & 1        & $\cdots$ \\
			\hline
			$\vdots$ & $\vdots$ & $\vdots$ & $\ddots$ & $\vdots$ & $\vdots$ & $\ddots$ \\
			\hline
		\end{tabular}
		\label{fig:k-order-staining-2}
	\end{subtable}
\end{table}


在k>4的情况下,我们仍然假设$m \equiv 1 (mod k), n \equiv 2 (mod k)$,我们可以给出以下引理。
\begin{theorem}
	\label{basic-theorem-3}
	当$m \times n$的缺陷两格广义棋盘可被$1 \times k$大小的骨牌平铺时,若$k \ge 4$,则缺陷格子分别只可能在坐标$(kx_1 + 1, ky_1 + 1)$和$(kx_2 + 1, ky_2 + 2)$处。
\end{theorem}
\begin{proof}
	我们不妨令坐标为$(1, 1)$和坐标为$(1, 2)$的格子染色为颜色1和颜色2,由引理?我们可以得知,我们只能挖去在所有染色组合中,均被染色为1或染色2的格子,再由定理?得知,坐标$(kx_1 + 1, ky_1 + 1)$和$(kx_2 + 1, ky_2 + 2)$处的格子在所有染色情况下,均会被染色成颜色1和颜色2,缺陷格子可以坐标$(kx_1 + 1, ky_1 + 1)$和$(kx_2 + 1, ky_2 + 2)$处。

	接下来我们证明,当坐标不是$(kx_1 + 1, ky_1 + 1)$和$(kx_2 + 1, ky_2 + 2)$时,均存在某种染色方案,使得该坐标染色既不是1也不是2,则定理得证。



	基于\ref{fig:k-order-staining-1},不妨把$(1, j)$染色为j, $(1, j), j \neq 1 or 2$时,该格子已经不是染色为1或2,故第一行的已排除。
	对于第1列和第2列且不为第一行时,当格子在\ref{fig:k-order-staining-1}中为2时,在\ref{fig:k-order-staining-2}为k,当格子在\ref{fig:k-order-staining-1}为1时,在\ref{fig:k-order-staining-2}为3,故第一列和第二列也已排除。

	对于$(i, j), i > 1, j > 2$,当格子在\ref{fig:k-order-staining-1}中为1时,由于$k \ge 4$,我们考虑第j列的元素,故$[2, k]$共计k-1个元素中,排除染色为1或者2的元素,还剩下$k - 1 - 2 = k-3$个元素,由于$k \ge 4$,故$k-3 \ge 1$,至少存在一个元素满足条件,染色为非1或2,且不在第一行,记为第k个元素。交换第i行和第j行后,这种染色也是满足条件的,但是在新染色中,$(i,j)$染色既不为1也不为2,故$(i, j)$也不能是格子。

	证毕。

\end{proof}


因此类似k=3的情况,我们只需考虑像这样的染色

\begin{table}[h]
	\centering
	\caption{k阶重染色}
	\begin{tabular}{|c|c|c|c|c|c|c|c|}
		\hline
		1        & 2        & 0        & $\cdots$ & 0        & 1        & 2        & $\cdots$ \\
		\hline
		0        & 0        & 0        & $\cdots$ & 0        & 0        & 0        & $\cdots$ \\
		\hline
		0        & 0        & 0        & $\cdots$ & 0        & 0        & 0        & $\cdots$ \\
		\hline
		$\vdots$ & $\vdots$ & $\vdots$ & $\ddots$ & $\vdots$ & $\vdots$ & $\vdots$ & $\ddots$ \\
		\hline
		0        & 0        & 0        & $\cdots$ & 0        & 0        & 0        & $\cdots$ \\
		\hline
		1        & 2        & 0        & $\cdots$ & 0        & 1        & 2        & $\cdots$ \\
		\hline
		0        & 0        & 0        & $\cdots$ & 0        & 0        & 0        & $\cdots$ \\
		\hline
		$\vdots$ & $\vdots$ & $\vdots$ & $\ddots$ & $\vdots$ & $\vdots$ & $\vdots$ & $\ddots$ \\
		\hline
	\end{tabular}
	\label{fig:k-order-staining-last}
\end{table}

其中染色1的格子的坐标为$(kx_1 + 1, ky_1 + 1)$,染色2的格子为$(kx_2 + 1, ky_2 + 2)$ 