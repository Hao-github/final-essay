\chapter{缺陷格子可能存在的位置与相对应的平铺方案}

\section{骨牌大小k = 3时的情况}
由\ref*{basic-theorem-1}有,此时棋盘的大小只可能为$m \times n, m \equiv 1 (mod 3), n \equiv 2 (mod 3)$,不妨假设缺陷格子的坐标分别为$(i_1, j_1), (i_2, j_2)$。

在这里我们给出一个定理,来辅助我们之后的证明。

\begin{theorem}
	\label{basic-theorem-2}
	对于$m \times n$的广义棋盘和$1 \times k$的骨牌,且$m \equiv 1 (mod k), n \equiv 2 (mod k)$,
	记某格子的坐标为$(i, j)$,意味着某格子处于第i行第j列,则如果存在$i \times j$的矩形缺陷$C$,$i \equiv 1 (mod k), j \equiv 2 (mod k), i \le m, j \le n$,
	且$Corner_C = \left\{(3k_{11} + 1, 3k_{12} + 1), (3k_{21} + 1, 3k_{22} + 2)\right\}$,
	则该棋盘一定能被平铺。
\end{theorem}
\begin{proof}
	当存在这样的缺陷时,我们可以把棋盘分成如\ref*{fig:nine-separate}的9个区域,分别如图所示,由\ref*{basic-lemma-1},
	可得除了矩形缺陷$C$外的8个区域均可被$1 \times k$的骨牌平铺,证毕。

	\begin{table}[htbp]
		\centering
		\caption{对缺陷棋盘的划分}
		\begin{tabular}{|c|c|c|}

			\hline
			$3k_{12} \times 3k_{11}$               & $(3k_{22} - 3k_{12} + 2)\times 3k_{11}$  & $3k_{22} \times 3k_{11}$               \\
			\hline
			$3k_{12} \times (3k_{21}-3k_{11} + 1)$ & $C: i \times j$                          & $3k_{22} \times (3k_{21}-3k_{11} + 1)$ \\
			\hline
			$3k_{12} \times 3k_{21}$               & $3 (k_{22} - k_{12}) + 2 \times 3k_{21}$ & $3k_{22} \times 3k_{21}$               \\
			\hline
		\end{tabular}
		\label{fig:nine-separate}
	\end{table}
\end{proof}

由定理\ref*{basic-theorem-2},我们可以得把问题转化为:
对于某个残缺两格的棋盘,如果他的两个缺陷格子可以组合成一个满足\ref*{basic-theorem-2}中条件的矩形缺陷$C$,那么该棋盘可以被平铺。

在图\ref*{fig:k-order-staining-1}的基础上,我们对棋盘进行两种染色,分别记为$Color^3_1$和$Color^3_2$,
显然有$f(1) = f(2) = f(3) + 1$,并据此按照以下规则重新染色得到$Color^3_3$,即图\ref*{fig:3-order-staining-last}。

\begin{itemize}
	\item $Color^3_1(i, j) = Color^3_2(i, j) = 1$,则$Color^3_3(i, j) = 1$。
	\item $Color^3_1(i, j) = Color^3_2(i, j) = 2$,则$Color^3_3(i, j) = 2$。
	\item $Color^3_1(i, j) = 1, Color^3_2(i, j) = 2$,则$Color^3_3(i, j) = 4$。
	\item $Color^3_1(i, j) = 2, Color^3_2(i, j) = 1$,则$Color^3_3(i, j) = 5$。
	\item $Color^3_1(i, j) = 3$或 $Color^3_2(i, j) = 3$,则$Color^3_3(i, j) = 0$。
\end{itemize}

而该棋盘可被平铺的充分条件为缺陷的两个格子在$Color^3_1$和$Color^3_2$中都落在染色1和染色2的位置,即以下两个条件之一:

\begin{itemize}
	\item $Color^3_3(i_1, j_1) = 1, Color^3_3(i_2, j_2) = 2$
	\item $Color^3_3(i_1, j_1) = 4, Color^3_3(i_2, j_2) = 5$
\end{itemize}

\begin{table}[htbp]
	\caption{3阶染色}
	\label{fig:3-order-staining}
	\begin{subtable}{.5\linewidth}
		\centering
		\caption{3阶染色-1}
		\begin{tabular}{|c|c|c|c|c|c|}
			\hline
			1        & 2        & 3        & 1        & 2        & $\cdots$ \\
			\hline
			2        & 3        & 1        & 2        & 3        & $\cdots$ \\
			\hline
			3        & 1        & 2        & 3        & 1        & $\cdots$ \\
			\hline
			1        & 2        & 3        & 1        & 2        & $\cdots$ \\
			\hline
			2        & 3        & 1        & 2        & 3        & $\cdots$ \\
			\hline
			$\vdots$ & $\vdots$ & $\vdots$ & $\vdots$ & $\vdots$ & $\ddots$ \\
			\hline
		\end{tabular}
		\label{fig:3-order-staining-1}
	\end{subtable}%
	\begin{subtable}{.5\linewidth}
		\centering
		\caption{3阶染色-2}
		\begin{tabular}{|c|c|c|c|c|c|}
			\hline
			1        & 2        & 3        & 1        & 2        & $\cdots$ \\
			\hline
			3        & 1        & 2        & 3        & 1        & $\cdots$ \\
			\hline
			2        & 3        & 1        & 2        & 3        & $\cdots$ \\
			\hline
			1        & 2        & 3        & 1        & 2        & $\cdots$ \\
			\hline
			3        & 1        & 2        & 3        & 1        & $\cdots$ \\
			\hline
			$\vdots$ & $\vdots$ & $\vdots$ & $\vdots$ & $\vdots$ & $\ddots$ \\
			\hline
		\end{tabular}
		\label{fig:3-order-staining-2}
	\end{subtable}
\end{table}

\begin{table}[htbp]
	\centering
	\caption{3阶染色-3}
	\begin{tabular}{|c|c|c|c|c|c|c|}
		\hline
		1        & 2        & 0        & 1        & 2        & 0        & $\cdots$ \\
		\hline
		0        & 0        & 4        & 0        & 0        & 4        & $\cdots$ \\
		\hline
		0        & 0        & 5        & 0        & 0        & 5        & $\cdots$ \\
		\hline
		1        & 2        & 0        & 1        & 2        & 0        & $\cdots$ \\
		\hline
		0        & 0        & 4        & 0        & 0        & 4        & $\cdots$ \\
		\hline
		0        & 0        & 5        & 0        & 0        & 5        & $\cdots$ \\
		\hline
		$\vdots$ & $\vdots$ & $\vdots$ & $\vdots$ & $\vdots$ & $\vdots$ & $\ddots$ \\
		\hline
	\end{tabular}
	\label{fig:3-order-staining-last}
\end{table}

\subsection{在染色中分别处于4和5}

由图\ref{fig:3-order-staining-last},我们可以简单的得到染色为4和5所对应的格子的坐标,

\begin{itemize}
	\item $Color^3_3(3k_{41} + 2, 3k_{42}) = 4$
	\item $Color^3_3(3k_{51} + 3, 3k_{52}) = 5$
\end{itemize}

由于棋盘具有对称性,在这里我们不妨假设$k_{51} \ge k_{41}$,则有以下2种情况:

\begin{table}[htbp]
	\centering
	\caption{$k_{52} \ge k_{42}$时的缺陷拼接}
	\begin{tabular}{|cc|c|cc|c|cc|}
		\hline
		\multicolumn{6}{|c|}{$1 \times (3k_{51} - 3k_{41} + 3)$}                & \multicolumn{2}{|c|}{\multirow{4}*{$(3k_{52} - 3k_{42} + 3) \times 2$}}                                                                                        \\
		\cline{1-6}
		\multicolumn{2}{|c|}{\multirow{4}*{$(3k_{52} - 3k_{42} + 3) \times 2$}} & 4                                                                       & \multicolumn{3}{|c|}{$1 \times (3k_{51} - 3k_{41})$}                       &   &     \\
		\cline{3-6}
		                                                                        &                                                                         & \multicolumn{4}{|c|}{$(3k_{52} - 3k_{42}) \times (3k_{51} - 3k_{41} + 1)$} &   &     \\
		\cline{3-6}
		                                                                        &                                                                         & \multicolumn{3}{|c|}{$1 \times (3k_{51} - 3k_{41})$}                       & 5 &   & \\
		\cline{3-8}
		                                                                        &                                                                         & \multicolumn{6}{|c|}{$1 \times (3k_{51} - 3k_{41} + 3)$}                             \\
		\hline
	\end{tabular}
	\label{fig:4-5-painting}
\end{table}

\begin{table}[htbp]
	\centering

	\caption{$k_{52} < k_{42}$的缺陷拼接}
	\begin{tabular}{|ccccc|cc|}
		\hline
		\multicolumn{5}{|c|}{\multirow{2}*{$2 \times (3k_{51} - 3k_{41} + 3)$}} & \multicolumn{2}{|c|}{\multirow{3}*{$3 \times 2$}}                                                                                                                                           \\
		                                                                        &                                                   &                                                                            &   &                                                   &  & \\
		\cline{1-5}
		\multicolumn{2}{|c|}{\multirow{3}*{$2 \times (3k_{42} - 3k_{52} + 3)$}} & 5                                                 & \multicolumn{2}{|c|}{$1 \times (3k_{51} - 3k_{41})$}                       &   &                                                        \\
		\cline{3-7}
		                                                                        &                                                   & \multicolumn{5}{|c|}{$(3k_{42} - 3k_{52}) \times (3k_{51} - 3k_{41} + 3)$}                                                              \\
		\cline{3-7}
		                                                                        &                                                   & \multicolumn{2}{|c|}{$1 \times (3k_{51} - 3k_{41})$}                       & 4 & \multicolumn{2}{|c|}{\multirow{3}*{$3 \times 2$}}      \\
		\cline{1-5}
		\multicolumn{5}{|c|}{\multirow{2}*{$2 \times (3k_{51} - 3k_{41} + 3)$}} &                                                   &                                                                                                                                         \\
		                                                                        &                                                   &                                                                            &   &                                                   &  & \\
		\hline
	\end{tabular}
	\label{fig:5-4-painting}
\end{table}

\begin{itemize}
	\item 在图\ref*{fig:4-5-painting}中,$Corner_C = \left\{(3k_{41} + 1, 3k_{42} - 2), (3k_{51} + 4, 3k_{52} + 2)\right\}$。
	\item 在图\ref*{fig:5-4-painting}中,$Corner_C = \left\{(3k_{51} + 1, 3k_{52} - 2), (3k_{41} + 4, 3k_{42} + 2)\right\}$。
\end{itemize}

由\ref*{fig:4-5-painting}和\ref*{fig:5-4-painting}可得,无论缺陷格子如何落在颜色4和颜色5上,
我们都能找到一种拼接方案,将两个缺陷拼接成满足\ref*{basic-theorem-2}中的条件的矩形缺陷。

\subsection{在染色中分别处于1和2}

同在染色中处于4和5的情况,我们可以写出染色为1和2所对应的格子坐标。

\begin{itemize}
	\item $Color^3_3(3k_{11} + 1, 3k_{12} + 1) = 1$
	\item $Color^3_3(3k_{21} + 1, 3k_{22} + 2) = 2$
\end{itemize}


由于对称性,在这里我们不妨假设$k_{21} \ge k_{11}$,则同样有以下两种情况:

\begin{table}[htbp]
	\centering
	\caption{$k_{22} \ge k_{12}$的缺陷拼接}
	\begin{tabular}{|ccc|c|}
		\hline
		1                                                                                    & \multicolumn{2}{|c|}{$1 \times (3k_{22} -3k_{12})$} & \multirow{2}*{$(3k_{21} - 3k_{11}) \times 1$}     \\
		\cline{1-3}
		\multicolumn{3}{|c|}{\multirow{2}*{$(3k_{21} - 3k_{11}) \times (3k_{22} -3k_{12})$}} &                                                                                                         \\
		\cline{4-4}
		                                                                                     &                                                     &                                               & 2 \\
		\hline
	\end{tabular}
	\label{fig:1-2-painting}
\end{table}

\begin{table}[htbp]
	\centering
	\caption{$k_{22} < k_{12}$且$k_{21} > k_{11}$的缺陷拼接}
	\begin{tabular}{|cc|ccc|cc|}
		\hline
		\multicolumn{2}{|c|}{\multirow{3}{*}{$(3k_{21} - 3k_{11}) \times 1$}} & 2 & \multicolumn{4}{|c|}{$1 \times  (3k_{12} - 3k_{22})$}                                                                                 \\
		\cline{3-7}
		                                                                      &   & \multicolumn{3}{c}{\multirow{2}{*}{...}}              & \multicolumn{2}{|c|}{\multirow{3}{*}{$(3k_{21} - 3k_{11}) \times 1$}}         \\
		                                                                      &   &                                                       &                                                                       &  &  & \\
		\cline{1-5}
		\multicolumn{4}{|c|}{$1 \times  (3k_{12} - 3k_{22})$}                 & 1 &                                                       &                                                                               \\
		\hline
		\label{fig:2-1-painting-1}
	\end{tabular}
\end{table}

\begin{itemize}
	\item 当$k_{22} \ge k_{12}$时,有缺陷拼接图\ref*{fig:1-2-painting},\\
	      $Corner_C = \left\{(3k_{11} + 1, 3k_{12} + 1), (3k_{21} + 1, 3k_{22} + 2)\right\}$。
	\item 当$k_{22} < k_{12}$时,有以下分类:
	      \begin{itemize}
		      \item 当$i_1 \neq i_2$时,有缺陷拼接图\ref*{fig:2-1-painting-1},\\
		            $Corner_C = \left\{(3k_{21} + 1, 3k_{22} + 1), (3k_{11} + 1, 3k_{22} + 2)\right\}$。
		      \item 当$i_1 = i_2 \notin {1, m}$时,有$m \ge 7$,当$n = 5$时,穷举得到无解;当$n \ge 8$时:
		            \begin{itemize}
			            \item 当$k_{12} - k_{22} = 1$时,存在缺陷拼接,\\
			                  $Corner_C = \left\{(3k_{21} - 2, 3k_{22} + 1), (3k_{11} + 1, 3k_{22} + 2)\right\}$,具体缺陷拼接图见附录。
			            \item 当$k_{12} - k_{22} > 1$时,存在缺陷拼接,\\
			                  $Corner_C = \left\{(3k_{21} - 2, 3k_{22} + 1), (3k_{11} + 4, 3k_{22} + 5)\right\}$,具体缺陷拼接图见附录。
		            \end{itemize}
		      \item 当$i_1 = i_2 \in {1, m}$时,我们称为棘手情况,把它放在下一节讨论。
	      \end{itemize}
\end{itemize}


除了棘手情况,无论缺陷如何分布在颜色1和颜色2上,我们们都能找到一种拼接方案,将两个缺陷拼接成满足\ref*{basic-theorem-2}中的条件的矩形缺陷。

因此由\ref*{basic-theorem-2},除棘手情况外,当缺陷分别落在颜色4和颜色5上,或者分别落在颜色1和颜色2上时,无论缺陷怎么分布,该缺陷棋盘都可被平铺。

\section{骨牌大小k > 3时的情况}

\begin{table}[h]
	\caption{k阶染色}
	\label{fig:k-order-staining}
	\begin{subtable}{.5\linewidth}
		\centering
		\caption{k阶染色-1}
		\begin{tabular}{|c|c|c|c|c|c|c|}
			\hline
			1        & 2        & 3        & $\cdots$ & k        & 1        & $\cdots$ \\
			\hline
			2        & 3        & 4        & $\cdots$ & 1        & 2        & $\cdots$ \\
			\hline
			3        & 4        & 5        & $\cdots$ & 2        & 3        & $\cdots$ \\
			\hline
			$\vdots$ & $\vdots$ & $\vdots$ & $\ddots$ & $\ddots$ & $\ddots$ & $\cdots$ \\
			\hline
			k        & 1        & 2        & $\cdots$ & k-1      & k        & $\cdots$ \\
			\hline
			1        & 2        & 3        & $\cdots$ & k        & 1        & $\cdots$ \\
			\hline
			$\vdots$ & $\vdots$ & $\vdots$ & $\ddots$ & $\vdots$ & $\vdots$ & $\ddots$ \\
			\hline
		\end{tabular}
		\label{fig:k-order-staining-1}
	\end{subtable}%
	\begin{subtable}{.5\linewidth}
		\centering
		\caption{k阶染色-2}
		\begin{tabular}{|c|c|c|c|c|c|c|}
			\hline
			1        & 2        & 3        & $\cdots$ & k        & 1        & $\cdots$ \\
			\hline
			k        & 1        & 2        & $\cdots$ & k-1      & k        & $\cdots$ \\
			\hline
			k-1      & k        & 1        & $\cdots$ & k-2      & k-1      & $\cdots$ \\
			\hline
			$\vdots$ & $\vdots$ & $\vdots$ & $\ddots$ & $\ddots$ & $\ddots$ & $\cdots$ \\
			\hline
			2        & 3        & 4        & $\cdots$ & 1        & 2        & $\cdots$ \\
			\hline
			1        & 2        & 3        & $\cdots$ & k        & 1        & $\cdots$ \\
			\hline
			$\vdots$ & $\vdots$ & $\vdots$ & $\ddots$ & $\vdots$ & $\vdots$ & $\ddots$ \\
			\hline
		\end{tabular}
		\label{fig:k-order-staining-2}
	\end{subtable}
\end{table}

\begin{table}[htbp]
	\centering
	\caption{$k \times k$的区域划分}
	\begin{tabular}{|c|c|}
		\hline
		A: $1 \times 2$       & B: $1  \times (k - 2)$     \\
		\hline
		C: $(k - 1) \times 2$ & D: $(k - 1)\times (k - 2)$ \\
		\hline
	\end{tabular}
	\label{fig:k-division}
\end{table}

在$k \ge 4$的情况下,我们可以给出以下定理。
\begin{theorem}
	\label{basic-theorem-3}
	当$m \times n$的缺陷两格广义棋盘可被$1 \times k$大小的骨牌平铺时,若$k \ge 4$,则缺陷格子分别只可能在坐标$(kx_1 + 1, ky_1 + 1)$和$(kx_2 + 1, ky_2 + 2)$处。
\end{theorem}
\begin{proof}
	我们能够很轻易的得到$f(1) = f(2) = f(x), x \neq 1, x \neq 2$,
	缺陷格子只能落在在所有染色组合中均被染色为1或染色2的格子,即$Color^k_p(i, j) \in \left\{1, 2\right\},  \forall p$,
	而$Color^k_p(kx_1 + 1, ky_1 + 1) = 1, Color^k_p(kx_2 + 1, ky_2 + 2) = 2, \forall p$,
	因此缺陷格子分别落在坐标$(kx_1 + 1, ky_1 + 1)$和$(kx_2 + 1, ky_2 + 2)$处的格子时,棋盘可能可以被平铺。

	接下来我们证明,当$(i, j) \neq (kx_1 + 1, ky_1 + 1), (i, j) \neq (kx_2 + 1, ky_2 + 2)$时,$\exists p, s.t. Color^k_p(i, j) \notin \left\{1, 2\right\}$。

	挑选k阶染色中最经典的两种染色方案$Color^k_1$和$Color^k_2$(图\ref{fig:k-order-staining-1}和图\ref{fig:k-order-staining-2}),
	考虑最左上的$k \times k$大小的区域,做出表\ref{fig:k-division}的区域划分,随后分别考虑模块B, C和D:

	\begin{itemize}
		\item[模块B] 由于模块B不会染色为1或2,所以缺陷格子不可能落在模块B中。
		\item[模块C]
			\begin{itemize}
				\item $Color^k_1(2, 1) = 2, Color^k_2(2, 1) = k$
				\item $Color^k_1(2, 2) = 3, Color^k_2(2, 2) = 1$
				\item $Color^k_1(k, 1) = k, Color^k_2(k, 1) = 2$
				\item $Color^k_1(k, 2) = 1, Color^k_2(k, 2) = 3$
			\end{itemize}
			由于模块C在两种染色下只有共计4个1或2,故模块C中的所有格子,均不可能满足$Color^k_p(i, j) \in \left\{1, 2\right\}, \forall p$。
		\item[模块D] 对于任意$i$,不妨假定$Color^k_1(i, j_1) = 1, Color^k_1(i, j_2) = 2$,考虑$Color^k_1(i, y) = 1, y \in [2, k], y \notin \left\{j_1, j_2\right\}$,
			由于$k \ge 4$,故$\exists j^{'}, s.t. Color^k_1(i, j^{'}) = t \notin \left\{1, 2\right\}$,此时交换第$j_1$行和第$j^{'}$行,仍然是一个满足条件的染色方案$p$,
			但$Color^k_p(i, j) = Color^k_1(i, j^{'}) = t \notin \left\{1, 2\right\}$,矛盾。
	\end{itemize}
	因此缺陷格子只能落在模块A中,证毕。
\end{proof}

因此类似$k=3$的情况,我们只需考虑像这样的染色

\begin{table}[htbp]
	\centering
	\caption{k阶重染色}
	\begin{tabular}{|c|c|c|c|c|c|c|c|}
		\hline
		1        & 2        & 0        & $\cdots$ & 0        & 1        & 2        & $\cdots$ \\
		\hline
		0        & 0        & 0        & $\cdots$ & 0        & 0        & 0        & $\cdots$ \\
		\hline
		0        & 0        & 0        & $\cdots$ & 0        & 0        & 0        & $\cdots$ \\
		\hline
		$\vdots$ & $\vdots$ & $\vdots$ & $\ddots$ & $\vdots$ & $\vdots$ & $\vdots$ & $\ddots$ \\
		\hline
		0        & 0        & 0        & $\cdots$ & 0        & 0        & 0        & $\cdots$ \\
		\hline
		1        & 2        & 0        & $\cdots$ & 0        & 1        & 2        & $\cdots$ \\
		\hline
		0        & 0        & 0        & $\cdots$ & 0        & 0        & 0        & $\cdots$ \\
		\hline
		$\vdots$ & $\vdots$ & $\vdots$ & $\ddots$ & $\vdots$ & $\vdots$ & $\vdots$ & $\ddots$ \\
		\hline
	\end{tabular}
	\label{fig:k-order-staining-last}
\end{table}

同$k=3$的情况,我们只需要考虑染色分别处于1和2的情况,此时可以完全的套用$k=3$的情况,且棘手情况也一样,为$k>3$时的边界情况,我们在下一节一起讨论。

\clearpage