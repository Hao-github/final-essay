%% chapter 5 dataset, network structure, experiment and result
\chapter{结论与展望}
\section{结论}
对于$m \times n$的缺陷两格广义棋盘,骨牌大小$1 \times k$,如果棋盘可以被平铺,棋盘一定满足条件$m \equiv 2 (mod k), n \equiv 1 (mod k)$。
且在此前提下,当$k \ge 3$时,当两个缺陷的坐标$(i_1, j_1)$和$(i_2, j_2)$满足$i_1 \equiv i_2 \equiv 1 (mod k), j_1 + 1 \equiv j_2 \equiv 2 (mod k)$,
且当$i_1 = i_2 = 1$或$i_1 = i_2 = m$时,$j_2 > j_1, m \ge 2k + 2, n \ge 2k + 1$,则此时棋盘可被平铺。

当$k = 3$的情况,当两个缺陷的坐标$(i_1, j_1)$和$(i_2, j_2)$满足$i_1 + 1\equiv i_2 \equiv 0 (mod 3), j_1 \equiv j_2 \equiv 0 (mod 3)$,棋盘也可以被平铺。

除此之外,所有的缺陷两格的$m \times n$棋盘都不可被平铺。

\section{展望}
事实上,关于残缺棋盘的完全覆盖问题是一个困难而且复杂的问题,在本论文解决了缺陷两格的完全覆盖问题后,可以沿着本文的解决思路结局缺陷三格甚至缺陷k格的完全覆盖问题,这是下一步的研究方向。
