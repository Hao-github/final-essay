\chapter{绪论}
%定义,过去的研究和现在的研究,意义,与图像分割的不同
棋盘完全覆盖问题(problem of perfect cover of chessboard)是一类组合问题,是研究对于一个$m \times n$的广义棋盘, 在缺陷$z$格的情况下(通常$z=0$), 被骨牌完全覆盖的问题。

在过去,我们对此问题的研究主要落在棋盘完全覆盖问题的方案数,却甚少对棋盘完全覆盖问题的可解性进行考虑,显然,只有在确保该缺陷棋盘可被覆盖的基础上,
棋盘完全覆盖的方案数的研究才有价值。此外,研究棋盘完全覆盖问题的可解性,有助于对现实中的地砖覆盖等问题提供参考。

\section{选题背景与意义}
\label{sec:background}
% What is the problem
% why is it interesting and important
% Why is it hards, why do naive approaches fails
% why hasn't it been solved before
% what are the key components of my approach and results, also include any specific limitations,do not repeat the abstract
%contribution
我们称某$1 \times k$骨牌能完全覆盖某缺陷棋盘,当且仅当以下条件满足:
\begin{enumerate}
    \item 每块骨牌能够连续覆盖棋盘上同一行或者同一列的相邻$k$格
    \item 棋盘上每一格都被骨牌覆盖。
    \item 没有两块骨牌同时覆盖一格。
    \item 对于棋盘的缺陷处,没有任意一张骨牌将其覆盖。
\end{enumerate}

而本文研究的缺陷两格棋盘的覆盖问题,则是研究在$m \times n$的广义棋盘上,挖去任意两格的格子,是否可被$1 \times k$的骨牌所覆盖。

对于此问题而言,我们一般会使用染色法求解,也能够在经典问题:国际棋盘挖去做左上右下两个格子是否可被$1 \times 2$骨牌平铺的问题上做出相当漂亮的证明。

可是对于高维度而言,普通的染色法有点力不从心,只能排除较为基础的情况而无法对整个棋盘的各种缺陷情况进行完整的考量。

在本文中,我推广了染色法进行更多情况的排除,得到了在普遍条件下,缺陷两格棋盘是否可被平铺的证明,

\section{国内外研究现状和相关工作}
\label{sec:related_work}
对国内外研究现状和相关领域中已有的研究成果的简要评述。
\section{本文的论文结构与章节安排}

\label{sec:arrangement}

本文共分为六章,各章节内容安排如下:

第一章绪论。简单说明了本文章的选题背景与意义。

第二章构建出可被$1 \times 3$大小的骨牌平铺的棋盘的必要条件,初步构建出满足条件的棋盘的大小。

第三章解析了在绝大部分情况下,格子残缺可以满足$1 \times 3$大小的骨牌进行平铺。

第四章对于极特殊的情况,给出了格子残缺不可被平铺的证明。

第五章则推广了对于$1 \times k$大小的骨牌,能够平铺棋盘的充分必要条件。

