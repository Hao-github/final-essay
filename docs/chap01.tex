\chapter{绪论}
%定义,过去的研究和现在的研究,意义,与图像分割的不同
棋盘完全覆盖问题(problem of perfect cover of chessboard)是一类组合问题,是研究对于一个$m \times n$的广义棋盘, 在缺陷$z$格的情况下(通常$z=0$), 被骨牌完全覆盖的问题。

在过去,我们对此问题的研究主要落在棋盘完全覆盖问题的方案数,却甚少对棋盘完全覆盖问题的可解性进行考虑,显然,只有在确保该缺陷棋盘可被覆盖的基础上,
棋盘完全覆盖的方案数的研究才有价值。此外,研究棋盘完全覆盖问题的可解性,有助于对现实中的地砖覆盖等问题提供参考。

\section{选题背景与意义}

\subsection{选题背景}

我们称某$1 \times k$骨牌能完全覆盖某缺陷棋盘,当且仅当以下条件满足:
\begin{enumerate}
    \item 每块骨牌能够连续覆盖棋盘上同一行或者同一列的相邻$k$格
    \item 棋盘上每一格都被骨牌覆盖。
    \item 没有两块骨牌同时覆盖一格。
    \item 对于棋盘的缺陷处,没有任意一张骨牌将其覆盖。
\end{enumerate}

而本文研究的缺陷两格棋盘的覆盖问题,则是研究在$m \times n$的广义棋盘上,挖去任意两格的格子,是否可被$1 \times k$的骨牌所覆盖。

对于此问题而言,我们一般会使用染色法求解,也能够在经典问题:国际棋盘挖去做左上右下两个格子是否可被$1 \times 2$骨牌平铺的问题上做出相当漂亮的证明。

可是对于高维度而言,普通的染色法有点力不从心,只能排除较为基础的情况而无法对整个棋盘的各种缺陷情况进行完整的考量。

在本文中,我推广了染色法进行更多情况的排除,得到了在普遍条件下,缺陷两格棋盘是否可被平铺的证明,

\subsection{国内外研究现状和相关应用}
\label{sec:related_work}
对棋盘覆盖问题的研究由来已久,冯跃峰证明了线型骨牌在完整矩形棋盘的完全覆盖充要条件;康庆德用染色法证明了残缺棋盘可被$1 \times 2$多米诺骨牌完全覆盖的充要条件;张媛则用轨道循环标号法证明了线型骨牌在残缺
一格棋盘上的覆盖问题。

而关于棋盘覆盖实际问题的研究的应用,宗传明则发现骨牌覆盖问题可应用于地砖铺设当中,不仅限于线型骨牌,还存在多边形骨牌覆盖,并且完整刻画了能构成六重晶格铺砌的所有铺砖,而经典的俄罗斯方块的拼图小游戏,
文件加密技术等都会应用到棋盘覆盖问题的相关结论。
\section{本文的论文结构与章节安排}

\label{sec:arrangement}

本文共分为五章,各章节内容安排如下:

第一章绪论。简单说明了本文章的选题背景与意义。

第二章构建出可被$1 \times k$大小的骨牌平铺的棋盘的必要条件,初步构建出满足条件的棋盘的大小。

第三章展示了在大部分情况下,残缺格子位于哪些位置可以满足$1 \times 3$大小的骨牌进行平铺,并给出了相对应的平铺方式。

第四章对于棘手的情况,给出了这类缺陷棋盘不可被平铺的证明。

第五章总结了问题的解,并展望以后的工作。

\section{基础定义与符号表}

在本文中,所有的符号与变量的取值范围都是整数集。

\subsection{基础定义}

\begin{definition}
    我们定义一个$m \times n$的广义棋盘的某个$k$阶染色为,规定第一行第$k$列染色为$(j - 1) \% k + 1$后,其余格子能满足
    每连续$k$个横着的或者竖着的格子中,有且只有一个$1, 2, ... , k$,这样的排布。
\end{definition}

\begin{definition}
    我们定义一个$m \times n$的广义棋盘的两个$k$阶染色不同,当且仅当在第一行的染色完全相同时,剩余的格子中存在染色不同的格子。
\end{definition}

由上述定义,我们能简单的得到一个引理
\begin{lemma}
    对于一个$m \times n$的广义棋盘的某个$k$阶染色,任意交换其中的两行或两列后,新的棋盘仍然是一个合法的$k$阶染色。
\end{lemma}


\subsection{符号表}
\begin{table}[htbp]
    \centering
    \caption{符号解释}
    \begin{tabular}{lcr}
        \toprule
        符号                & 意义                         & 取值范围                          \\
        \midrule
        $m \times n$      & 棋盘大小为$m$行$n$列              & $[1, \inf) \times [1, \inf) $ \\
        $1 \times k $     & 骨牌大小为$1$行$k$列              & $[3, min(m, n)]$              \\
        $(i, j)$          & 第$i$行第$j$列的格子              & $[1, m] \times [1, n]$        \\
        $Color(i, j)$     & 第$i$行第$j$列的格子所染的颜色         & $[1, k]$                      \\
        $Color^k_p$       & $k$阶染色方案$p$                & $\mathbb{Z}$                  \\
        $Color^k_p(i, j)$ & $(i, j)$在$Color^k_p$下所染的颜色 & $[1, k]$                      \\
        $f(x)$            & 染色为$x$的格子的个数               & $[1, k]$                      \\
        $g(x)$            & 标记为$x$的格子的个数               & $[1, m + n]$                  \\
        $Corner_C$        & 矩形缺陷$C$的左上角和右下角的坐标         & $-$                           \\
        \bottomrule
    \end{tabular}
\end{table}