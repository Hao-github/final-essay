\chapter{棋盘所满足的条件}
% \renewcommand{\theenumii}{Subcase \arabic{enumi}.\arabic{enumii}}
% \renewcommand{\labelenumii}{\theenumii.}
\label{cha:sysu-thesis-contents-format-requirement}
对一般的$m \times n$的完整矩形棋盘, 冯跃峰给出了$k$格线型骨牌完全覆盖的充分必要条件, 这是我们研究覆盖问题的基础。

\begin{lemma}
    \label{basic-lemma-1}
    $m \times n$棋盘中存在$k$格线型骨牌完全覆盖,当且仅当$k \mid m$或$k \mid n$。
\end{lemma}

而对于多格残缺棋盘,重点是确定棋盘的大小$(m, n)$和残缺格子所在的位置$(i, j)$,在这一章节我们先考虑棋盘的大小所满足的条件。

以下的符号均建立在整数集的基础上。

显而易见的,我们能简单得到以下引理。
\begin{lemma}
    \label{basic-lemma-2}
    $m \times n$的广义棋盘中,挖去2个格子,如果存在$k$格线型骨牌完全覆盖,当且仅当$k \mid mn-2$。
\end{lemma}

仅仅有这个条件是不够的,我们希望说能多加限制条件来排除更多的棋盘,因此我们对棋盘重新考虑,令
$$
    \left\{
    \begin{aligned}
        m & =  kx_m + m_1, m_1 \in [0, k) \\
        n & =  kx_n + n_1, n_1 \in [0, k)
    \end{aligned}
    \right.
$$

接下来我们对棋盘进行划分,有

\begin{table}[h]
    \centering
    \caption{对棋盘的初步划分}
    \begin{tabular}{|c|c|}

        \hline
        $kx_m \times kx_n$ & $m_1 \times kx_n$ \\
        \hline
        $kx_m \times n_1$  & $m_1 \times n_1$  \\
        \hline
    \end{tabular}
    \label{fig:first-picture}
\end{table}

由\ref*{basic-lemma-1}, 可得除$m_1 \times n_1$外,均可被$1 \times k$骨牌平铺,故我们只需要考虑$m_1 \times n_1$这一块, 记为模块C。

我们给出以下定理,来规整棋盘的大小
\begin{theorem}
    \label{basic-theorem-1}
    当$m \times n$的缺陷两格广义棋盘可被$1 \times k$大小的骨牌平铺时,一定满足$m \equiv 1 (mod k), n \equiv 2 (mod k)$
\end{theorem}

\begin{proof}
    \begin{table}[ht]
        \centering
        \caption{k阶染色}
        \begin{tabular}{|c|c|c|c|c|c|c|c|}
            \hline
            1        & 2        & $\cdots$ & $\cdots$ & $m_1$     & $\cdots$ & $n_1 - 1$ & $n_1$          \\
            \hline
            2        & 3        & $\cdots$ & $\cdots$ & $m_1 + 1$ & $\cdots$ & $n_1$     & $n_1 + 1$      \\
            \hline
            $\vdots$ & $\vdots$ & $\ddots$ & $\ddots$ &           &          & $\vdots$  & $\vdots$       \\
            \hline
            $m_1$    & $\cdots$ & $n_1-1$  & $n_1$    & $\cdots$  & $\cdots$ & $\vdots$  & $m_1 + n_1 -k$ \\
            \hline
        \end{tabular}
        \label{fig:k-order-staining-example}
    \end{table}
    不妨设$m_1 \le n_1$,对表格进行染色,有\ref*{fig:k-order-staining-1}

    由\ref*{fig:k-order-staining-1},记$f(x)$为棋盘中染色为x的格子数目,有$f(m_1) = f(n_1) = f(n_1 + 1) + 1$,
    而$n_1 < k  \rightarrow n_1 + 1 \le k$,故$f(1) < 1 + f(n_1 + 1) = 1 + f(n_1) - 1 = f(n_1)$,得到式子
    \begin{equation}
        \label{eq1}
        f(m_1) = f(n_1) > f(1)
    \end{equation}
    而缺陷两格至多使某个$f(x)$减少2或者使某两个$f(x)$减少1,而每个平铺会且仅会覆盖1个1, 2, ... ,k,故我们需要所有的f(x)相等,且至多2个f(x)大于f(1)

    此时分情况讨论

    \begin{enumerate}
        \item $m_1 < n_1 - 1$,由\ref*{fig:k-order-staining-1}有$f(m_1) = f(n_1) - 1 = f(n_1) > f(1)$,而缺陷格子为2个,至多两个$f(x)$减少1,矛盾。
        \item $m_1 = n_1$,有$f(m_1) \ge f(1) + 1$
              \begin{enumerate}
                  \item $f(m_1) = f(1) + 1$,则有$f(x) = f(m_1) - 1, \forall x < k, x \neq m_1$, 而缺陷格子数目为2,则挖去一个染色为$m_1$的格子后,所有颜色的格子数目已经平衡,
                        此时挖去任意其他格子均无法保证所有x相等,故矛盾。
                  \item $f(m_1) > f(1) + 1$, 则有$f(m_1) = f(m_1+1) + 1 = f(m_1 + 2) + 2\ge f(1)$, 因此我们至少需要挖去染色为$m_1$的两个格子和染色为$m_1 + 1$的一个格子,但我们只能挖去两个格子,矛盾
              \end{enumerate}
        \item $m_1 = n_1 - 1$ 由上述讨论我们可以得到,我们能且只能挖去染色为$m_1$和$n_1$的格子各一个,此时将有$1 \equiv n_1 + 2 (mod k) \rightarrow 1 = n_1 + 2 - k \rightarrow n_1 = k - 1, m_1 = k - 2$
              此时我们能证明(待补充),这种情况我们能找到一个染色方案不满足条件,故成立

              故我们一定只能得到$m_1 =1, n_1 = 2$,得证。
    \end{enumerate}
\end{proof}