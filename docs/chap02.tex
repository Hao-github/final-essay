\chapter{棋盘所满足的条件}

对一般的$m \times n$的完整矩形棋盘, 冯跃峰\cite{fengBasicTheorem}给出了$k$格线型骨牌完全覆盖的充分必要条件, 这是我们研究覆盖问题的基础。

\begin{lemma}
    \label{basic-lemma-1}
    $m \times n$棋盘中存在$k$格线型骨牌完全覆盖,当且仅当$k \mid m$或$k \mid n$。
\end{lemma}

而对于多格残缺棋盘,重点是确定棋盘的大小$(m, n)$和残缺格子所在的位置$(i, j)$,在这一章节我们先考虑棋盘的大小所满足的条件。

以下的符号均建立在整数集的基础上,由于康庆德已经研究完毕$1 \times 2$骨牌的覆盖\cite{ZRZZ199205013},因此我们的只研究骨牌长度大于3的情况。

显而易见的,我们能简单得到以下引理。
\begin{lemma}
    \label{basic-lemma-2}
    $m \times n$的广义棋盘中,挖去2个格子,存在$k$格线型骨牌完全覆盖的充分条件是$k \mid mn - 2$。
\end{lemma}
\begin{proof}
    简单的,假设存在t个线性骨牌能够完全覆盖整个棋盘,由于骨牌不会互相重复的覆盖同一个格子,
    所以所有骨牌所覆盖的格子数目应该要和棋盘的总的格子数目相同,即
    \begin{equation}
        tk = mn - 2 \rightarrow k \mid mn - 2
    \end{equation}
\end{proof}
我们给出以下定理,来规整棋盘的大小
\begin{theorem}
    \label{basic-theorem-1}
    当$m \times n$的缺陷两格广义棋盘可被$1 \times k$大小的骨牌平铺时,一定满足$m \equiv 1 (mod k), n \equiv 2 (mod k)$。
\end{theorem}

\begin{proof}
    \begin{align}
        \left\{
        \begin{aligned}
            m & =  kx_m + m_1, m_1 \in [0, k) \\
            n & =  kx_n + n_1, n_1 \in [0, k)
        \end{aligned}
        \right.
        \label{theBegin}
    \end{align}

    简单的,不妨设\ref{theBegin},并据此对大小相同的完整棋盘进行划分,得到表\ref{fig:first-picture}。

    \begin{table}[h]
        \centering
        \caption{对棋盘的初步划分}
        \begin{tabular}{|c|c|}
            \hline
            A: $kx_m \times kx_n$ & B: $m_1 \times kx_n$ \\
            \hline
            C: $kx_m \times n_1$  & D: $m_1 \times n_1$  \\
            \hline
        \end{tabular}
        \label{fig:first-picture}
    \end{table}

    由引理\ref{basic-lemma-1}, 模块A,B,C均可被$1 \times k$骨牌平铺,故我们只需要考虑模块D。不妨设$m_1 \le n_1$,并基于张媛对残缺一格棋盘的研究\cite{zhangOneStain}对模块D进行标记,有表\ref{fig:k-order-staining-example}。

    \begin{table}[ht]
        \centering
        \caption{k阶染色}
        \begin{tabular}{|c|c|c|c|c|c|c|c|}
            \hline
            1        & 2        & $\cdots$ & $\cdots$ & $m_1$     & $\cdots$ & $n_1 - 1$ & $n_1$           \\
            \hline
            2        & 3        & $\cdots$ & $\cdots$ & $m_1 + 1$ & $\cdots$ & $n_1$     & $n_1 + 1$       \\
            \hline
            $\vdots$ & $\vdots$ & $\ddots$ & $\ddots$ &           &          & $\vdots$  & $\vdots$        \\
            \hline
            $m_1$    & $\cdots$ & $n_1-1$  & $n_1$    & $\cdots$  & $\cdots$ & $\vdots$  & $m_1 + n_1 - 1$ \\
            \hline
        \end{tabular}
        \label{fig:k-order-staining-example}
    \end{table}

    \begin{table}[htbp]
        \centering
        \caption{对表\ref{fig:k-order-staining-example}的计数}
        \begin{tabular}{|c|c|c|c|c|c|c|c|c|c|c|c|c|}
            \hline
            $x$    & $1$ & $2$ & $3$ & $...$ & $m_1$ & $m_1 + 1$ & ... & $n_1$ & $n_1 + 1$ & $...$ & $m_1 + n_1 - 1$ \\
            \hline
            $g(x)$ & $1$ & $2$ & $3$ & $...$ & $m_1$ & $m_1$     & ... & $m_1$ & $m_1 - 1$ & $...$ & $1$             \\
            \hline
        \end{tabular}
        \label{count}
    \end{table}
    记$g(x)$标记为$x$的格子的个数,对表\ref{fig:k-order-staining-example}进行计数,得到表\ref{count},就可以简单的得到公式\ref{functionOfSymbolX}。
    \begin{equation}
        g(x)=\left\{
        \begin{aligned}
            x              & , x \in [1, m_1]           \\
            m_1            & , x \in (m_1, n_1)         \\
            m_1 + n_1  - x & , x \in [n_1, m_1 + n_1 - 1]
        \end{aligned}
        \right.
        \label{functionOfSymbolX}
    \end{equation}

    我们将所有标记为$x$的染色为$x \% k$,记$f(x)$为棋盘中染色为x的格子数目,有$f(x) = g(x) + g(x + k)$。而$n_1 < k \rightarrow n_1 + 2 \le k + 1$,故
    $f(1) = g(1) + g(k + 1) \le 1 + g(n_1 + 2) = 1 + g(n_1) - 2 < g(n_1) = f(n_1)$
    ,因此我们能得到等式\ref{eq1}。

    \begin{equation}
        \label{eq1}
        f(m_1) = f(n_1) > f(1)
    \end{equation}

    现在我们考虑残缺两格的棋盘上的直线覆盖,由于每个直线骨牌会且仅会分别覆盖1个染色为$1, 2, ... ,k$的方块,
    因此残缺两格后的棋盘能被直线骨牌覆盖的充分条件是,$\forall p_1, p_2, f(p_1) = f(p_2)$
    因此根据残缺两格分布在同种颜色和不同颜色的可能上,需要满足以下两个条件之一:

    \begin{itemize}
        \item 条件1: $\exists p \in [0, k), s.t. f(p) = f(x) + 2, x \neq p , x \in [0, k)$
        \item 条件2: $\exists p_1, p_2 \in [0, k), s.t. f(p_1) = f(p_2) = f(x) + 1, x \neq p_1 \neq p_2, x \in [0, k)$
    \end{itemize}

    因此如果模块D存在,分情况讨论:

    \begin{enumerate}
        \item $m_1 < n_1 - 1$,由\ref{fig:k-order-staining-example}有$f(m_1) = f(n_1) - 1 = f(n_1) > f(1)$,$\exists p_1, p_2, p_3, s.t. f(p_1) = f(p_2) = f(p_3) > f(1)$,矛盾。
        \item $m_1 = n_1 - 1$ 由上述讨论我们可以得到,此时$p_1 = m_1$,$p_2 = n_1 = m_1 + 1$,满足条件2,
        而$f(1) = f(n_1) - 1 \rightarrow g(1) + g(k + 1) = f(n_1) - 1 \rightarrow g(k + 1) = g(n_1) - 2 \rightarrow n_1 = k - 1$,即$m_1 = k - 2, n_1 = k - 1$。

              假设对于所有的染色,都存在两个格子使得在所有的染色下,这两个格子都被染色为$p_1$和$p_2$,记$Color^k_p(i, j) = a$为第$i$行第$j$列在染色方案$p$下染色为$a$,
              任取一种染色方案$Color^k_p$:
              \begin{itemize}
                \item 当$2 \mid k$时,由对称性,此时两个缺陷格子格子必然位于$(\frac{k}{2} - 1,\frac{k}{2}), (\frac{k}{2},\frac{k}{2})$,
                由\ref{fig:k-order-staining-example}简单的得到$Color^k_p(\frac{k}{2} - 1,\frac{k}{2}) = 1, Color^k_p(\frac{k}{2},\frac{k}{2}) = 2$,
                此时交换第一列和第$\frac{k}{2}$列,仍然是一个合法的染色$Color^k_{p^{'}}$,但是交换后的染色中,
                由于$Color^k_p(1, 1) = 1$,因此交换后的$Color^k_{p^{'}}(\frac{k}{2} - 1,\frac{k}{2}) \neq Color^k_{p^{'}}(\frac{k}{2},\frac{k}{2}) \neq 1$,因此矛盾。
                \item 当$2 \nmid k$时,由对称性,此时两个缺陷格子必然位于$(\frac{k - 1}{2},\frac{k - 1}{2}), (\frac{k - 1}{2},\frac{k + 1}{2})$,
                由\ref{fig:k-order-staining-example}简单的得到$Color^k_p(\frac{k - 1}{2},\frac{k - 1}{2}) = 1, Color^k_p(\frac{k - 1}{2},\frac{k + 1}{2}) = 2$,
                此时交换第一行和第$\frac{k - 1}{2}$行,仍然是一个合法的染色$Color^k_{p^{'}}$,但是交换后的染色中,
                由于$Color^k_p(1, 1) = 1$,因此交换后的$Color^k_{p^{'}}(\frac{k - 1}{2},\frac{k - 1}{2}) \neq Color^k_{p^{'}}(\frac{k - 1}{2},\frac{k + 1}{2}) \neq 1$,也矛盾。
              \end{itemize}
              因此我们总能找到一种染色不满足条件,故矛盾。
        \item $m_1 = n_1$,由\ref{eq1},有$f(m_1) \ge f(1) + 1$
              \begin{enumerate}
                  \item $f(m_1) = f(1) + 1$,则有$f(x) = f(m_1) - 1, \forall x < k, x \neq m_1$, 既不满足条件1也不满足条件2,矛盾。
                  \item $f(m_1) > f(1) + 1$, 则有$f(m_1) = f(m_1 + 1) + 1 = f(m_1 + 2) + 2 \ge f(1)$, 故令$p = m_1$后,$f(p) \ge f(1) + 2$但$f(m_1) > f(1)$,矛盾。
              \end{enumerate}
    \end{enumerate}

    所以挖去两格后,若模块D仍然存在,则所有条件均矛盾,故模块D不存在,即此时$m_1 = 1, n_1 = 2$,证毕。
\end{proof}

因此,对于我们已经规整好了棋盘的规模,并把问题转化如下:

对于大小为$1 \times k$的骨牌,我们考虑缺陷两格的$m \times n$的棋盘,其中$m \equiv 1 (mod k), n \equiv 2 (mod k)$,哪些棋盘可以被平铺。

\clearpage