\chapter{棘手情况的处理}

在上一节,我们遗留了一个情况没有处理,我们称为棘手情况。不妨假设棘手情况的缺陷格子坐标为$(1, t_1 \times k + 2)$和$(1, t_2 \times k + 1)$,其中$t_1 < t_2$。
棋盘规模为$(t^{'}k + 1) \times (tk + 2)$,骨牌大小为$1 \times k$。
这一节我们将证明在此情况下,棋盘不可被平铺。
% +替换为$\oplus$
% 数字表示没有铺的地方
% $1_{\vert}, k_{-}$
% = 表示怎么平铺
% $\rightarrow$表示转化
\section{符号定义}
我们先考虑长度为$x$的位于边界上的待铺一行,由于这一行可以被若干个横着的或者竖着的骨牌所铺设,
因此我们用类似等式\ref{chapter4example}的等式表示一个铺设,

\begin{equation}
    x = k_{-} \oplus 1_{\vert} \oplus m * k_{-} \oplus (1) \oplus r
    \label{chapter4example}
\end{equation}

记$1_{\vert}$为一个竖着的骨牌,$k_{-}$为一个横着的骨牌,$=$表示铺设,$a \oplus b$表示两边骨牌(区域)的相接合,
$m * k_{-}$表示$m$个横着的骨牌的连续铺设,$(k)$表示$k$个待铺设区域,$r$表示不需要铺设的区域。

如等式\ref{chapter4example}表示长度为$x$的一行从左到右按顺序被
$1$个横着的骨牌,$1$个竖着的骨牌,$m$个横着的骨牌,$1$个待铺设格子,和$1$个缺陷格子的铺设。

简单的,根据以上规则命名我们可以得到如下的转化公式,即引理\ref{basicLemma41}。

\begin{lemma}
    $x =... \oplus k * 1_{\vert} \oplus ... \rightarrow x = ... \oplus k_{-} \oplus ...$
    \label{basicLemma41}
\end{lemma}
\begin{proof}
    对于边界上的一行,连续水平铺设$k$个竖着的骨牌和铺设一个横着的骨牌表现上都是能把连续k个格子铺满,
    而连续垂直铺设$k$个横着的骨牌和连续水平铺设$k$个竖着的骨牌所覆盖的区域是相同的,也就意味着,
    对于边界上的一行,连续水平铺设$k$个竖着的骨牌只是只铺设一个横着的骨牌的子情况,因此我们只需要考虑铺设一个横着的骨牌的情况,
    即$x =... \oplus k * 1_{\vert} \oplus ... \rightarrow x = ... \oplus k_{-} \oplus ...$。
\end{proof}

\section{棘手情况处理的前置}

我们先考虑长度为$x$的位于边界上的待铺一行,
\begin{theorem}
    对于一个棋盘的边界上的一行长度为$x$,骨牌的的大小为$1 \times k$,
    则所有的关于$x$铺设都可以转换为$x = m * k_{-} \oplus t * 1_{\vert} \oplus n * k_{-}, t < k$
    \label{basic-theorem-4}
\end{theorem}
\begin{proof}

    当长度$x < k$时,此时$m = n = 0$,有$x = x * 1_{\vert}$。

    当长度$x = k$时,一个横着的骨牌可以平铺一整行,有$x = k_{-}$,假若有竖着的骨牌,则有等式

    $x = (t) \oplus 1 \oplus (k - t - 1) = t * 1_{\vert} \oplus 1 \oplus (k - t - 1) * 1_{\vert}= k * 1_{\vert} = k_{-}$。

    此时考虑边界上的某个横着的骨牌,倘若他的左右都有一根竖着的骨牌,那么考虑第二行的这个区域,满足前述的长度为$x = k$的情况,
    因此等价于铺设一根横着的骨牌,重复考虑k次,则相当于连续铺设了k根横着的骨牌,也是属于连续铺设k根竖着的骨牌的子情况,
    因此我们能得到第二条转化公式,即引理\ref{basicLemma42}。

    \begin{lemma}
        $x = ... \oplus 1_{\vert} \oplus  k_{-} \oplus 1_{\vert} = ... \oplus (k + 2) * 1_{\vert} \oplus ...$
        \label{basicLemma42}
    \end{lemma}

    假设对于长度小于$x - 1$的行都成立时,则我们考虑长度为$wk$的一行两端各有$1_{\vert}$,$rk < x$,根据递推,我们有\ref*{basicEq43}。
    \begin{equation}
        x = ... \oplus 1_{\vert} \oplus  (wk) \oplus 1_{\vert} \oplus ... = ... \oplus 1_{\vert} \oplus  w * k_{-} \oplus 1_{\vert} \oplus ...
        \label{basicEq43}
    \end{equation}

    由于对于随后的每一行,我们都需要这么铺设,连续$k$次后,就等价于铺设了$wk \times k$的区域,与$wk * 1_{\vert}$表现相同,有哪次我们能得到第三条转化公式。

    \begin{lemma}
        $x = ... \oplus 1_{\vert} \oplus  w * k_{-} \oplus 1_{\vert} = ... \oplus (wk + 2) * 1_{\vert} \oplus ...$
        \label{basicLemma43}
    \end{lemma}

    我们考虑长度为$x$的行,有
    $$
        \begin{aligned}
                        & x = \sum_{i=0}^{a} (m_i * k_{-} \oplus n_i * 1_{\vert}) \oplus m_{a+1} * k_{-}                                                \\
            \rightarrow & x = m_1 * k_{-} \oplus n_1 * 1_{\vert} \oplus \sum_{i=1}^{a} (m_i * k_{-} \oplus n_i * 1_{\vert})\oplus m_{a+1} * k_{-}       \\
            \rightarrow & x = m_1 * k_{-} \oplus n_1 * 1_{\vert} \oplus \sum_{i=1}^{a} (m_ik * 1_{\vert} \oplus n_i * 1_{\vert}) \oplus m_{a+1} * k_{-} \\
            \rightarrow & x = m_1 * k_{-} \oplus n_1 * 1_{\vert} \oplus \sum_{i=1}^{a} (m_ik + n_i) * 1_{\vert} \oplus m_{a+1} * k_{-}                  \\
            \rightarrow & x = m_1 * k_{-} \oplus (x - m_1k - m_{a+1}k) * 1_{\vert} \oplus m_{a+1} * k_{-}                                               \\
            \rightarrow & x = m_1 * k_{-} \oplus n_1 * k_{-} \oplus t * 1_{\vert} \oplus n_2 * k_{-} \oplus m_{a+1} * k_{-}                             \\
            \rightarrow & x = m * k_{-} \oplus t * 1_{\vert} \oplus n * k_{-}.
        \end{aligned}
    $$

    故$\forall x, \exists m, t, n, s.t. x = m * k_{-} \oplus t * 1_{\vert} \oplus n * k_{-}, t < k$
\end{proof}

\section{棘手情况的正式处理}

不妨记$x_k$为棘手情况第$k$行的表示,我们很自然的能写出棘手情况的第一行的表示\ref{firstLine}。

\begin{equation}
    x_1 = (t_1k + 1) \oplus r \oplus ((t_2 - t_1)k - 2) \oplus r \oplus  (t_2k + 1)
    \label{firstLine}
\end{equation}

\begin{equation}
    \left\{
    \begin{aligned}
        t_1k + 1         & = (t_1 - u) * k_{-}  \oplus 1_{\vert} \oplus u * k_{-}                     \\
        (t_2 - t_1)k - 2 & = (t_2 - t_1 - 1 - u) * k_{-}  \oplus (k - 2) * 1_{\vert} \oplus u * k_{-} \\
        t_2k + 1         & = (t - t_2 - u) * k_{-}  \oplus 1_{\vert} \oplus u * k_{-}
    \end{aligned}
    \right.
    \label{first-difficult-sep}
\end{equation}

由定理\ref*{basic-theorem-4},我们得到式子\ref*{first-difficult-sep},因此第一行的铺设为:

$x_1 = (t_1 - u) * k_{-}  \oplus 1_{\vert} \oplus u * k_{-} \oplus r \oplus  (t_2 - t_1 - 1 - u) * k_{-}  \oplus
    (k - 2) * 1_{\vert} \oplus u * k_{-} \oplus r \oplus (t - t_2 - u) * k_{-}  \oplus 1_{\vert} \oplus u * k_{-}$

对于棋盘的两端,我们可以看做是一条无限长的竖着铺的骨牌,我们用$-1_{\vert}$表示向两端借铺设一个竖着的骨牌,
则可以再次进行转化化简式子:
$$
    \begin{aligned}
                    & x_1 = (t_1 - u) * k_{-}  \oplus 1_{\vert} \oplus u * k_{-} \oplus r \oplus ...                                    \\
                    & \oplus r \oplus (t - t_2 - u) * k_{-}  \oplus 1_{\vert} \oplus u * k_{-}                                          \\
        \rightarrow & x_1 = -1_{\vert} \oplus 1_{\vert} \oplus (t_1 - u) * k_{-}  \oplus 1_{\vert} \oplus u * k_{-} \oplus r \oplus ... \\
                    & \oplus r \oplus (t - t_2 - u) * k_{-}  \oplus 1_{\vert} \oplus u * k_{-} \oplus 1_{\vert} \oplus (-1_{\vert})     \\
        \rightarrow & x_1 = -1_{\vert} \oplus ((t_1 - u)k + 2)  * 1_{\vert} \oplus u * k_{-} \oplus r \oplus ...                        \\
                    & \oplus r \oplus (t - t_2 - u) * k_{-}  \oplus  (uk + 2) * 1_{\vert} \oplus (-1_{\vert})                           \\
        \rightarrow & x_1 = -1_{\vert} \oplus 2 * 1_{\vert} \oplus (t_1 - u) * k_{-}  \oplus u * k_{-} \oplus r ...                     \\
                    & r \oplus (t - t_2 - u) * k_{-}  \oplus  u * k_{-} \oplus 2 * 1_{\vert} \oplus (-1_{\vert})                        \\
        \rightarrow & x_1 = 1_{\vert} \oplus  t_1 * k_{-} \oplus r ... \oplus r \oplus (t - t_2) * k_{-} \oplus 1_{\vert}
    \end{aligned}
$$

显然的,第一行的所有$1_{\vert}$会同时占用第$1$行到$k$行的所有对应的格子。现在我们考虑第二行,因此我们可以把所有第一行的$1_{\vert}$在第二行的铺设中替换为$r$,因此得到第二行的表示\ref{secondLine}。

\begin{equation}
    x_2 = r \oplus ((t_2 - 1 - u)k + 1) \oplus (k - 2) * r \oplus ((t - t_2 + u)k + 1) \oplus r
    \label{secondLine}
\end{equation}


\begin{equation}
    \left\{
    \begin{aligned}
        (t_2 - 1 - u)k + 1 & = (t_2 - 1 - u - u^{'}) * k_{-} \oplus 1_{\vert} \oplus u^{'} *  k_{-}     \\
        (t - t_2 + u)k + 1 & = (t - t_2 + u - u^{''}) *  k_{-} \oplus 1_{\vert} \oplus  u^{''} *  k_{-}
    \end{aligned}
    \right.
    \label{second-difficult-sep}
\end{equation}

由定理\ref*{basic-theorem-4},有式子\ref*{second-difficult-sep},因此第二行可以分解为

$x_2 =  r \oplus  (t_2- 1 - u - u^{'}) * k_{-} \oplus 1_{\vert} \oplus  u^{'} * k_{-}  \oplus (k - 2) * r \oplus
    (t - t_2 + u - u^{''}) * k_{-} \oplus 1_{\vert} \oplus  u^{''} * k_{-} \oplus r$

由于第二行的所有$1_{\vert}$也会同时占用第$2$行到$k + 1$行的所有对应的格子。而$k \ge 3$ 所以在第一行和第二行的所有$1_{\vert}$在第三行的铺设中仍需替换为$r$,我们就写出了第三行的铺设表示\ref{thirdLine}。

\begin{equation}
    x_3 =  r \oplus  ((t_2- 1 - u - u^{'})k) \oplus r \oplus  (u^{'}k) \oplus (k - 2) * r \oplus
    ((t - t_2 + u - u^{''})k) \oplus r \oplus  (u^{''}k) \oplus r
    \label{thirdLine}
\end{equation}

由定理\ref*{basic-theorem-4},有\ref*{last-difficult-sep}:

\begin{equation}
    \left\{
    \begin{aligned}
        (t_2 - 1 - u - u^{'})k  & = (t_2- 1 - u - u^{'}) * k_{-}   \\
        (t - t_2 + u - u^{''})k & = (t - t_2 + u - u^{''}) * k_{-} \\
        u^{'}k                  & = u^{'} * k_{-}                  \\
        u^{''}k                 & = u^{''} * k_{-}
    \end{aligned}
    \right.
    \label{last-difficult-sep}
\end{equation}

故第三行已经不需要铺设任何竖着的骨牌,对于第$k^{'}\in [3, k]$行,我们全部像可以第三行一样考虑和铺设。

此时前$k$行已经全部铺设完毕,最后我们考虑第$k + 1$行,只有第$2$行的$1_{\vert}$会影响到第$k+1$行,因此第$k+1$行的铺设表示为等式\ref{lastLine},恰好对应第一行的铺设,因此也就意味着,$(tk + 2) \times (t^{'}k + 1)$的棘手情况与$(t^{'}k + 1 - k) \times (tk + 2)$中的某种棘手情况对应。

\begin{equation}
    x_{k+1} =  (t^{'}_1k + 1) \oplus r \oplus  ((t^{'}_2 - t^{'}_1)k - 2) \oplus r \oplus  (t^{'}_2k + 1)
    \label{lastLine}
\end{equation}

如此反复归纳,我们可以得到,所有棘手情况都和$1 \times (tk + 2)$的某种棘手情况相对应,而在$1 \times (tk + 2)$的棘手情况中,我们是无法平铺的,因此所有的棘手情况,我们都无法平铺。

\clearpage