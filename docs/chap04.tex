\chapter{棘手情况的处理}


\section{棘手情况处理的前置}
在上一节,我们遗留了一个情况没有处理,也就是当缺陷坐标$(i_1, j_1)$和$(i_2, j_2)$满足$i_1 = i_2 = 1$或$i_1 = i_2 = m$时,$j_2 < j_1$,该情况我们称为棘手情况,这一节我们将讨论在此情况下,棋盘不可被平铺。

我们先记一行的长度为x,骨牌大小为$1 \times k$,则x可以分为若干个k和若干个1的相加,如$x = k + 1 + k$等价于这一行按顺序为1个横着的骨牌,一个竖着的骨牌,一个横着的骨牌的平铺。记$n * 1$为n根竖着的并排放置的骨牌,$m * k$为m根横着并排放置的骨牌。

我们先定义一个平铺方案的转化
\begin{definition}
    我们称一个平铺方案A可以转化为染色B,当且仅当对于存在一种基于平铺方案B的平铺方案C,平铺方案C已经铺设的格子与平铺方案A相同。
\end{definition}

由于k个竖着的骨牌平铺等价于k个横着的骨牌平铺,因此对于处于边界上的一行来说,铺设k个竖着的骨牌和一个横着的骨牌表现是相同的,因此也就意味着对于$x =... + k * 1 + ... \rightarrow x = ... + 1 * k + ...$

而假设对于边界上的某个横着的骨牌,他的左右都有一根竖着的骨牌,则此时递归的对他铺设横着的骨牌,铺满k根,则此时相当于铺设了k跟竖着的骨牌,也就意味着$x = ... + 1 + k + 1 = ... + 1 + k * 1 + 1 + ...= ... + (k + 2) * 1 + ...$

% 不妨假设棘手情况的缺陷坐标的$(i_1, j_1)$和$(i_2, j_2)$满足$i_1 = i_2 = 1$,以便我们后续讨论。我们先提出一个引理:
\begin{theorem}
    对于一个棋盘的边界上的一行长度为x,骨牌的的大小为$1 \times k$,则x一定可以分解为$x = m * k + t * 1 + n * k, m, t < k$

    \label{basic-theorem-4}
\end{theorem}
\begin{proof}
    当长度$x < k$时,此时$m = n = 0$,有$x = x * 1$。

    当长度$x = k$时,一个横着的骨牌可以平铺一整行,有$x = 1 * k$,假若有竖着的骨牌则有等式$x = 1 + (x - 1) = 1 + (x - 1) * 1 = x * 1 = 1 * x = 1 * k$,符合。

    当长度$x > k$,则有
    $$
        \begin{aligned}
             & x = \sum_{i=0}^{a} (m_i * k + n_i * 1)     + m_{a+1} * k                          \\
             & = m_1 * k + n_1 * 1 + \sum_{i=1}^{a} (m_i * k + n_i * 1)+ m_{a+1} * k             \\
             & = m_1 * k + n_1 * 1 + \sum_{i=1}^{a} ((m_i \times k) * 1 + n_i * 1) + m_{a+1} * k \\
             & = m_1 * k + n_1 * 1 + \sum_{i=1}^{a} ((m_i \times k) + n_i) * 1 + m_{a+1} * k     \\
             & = m_1 * k + (x - m_1 \times k - m_{a+1} \times k) * 1 + m_{a+1} * k               \\
             & = m_1 * k + n_1 * k + t * 1 + n_2 * k + m_{a+1} * k                               \\
             & = m * k + t * 1 + n * k.
        \end{aligned}
    $$
\end{proof}

\section*{处理}
故对于棘手情况的第一行,不妨假设棘手情况的缺陷坐标的$(i_1, j_1)$和$(i_2, j_2)$满足$i_1 = i_2 = 1, j_1 = t_1 \times k + 2, j_2 = t_2 \times k + 1$,且长度$x = tk + 2$, 以便我们后续讨论。

那么分别考虑第一行中第1到第$j_1 - 1$个方格,第$j_1 + 1$ 到$j_2 - 1$, $j_2 + 1$ 到$x$的方格,由\ref*{basic-theorem-4},他们可以分别分解为$j_1 - 1 = (t_1 - u) * k  + 1 * 1 + u * k$,$j_2 - j_1 - 3 = (t_2 - t_1 - 1 - u) * k  + (k - 2) * 1 + u * k$,$x - j_2 = (t - t_2 - u) * k  + 1 * 1 + u * k$,即第一行可以分解为(r代表缺陷)

$x = (t_1 - u) * k  + 1 * 1 + u * k + r +  (t_2 - t_1 - 1 - u) * k  + (k - 2) * 1 + u * k + r + (t - t_2 - u) * k  + 1 * 1 + u * k$

由于对于边界来说,我们可以看做是一条无限长的竖着铺的骨牌,因此可以通过操作转化,重写上述等式为

$x =  1 * 1 +  (t_1 - u) * k  + u * k + r +  (t_2 - t_1 - 1 - u) * k  + (k - 2) * 1 + u * k + r + (t - t_2 - u) * k  + u * k + 1 * 1$

也就是

$x =  1 * 1 +  t_1 * k + r +  (t_2 - t_1 - 1 - u) * k  + (k - 2) * 1 + u * k + r + (t - t_2) * k + 1 * 1$

对于每一根竖着铺的骨牌,他就是下一行的缺陷,因此对于第二行,我们等价于考虑两条长度为$(t_2- 1 - u) \times k + 1$和$(t - t_2 + u) \times k + 1$的需要平铺的行,由\ref*{basic-theorem-4},他们可以分别分解为$(t_2- 1 - u) \times k + 1 = (t_2- 1 - u - u^{'}) * k + 1* 1 +  u^{'} * k)$和$(t - t_2 + u) \times k + 1 = (t - t_2 + u - u^{''}) * k + 1 * 1 +  u^{''} * k$,则第二行可以分解为

$x =  1 * r +  (t_2- 1 - u - u^{'}) * k + 1* 1 +  u^{'} * k  + (k - 2) * r + (t - t_2 + u - u^{''}) * k + 1 * 1 +  u^{''} * k + 1 * 1$

由于对于第二行的缺陷来说,该缺陷所对应的列直到第k行仍然是缺陷,所以对于第三行来说,我们只剩下两条长为$(t_2- 1 - u - u^{'}) \times k$和$(t - t_2 + u - u^{''}) \times k$的行,我们可以分别全部使用横着的骨牌平铺直到第k行,如此前k行就已经全部平铺完毕。

此时考虑第k+1行的缺陷所在位置,只有第2行的竖着的骨牌铺设会影响到第k+1行,第二行的竖着的骨牌所在位置分别为$2 + (t_2- 1 - u - u^{'}) \times k$和$x - u^{''} \times k - 2$,故假设原先的棋盘大小为$(ka+1) \times (kb + 2)$,则这个棋盘中的一种棘手情况将对应着$(ka+1 -k) \times (kb + 2)$中的某种棘手情况,如此反复使用,也就意味着所有棘手情况都和$1 \times (kb + 2)$的某种棘手情况相对应,而只有一行的棘手情况中,我们是无法平铺的,因此也就是所有的棘手情况,我们都无法平铺\cite{liu2011sift}

\clearpage