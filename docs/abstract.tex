%%
% 摘要信息
% 本文档中前缀"c-"代表中文版字段, 前缀"e-"代表英文版字段
% 摘要内容应概括地反映出本论文的主要内容,主要说明本论文的研究目的、内容、方法、成果和结论。要突出本论文的创造性成果或新见解,不要与引言相 混淆。语言力求精练、准确,以 300—500 字为宜。
% 在摘要的下方另起一行,注明本文的关键词(3—5 个)。关键词是供检索用的主题词条,应采用能覆盖论文主要内容的通用技术词条(参照相应的技术术语 标准)。按词条的外延层次排列(外延大的排在前面)。摘要与关键词应在同一页。
% modifier: 黄俊杰(huangjj27, 349373001dc@gmail.com)
% update date: 2017-04-15
%%

\cabstract{
    由古老的国际象棋游戏演变进化而来的骨牌平铺问题,又称棋盘覆盖问题,是组合数学中历史悠久的经典问题。
    最初的研究是关于$2$格多米诺骨牌在 $8 \times 8$ 的国际象棋棋盘上的覆盖问题,逐步推广到了k格骨牌在$m \times n$阶的数学棋盘上的覆盖问题。
    
    对棋盘覆盖问题的研究由来已久,冯跃峰证明了线型骨牌在完整矩形棋盘的完全覆盖充要条件;康庆德用染色法证明了残缺棋盘可被$1 \times 2$多米诺骨牌完全覆盖的充要条件;张媛则用轨道循环标号法证明了线型骨牌在残缺
    一格棋盘上的覆盖问题。

    而关于棋盘覆盖实际问题的研究的应用,宗传明则发现骨牌覆盖问题可应用于地砖铺设当中,不仅限于线型骨牌,还存在多边形骨牌覆盖,并且完整刻画了能构成六重晶格铺砌的所有铺砖,而经典的俄罗斯方块的拼图小游戏,
    文件加密技术等都会应用到棋盘覆盖问题的相关结论。
    
    本文在残缺一格棋盘的平铺问题的基础上进行扩展,利用循环染色法,构建出残缺两格棋盘的平铺问题的解,并利用分治的思想构建了对边界条件上的棘手情况的处理,成功解决了这个问题。
}
% 中文关键词(每个关键词之间用“,”分开,最后一个关键词不打标点符号。)
\ckeywords{本科毕业论文,染色方法,轮换,中山大学}

\eabstract{
    % 英文摘要及关键词内容应与中文摘要及关键词内容相同。中英文摘要及其关键词各置一页内。
    The content of the English abstract is the same as the Chinese abstract, 250-400 content words are appropriate. Start another line below the abstract to indicate English keywords (Keywords 3-5).
}
% 英文文关键词(每个关键词之间用,分开, 最后一个关键词不打标点符号。)
\ekeywords{undergraduate thesis, Sun Yat-Sen University}

