%%
% 摘要信息
% 本文档中前缀"c-"代表中文版字段, 前缀"e-"代表英文版字段
% 摘要内容应概括地反映出本论文的主要内容,主要说明本论文的研究目的、内容、方法、成果和结论。要突出本论文的创造性成果或新见解,不要与引言相 混淆。语言力求精练、准确,以 300—500 字为宜。
% 在摘要的下方另起一行,注明本文的关键词(3—5 个)。关键词是供检索用的主题词条,应采用能覆盖论文主要内容的通用技术词条(参照相应的技术术语 标准)。按词条的外延层次排列(外延大的排在前面)。摘要与关键词应在同一页。
% modifier: 黄俊杰(huangjj27, 349373001dc@gmail.com)
% update date: 2017-04-15
%%

\cabstract{
    由古老的国际象棋游戏演变进化而来的骨牌平铺问题,又称棋盘覆盖问题,是组合数学中历史悠久的经典问题,目前对于缺陷棋盘的直线骨牌覆盖问题中,只研究到了残缺一格的直线骨牌覆盖问题。

    本文在残缺一格棋盘的平铺问题的解法基础上进行扩展,利用循环染色法,构建出残缺两格棋盘的直线骨牌平铺问题的解并给出例子,对于边界条件上的不容易处理的问题,本文利用了分治的思想,
    结合数学归纳和无穷递降的思想,把边界情况上的问题缩小到规模更小的问题来证明了边界条件上的平铺的不可行,较为完整的解决了残缺两格棋盘的直线骨牌平铺问题。

    本文在以往的文献基础上,通过拓展的循环染色法有助于解决更多残缺格子的直线骨牌覆盖问题,而在边界条件上的无穷递降的解法的构建,为各类骨牌覆盖问题的边界条件提供了一个例子以供拓展和使用。
}
% 中文关键词(每个关键词之间用“,”分开,最后一个关键词不打标点符号。)
\ckeywords{本科毕业论文,染色方法,分治,无穷递降,中山大学}

\eabstract{
    % 英文摘要及关键词内容应与中文摘要及关键词内容相同。中英文摘要及其关键词各置一页内。
    The domino tiling problem, also known as the board coverage problem, evolved from the ancient chess game, is a classic problem with a long history in combinatorics, and only the problem of incomplete straight domino coverage has been studied in the problem of straight domino coverage of defective chessboards.

    This paper expands on the solution of the tiling problem of the incomplete one-grid chessboard, and uses the circular staining method to construct the solution of the rectilinear domino tiling problem of the incomplete two-square chessboard and gives examples.
    Combined with mathematical induction and the idea of infinite descension, the problem of boundary case is reduced to a smaller scale problem to prove that tiling on boundary conditions is not feasible, and the problem of rectilinear domino tiling with a broken two-square chessboard is solved more completely.

    Based on the previous literature, the extended cyclic staining method helps to solve the problem of linear domino coverage of more incomplete lattices, and the construction of an infinitely decreasing solution on the boundary condition provides an example for the boundary conditions of various domino cover problems for extension and use.
}
% 英文文关键词(每个关键词之间用,分开, 最后一个关键词不打标点符号。)
\ekeywords{undergraduate thesis, stain, partition, infinite descending, Sun Yat-Sen University}

