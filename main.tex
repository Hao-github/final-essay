% 可自定义论文时间戳 \today
% \year=2021
% \month=5
% \day=20
\documentclass[fontset=windows]{sysuthesis} % 默认使用电子版(不填充空白页)。如果需要双面打印版,请注释掉本行并启用下一行
% \documentclass[print-both-sides]{sysuthesis} % 使用双面打印版(填充额外空白页以保证每一章开头都在奇数页)
\usepackage{sysucode}  % 在论文中使用代码
\usepackage{subcaption}
\usepackage{multirow}
\usepackage{pgfplots}
\pgfplotsset{compat=1.18}
%%
% 论文相关信息
% 本文档中前缀"c-"代表中文版字段, 前缀"e-"代表英文版字段
% modifyer: 黄俊杰(huangjj27, 349373001dc@gmail.com)
% update date: 2017-04-13
%%

% 标题

% 封面标题。由于技术所限,封面题目过长的划分交由用户您进行定夺
% 这也能让您的论文封面看起来更有美感
\covertitlefirst{在缺陷两格的棋盘上}
\covertitlesecond{的直线骨牌平铺问题}

% Author:   Souler Ou
% 修改者:    欧一锋
% Date:     3/30/2018
% Mail:     ou@souler.cc
%如果英文标题过长可以使用此两项作为表三(答辩记录表)的标题。
\etitlefirst{SYSU Graduation Thesis}

% 中文标题
\ctitle{中山大学本科毕业论文}
\etitle{SYSU Graduation Thesis}

% 作者详细信息
\author{卢皓斌}
\cauthor{卢\ 皓\ 斌}    % 封面作者
\eauthor{Lu Haobin}
\studentid{19337077}
\cschool{数学学院(珠海)}

\cmajor{信息与计算科学}
\emajor{Information and Computing Science}

% 指导老师
\cmentor{崔潇易}
\ementor{Prof. 崔潇易}

     % 论文相关信息
\input{docs/proposal}   % 开题报告内容
%%
% 摘要信息
% 本文档中前缀"c-"代表中文版字段, 前缀"e-"代表英文版字段
% 摘要内容应概括地反映出本论文的主要内容,主要说明本论文的研究目的、内容、方法、成果和结论。要突出本论文的创造性成果或新见解,不要与引言相 混淆。语言力求精练、准确,以 300—500 字为宜。
% 在摘要的下方另起一行,注明本文的关键词(3—5 个)。关键词是供检索用的主题词条,应采用能覆盖论文主要内容的通用技术词条(参照相应的技术术语 标准)。按词条的外延层次排列(外延大的排在前面)。摘要与关键词应在同一页。
% modifier: 黄俊杰(huangjj27, 349373001dc@gmail.com)
% update date: 2017-04-15
%%

\cabstract{
    由古老的国际象棋游戏演变进化而来的骨牌平铺问题,又称棋盘覆盖问题,是组合数学中历史悠久的经典问题。
    最初的研究是关于$2$格多米诺骨牌在 $8 \times 8$ 的国际象棋棋盘上的覆盖问题,逐步推广到了k格骨牌在$m \times n$阶的数学棋盘上的覆盖问题。
    
    对棋盘覆盖问题的研究由来已久,冯跃峰证明了线型骨牌在完整矩形棋盘的完全覆盖充要条件;康庆德用染色法证明了残缺棋盘可被$1 \times 2$多米诺骨牌完全覆盖的充要条件;张媛则用轨道循环标号法证明了线型骨牌在残缺
    一格棋盘上的覆盖问题。

    而关于棋盘覆盖实际问题的研究的应用,宗传明则发现骨牌覆盖问题可应用于地砖铺设当中,不仅限于线型骨牌,还存在多边形骨牌覆盖,并且完整刻画了能构成六重晶格铺砌的所有铺砖,而经典的俄罗斯方块的拼图小游戏,
    文件加密技术等都会应用到棋盘覆盖问题的相关结论。
    
    本文在残缺一格棋盘的平铺问题的基础上进行扩展,利用循环染色法,构建出残缺两格棋盘的平铺问题的解,从而解决了该问题。
}
% 中文关键词(每个关键词之间用“,”分开,最后一个关键词不打标点符号。)
\ckeywords{本科毕业论文,染色方法,轮换,中山大学}

\eabstract{
    % 英文摘要及关键词内容应与中文摘要及关键词内容相同。中英文摘要及其关键词各置一页内。
    The content of the English abstract is the same as the Chinese abstract, 250-400 content words are appropriate. Start another line below the abstract to indicate English keywords (Keywords 3-5).
}
% 英文文关键词(每个关键词之间用,分开, 最后一个关键词不打标点符号。)
\ekeywords{undergraduate thesis, Sun Yat-Sen University}

     % 摘要内容
\input{docs/grading}    % 成绩评定记录表评语
\input{docs/progress}   % 过程检查报告数据
\begin{document}
\setlength{\headheight}{13pt}
% 论文前置部分
\frontmatter
\pagenumbering{Roman}
\makeUndergraduateCover    % 封面
% \makeProgressCheck  % 过程检查记录表
% \makeDefenseRecord  % 答辩情况等级表
\makedisclaim       % 学术诚信声明
\makeabstract       % 中英文摘要
\maketableofcontents        % 目录
\makelistoffiguretable

% 论文主体部分
\mainmatter
% 引言

% 正文
\chapter{绪论}
%定义,过去的研究和现在的研究,意义,与图像分割的不同
棋盘完全覆盖问题(problem of perfect cover of chessboard)是一类组合问题,是研究对于一个$m \times n$的广义棋盘, 在缺陷$z$格的情况下(通常$z=0$), 被骨牌完全覆盖的问题。

在过去,我们对此问题的研究主要落在棋盘完全覆盖问题的方案数,却甚少对棋盘完全覆盖问题的可解性进行考虑,显然,只有在确保该缺陷棋盘可被覆盖的基础上,
棋盘完全覆盖的方案数的研究才有价值。此外,研究棋盘完全覆盖问题的可解性,有助于对现实中的地砖覆盖等问题提供参考。

\section{选题背景与意义}
\label{sec:background}
% What is the problem
% why is it interesting and important
% Why is it hards, why do naive approaches fails
% why hasn't it been solved before
% what are the key components of my approach and results, also include any specific limitations,do not repeat the abstract
%contribution
我们称某$1 \times k$骨牌能完全覆盖某缺陷棋盘,当且仅当以下条件满足:
\begin{enumerate}
    \item 每块骨牌能够连续覆盖棋盘上同一行或者同一列的相邻$k$格
    \item 棋盘上每一格都被骨牌覆盖。
    \item 没有两块骨牌同时覆盖一格。
    \item 对于棋盘的缺陷处,没有任意一张骨牌将其覆盖。
\end{enumerate}

而本文研究的缺陷两格棋盘的覆盖问题,则是研究在$m \times n$的广义棋盘上,挖去任意两格的格子,是否可被$1 \times k$的骨牌所覆盖。

对于此问题而言,我们一般会使用染色法求解,也能够在经典问题:国际棋盘挖去做左上右下两个格子是否可被$1 \times 2$骨牌平铺的问题上做出相当漂亮的证明。

可是对于高维度而言,普通的染色法有点力不从心,只能排除较为基础的情况而无法对整个棋盘的各种缺陷情况进行完整的考量。

在本文中,我推广了染色法进行更多情况的排除,得到了在普遍条件下,缺陷两格棋盘是否可被平铺的证明,

\section{国内外研究现状和相关工作}
\label{sec:related_work}
对棋盘覆盖问题的研究由来已久,冯跃峰证明了线型骨牌在完整矩形棋盘的完全覆盖充要条件;康庆德用染色法证明了残缺棋盘可被$1 \times 2$多米诺骨牌完全覆盖的充要条件;张媛则用轨道循环标号法证明了线型骨牌在残缺
一格棋盘上的覆盖问题。

而关于棋盘覆盖实际问题的研究的应用,宗传明则发现骨牌覆盖问题可应用于地砖铺设当中,不仅限于线型骨牌,还存在多边形骨牌覆盖,并且完整刻画了能构成六重晶格铺砌的所有铺砖,而经典的俄罗斯方块的拼图小游戏,
文件加密技术等都会应用到棋盘覆盖问题的相关结论。
\section{本文的论文结构与章节安排}

\label{sec:arrangement}

本文共分为五章,各章节内容安排如下:

第一章绪论。简单说明了本文章的选题背景与意义。

第二章构建出可被$1 \times k$大小的骨牌平铺的棋盘的必要条件,初步构建出满足条件的棋盘的大小。

第三章展示了在大部分情况下,残缺格子位于哪些位置可以满足$1 \times 3$大小的骨牌进行平铺,并给出了相对应的平铺方式。

第四章对于棘手的情况,给出了这类缺陷棋盘不可被平铺的证明。

第五章总结了问题的解,并展望以后的工作。


\newclearpage
\chapter{棋盘所满足的条件}

对一般的$m \times n$的完整矩形棋盘, 冯跃峰\cite{fengBasicTheorem}给出了$k$格线型骨牌完全覆盖的充分必要条件, 这是我们研究覆盖问题的基础。

\begin{lemma}
    \label{basic-lemma-1}
    $m \times n$棋盘中存在$k$格线型骨牌完全覆盖,当且仅当$k \mid m$或$k \mid n$。
\end{lemma}

而对于多格残缺棋盘,重点是确定棋盘的大小$(m, n)$和残缺格子所在的位置$(i, j)$,在这一章节我们先考虑棋盘的大小所满足的条件。

以下的符号均建立在整数集的基础上,由于康庆德已经研究完毕$1 \times 2$骨牌的覆盖\cite{ZRZZ199205013},因此我们的只研究骨牌长度大于3的情况。

显而易见的,我们能简单得到以下引理。
\begin{lemma}
    \label{basic-lemma-2}
    $m \times n$的广义棋盘中,挖去2个格子,存在$k$格线型骨牌完全覆盖的充分条件是$k \mid mn - 2$。
\end{lemma}
\begin{proof}
    简单的,假设存在t个线性骨牌能够完全覆盖整个棋盘,由于骨牌不会互相重复的覆盖同一个格子,
    所以所有骨牌所覆盖的格子数目应该要和棋盘的总的格子数目相同,即
    \begin{equation}
        tk = mn - 2 \rightarrow k \mid mn - 2
    \end{equation}
\end{proof}
我们给出以下定理,来规整棋盘的大小
\begin{theorem}
    \label{basic-theorem-1}
    当$m \times n$的缺陷两格广义棋盘可被$1 \times k$大小的骨牌平铺时,一定满足$m \equiv 1 (mod k), n \equiv 2 (mod k)$。
\end{theorem}

\begin{proof}
    \begin{align}
        \left\{
        \begin{aligned}
            m & =  kx_m + m_1, m_1 \in [0, k) \\
            n & =  kx_n + n_1, n_1 \in [0, k)
        \end{aligned}
        \right.
        \label{theBegin}
    \end{align}

    简单的,不妨设\ref{theBegin},并据此对大小相同的完整棋盘进行划分,得到表\ref{fig:first-picture}。

    \begin{table}[h]
        \centering
        \caption{对棋盘的初步划分}
        \begin{tabular}{|c|c|}
            \hline
            A: $kx_m \times kx_n$ & B: $m_1 \times kx_n$ \\
            \hline
            C: $kx_m \times n_1$  & D: $m_1 \times n_1$  \\
            \hline
        \end{tabular}
        \label{fig:first-picture}
    \end{table}

    由引理\ref{basic-lemma-1}, 模块A,B,C均可被$1 \times k$骨牌平铺,因此模块A,B,C的标记和染色我们都不需要考虑,只需要考虑模块D的标记和染色\cite{fengBasicTheorem}。
    不妨设$m_1 \le n_1$,并基于张媛对残缺一格棋盘的研究\cite{zhangOneStain}对模块D进行标记,有表\ref{fig:k-order-staining-example}。

    \begin{table}[ht]
        \centering
        \caption{k阶染色}
        \begin{tabular}{|c|c|c|c|c|c|c|c|}
            \hline
            1        & 2        & $\cdots$ & $\cdots$ & $m_1$     & $\cdots$ & $n_1 - 1$ & $n_1$           \\
            \hline
            2        & 3        & $\cdots$ & $\cdots$ & $m_1 + 1$ & $\cdots$ & $n_1$     & $n_1 + 1$       \\
            \hline
            $\vdots$ & $\vdots$ & $\ddots$ & $\ddots$ &           &          & $\vdots$  & $\vdots$        \\
            \hline
            $m_1$    & $\cdots$ & $n_1-1$  & $n_1$    & $\cdots$  & $\cdots$ & $\vdots$  & $m_1 + n_1 - 1$ \\
            \hline
        \end{tabular}
        \label{fig:k-order-staining-example}
    \end{table}

    \begin{table}[htbp]
        \centering
        \caption{对表\ref{fig:k-order-staining-example}的计数}
        \begin{tabular}{|c|c|c|c|c|c|c|c|c|c|c|c|c|}
            \hline
            $x$    & $1$ & $2$ & $3$ & $...$ & $m_1$ & $m_1 + 1$ & ... & $n_1$ & $n_1 + 1$ & $...$ & $m_1 + n_1 - 1$ \\
            \hline
            $g(x)$ & $1$ & $2$ & $3$ & $...$ & $m_1$ & $m_1$     & ... & $m_1$ & $m_1 - 1$ & $...$ & $1$             \\
            \hline
        \end{tabular}
        \label{count}
    \end{table}
    记$g(x)$标记为$x$的格子的个数,对表\ref{fig:k-order-staining-example}进行计数,得到表\ref{count},就可以简单的得到公式\ref{functionOfSymbolX}。
    \begin{equation}
        g(x)=\left\{
        \begin{aligned}
            x              & , x \in [1, m_1]             \\
            m_1            & , x \in (m_1, n_1)           \\
            m_1 + n_1  - x & , x \in [n_1, m_1 + n_1 - 1]
        \end{aligned}
        \right.
        \label{functionOfSymbolX}
    \end{equation}

    我们将所有标记为$x$的染色为$x \% k$,记$f(x)$为棋盘中染色为x的格子数目,有$f(x) = g(x) + g(x + k)$。而$n_1 < k \rightarrow n_1 + 2 \le k + 1$,故
    $$
        \begin{aligned}
            f(1) & = g(1) + g(k + 1)  \\
                 & \le 1 + g(n_1 + 2) \\
                 & = 1 + g(n_1) - 2   \\
                 & < g(n_1) = f(n_1)
        \end{aligned}
    $$
    因此我们能得到等式\ref{eq1}。

    \begin{equation}
        \label{eq1}
        f(m_1) = f(n_1) > f(1)
    \end{equation}

    现在我们考虑残缺两格的棋盘上的直线覆盖,由于每个直线骨牌会且仅会分别覆盖1个染色为$1, 2, ... ,k$的方块,
    因此残缺两格后的棋盘能被直线骨牌覆盖的充分条件是,$\forall p_1, p_2, f(p_1) = f(p_2)$
    因此根据残缺两格分布在同种颜色和不同颜色的可能上,需要满足以下两个条件之一:

    \begin{itemize}
        \item 条件1: $\exists p \in [0, k), s.t. f(p) = f(x) + 2, x \neq p , x \in [0, k)$
        \item 条件2: $\exists p_1, p_2 \in [0, k), s.t. f(p_1) = f(p_2) = f(x) + 1, x \neq p_1 \neq p_2, x \in [0, k)$
    \end{itemize}

    因此如果模块D存在,分情况讨论:

    \begin{enumerate}
        \item $m_1 < n_1 - 1$,由\ref{fig:k-order-staining-example}有$f(m_1) = f(n_1) - 1 = f(n_1) > f(1)$,$\exists p_1, p_2, p_3, s.t. f(p_1) = f(p_2) = f(p_3) > f(1)$,矛盾。
        \item $m_1 = n_1 - 1$ 由上述讨论我们可以得到,此时$p_1 = m_1$,$p_2 = n_1 = m_1 + 1$,满足条件2,
              而$f(1) = f(n_1) - 1 \rightarrow g(1) + g(k + 1) = f(n_1) - 1 \rightarrow g(k + 1) = g(n_1) - 2 \rightarrow n_1 = k - 1$,即$m_1 = k - 2, n_1 = k - 1$。

              假设对于所有的染色,都存在两个格子使得在所有的染色下,这两个格子都被染色为$p_1$和$p_2$,记$Color^k_p(i, j) = a$为第$i$行第$j$列在染色方案$p$下染色为$a$,
              任取一种染色方案$Color^k_p$:
              \begin{itemize}
                  \item 当$2 \mid k$时,由对称性,此时两个缺陷格子格子必然位于$(\frac{k}{2} - 1,\frac{k}{2}), (\frac{k}{2},\frac{k}{2})$,
                        由\ref{fig:k-order-staining-example}简单的得到$Color^k_p(\frac{k}{2} - 1,\frac{k}{2}) = 1, Color^k_p(\frac{k}{2},\frac{k}{2}) = 2$,
                        此时交换第一列和第$\frac{k}{2}$列,仍然是一个合法的染色$Color^k_{p^{'}}$,但是交换后的染色中,
                        由于$Color^k_p(1, 1) = 1$,因此交换后的$Color^k_{p^{'}}(\frac{k}{2} - 1,\frac{k}{2}) \neq Color^k_{p^{'}}(\frac{k}{2},\frac{k}{2}) \neq 1$,因此矛盾。
                  \item 当$2 \nmid k$时,由对称性,此时两个缺陷格子必然位于$(\frac{k - 1}{2},\frac{k - 1}{2}), (\frac{k - 1}{2},\frac{k + 1}{2})$,
                        由\ref{fig:k-order-staining-example}简单的得到$Color^k_p(\frac{k - 1}{2},\frac{k - 1}{2}) = 1, Color^k_p(\frac{k - 1}{2},\frac{k + 1}{2}) = 2$,
                        此时交换第一行和第$\frac{k - 1}{2}$行,仍然是一个合法的染色$Color^k_{p^{'}}$,但是交换后的染色中,
                        由于$Color^k_p(1, 1) = 1$,因此交换后的$Color^k_{p^{'}}(\frac{k - 1}{2},\frac{k - 1}{2}) \neq Color^k_{p^{'}}(\frac{k - 1}{2},\frac{k + 1}{2}) \neq 1$,也矛盾。
              \end{itemize}
              因此我们总能找到一种染色不满足条件,故矛盾。
        \item $m_1 = n_1$,由\ref{eq1},有$f(m_1) \ge f(1) + 1$
              \begin{enumerate}
                  \item $f(m_1) = f(1) + 1$,则有$f(x) = f(m_1) - 1, \forall x < k, x \neq m_1$, 既不满足条件1也不满足条件2,矛盾。
                  \item $f(m_1) > f(1) + 1$, 则有$f(m_1) = f(m_1 + 1) + 1 = f(m_1 + 2) + 2 \ge f(1)$, 故令$p = m_1$后,$f(p) \ge f(1) + 2$但$f(m_1) > f(1)$,矛盾。
              \end{enumerate}
    \end{enumerate}

    所以挖去两格后,若模块D仍然存在,则所有条件均矛盾,故模块D不存在,即此时$m_1 = 1, n_1 = 2$,证毕。
\end{proof}

因此,对于我们已经规整好了棋盘的规模,并把问题转化如下:

对于大小为$1 \times k$的骨牌,我们考虑缺陷两格的$m \times n$的棋盘,其中$m \equiv 1 (mod k), n \equiv 2 (mod k)$,哪些棋盘可以被平铺。

\clearpage
\newclearpage
\chapter{缺陷格子可能存在的位置与相对应的平铺方案}

\section{骨牌大小k = 3时的情况}
由\ref*{basic-theorem-1}有,此时棋盘的大小只可能为$m \times n, m \equiv 1 (mod 3), n \equiv 2 (mod 3)$,
不妨假设缺陷格子分别为$(i_1, j_1), (i_2, j_2)$。

在这里我们给出一个定理,来辅助我们之后的证明。

\begin{theorem}
	\label{basic-theorem-2}
	对于$m \times n$的广义棋盘和$1 \times k$的骨牌,且$m \equiv 1 (mod k), n \equiv 2 (mod k)$,
	记某格子的坐标为$(i, j)$,意味着某格子处于第$i$行第$j$列,则如果存在$i \times j$的矩形缺陷$C$,$i \equiv 1 (mod k), j \equiv 2 (mod k), i \le m, j \le n$,
	且$Corner_C = \left\{(3k_{11} + 1, 3k_{12} + 1), (3k_{21} + 1, 3k_{22} + 2)\right\}$,
	则该棋盘一定能被平铺。
\end{theorem}
\begin{proof}
	当存在这样的缺陷时,我们可以把棋盘分成如\ref*{fig:nine-separate}的9个区域,分别如图所示,由\ref*{basic-lemma-1},
	可得除了矩形缺陷C外的8个区域均可被$1 \times k$的骨牌平铺,证毕。

	\begin{table}[htbp]
		\centering
		\caption{对缺陷棋盘的划分}
		\begin{tabular}{|c|c|c|}

			\hline
			$3k_{11} \times 3k_{12}$               & $3k_{11} \times (3k_{22} - 3k_{12} + 2)$ & $3k_{11}\times 3k_{22} $               \\
			\hline
			$(3k_{21}-3k_{11} + 1) \times 3k_{12}$ & $C: i \times j$                          & $(3k_{21}-3k_{11} + 1) \times 3k_{22}$ \\
			\hline
			$3k_{21} \times 3k_{12} $              & $3k_{21}  \times 3(k_{22} - k_{12}) + 2$ & $3k_{21} \times 3k_{22}$               \\
			\hline
		\end{tabular}
		\label{fig:nine-separate}
	\end{table}
\end{proof}

由定理\ref*{basic-theorem-2},我们可以得把问题转化为:
对于某个残缺两格的棋盘,如果他的两个缺陷格子可以组合成一个满足\ref*{basic-theorem-2}中条件的矩形缺陷$C$,那么该棋盘可以被平铺。

在图\ref*{fig:k-order-staining-1}的基础上,我们对棋盘进行两种染色,分别记为$Color^3_1$和$Color^3_2$,
显然有$f(1) = f(2) = f(3) + 1$,并据此按照以下规则重新染色得到$Color^3_3$,即图\ref*{fig:3-order-staining-last}。

\begin{itemize}
	\item $Color^3_1(i, j) = Color^3_2(i, j) = 1$,则$Color^3_3(i, j) = 1$。
	\item $Color^3_1(i, j) = Color^3_2(i, j) = 2$,则$Color^3_3(i, j) = 2$。
	\item $Color^3_1(i, j) = 1, Color^3_2(i, j) = 2$,则$Color^3_3(i, j) = 4$。
	\item $Color^3_1(i, j) = 2, Color^3_2(i, j) = 1$,则$Color^3_3(i, j) = 5$。
	\item $Color^3_1(i, j) = 3$或 $Color^3_2(i, j) = 3$,则$Color^3_3(i, j) = 0$。
\end{itemize}

而该棋盘可被平铺的充分条件为缺陷的两个格子在$Color^3_1$和$Color^3_2$中都落在染色1和染色2的位置,即以下两个条件之一:

\begin{itemize}
	\item $Color^3_3(i_1, j_1) = 1, Color^3_3(i_2, j_2) = 2$
	\item $Color^3_3(i_1, j_1) = 4, Color^3_3(i_2, j_2) = 5$
\end{itemize}

\begin{table}[htbp]
	\caption{3阶染色}
	\label{fig:3-order-staining}
	\begin{subtable}{.5\linewidth}
		\centering
		\caption{3阶染色-1}
		\begin{tabular}{|c|c|c|c|c|c|}
			\hline
			1        & 2        & 3        & 1        & 2        & $\cdots$ \\
			\hline
			2        & 3        & 1        & 2        & 3        & $\cdots$ \\
			\hline
			3        & 1        & 2        & 3        & 1        & $\cdots$ \\
			\hline
			1        & 2        & 3        & 1        & 2        & $\cdots$ \\
			\hline
			2        & 3        & 1        & 2        & 3        & $\cdots$ \\
			\hline
			$\vdots$ & $\vdots$ & $\vdots$ & $\vdots$ & $\vdots$ & $\ddots$ \\
			\hline
		\end{tabular}
		\label{fig:3-order-staining-1}
	\end{subtable}%
	\begin{subtable}{.5\linewidth}
		\centering
		\caption{3阶染色-2}
		\begin{tabular}{|c|c|c|c|c|c|}
			\hline
			1        & 2        & 3        & 1        & 2        & $\cdots$ \\
			\hline
			3        & 1        & 2        & 3        & 1        & $\cdots$ \\
			\hline
			2        & 3        & 1        & 2        & 3        & $\cdots$ \\
			\hline
			1        & 2        & 3        & 1        & 2        & $\cdots$ \\
			\hline
			3        & 1        & 2        & 3        & 1        & $\cdots$ \\
			\hline
			$\vdots$ & $\vdots$ & $\vdots$ & $\vdots$ & $\vdots$ & $\ddots$ \\
			\hline
		\end{tabular}
		\label{fig:3-order-staining-2}
	\end{subtable}
\end{table}

\begin{table}[htbp]
	\centering
	\caption{3阶染色-3}
	\begin{tabular}{|c|c|c|c|c|c|c|}
		\hline
		1        & 2        & 0        & 1        & 2        & 0        & $\cdots$ \\
		\hline
		0        & 0        & 4        & 0        & 0        & 4        & $\cdots$ \\
		\hline
		0        & 0        & 5        & 0        & 0        & 5        & $\cdots$ \\
		\hline
		1        & 2        & 0        & 1        & 2        & 0        & $\cdots$ \\
		\hline
		0        & 0        & 4        & 0        & 0        & 4        & $\cdots$ \\
		\hline
		0        & 0        & 5        & 0        & 0        & 5        & $\cdots$ \\
		\hline
		$\vdots$ & $\vdots$ & $\vdots$ & $\vdots$ & $\vdots$ & $\vdots$ & $\ddots$ \\
		\hline
	\end{tabular}
	\label{fig:3-order-staining-last}
\end{table}

\subsection{在染色中分别处于4和5}

由图\ref{fig:3-order-staining-last},我们可以简单的得到染色为4和5所对应的格子的坐标,

\begin{itemize}
	\item $Color^3_3(i_1, j_1) = Color^3_3(3k_{41} + 2, 3k_{42}) = 4$
	\item $Color^3_3(i_2, j_2) = Color^3_3(3k_{51} + 3, 3k_{52}) = 5$
\end{itemize}

由于棋盘具有对称性,在这里我们不妨假设$k_{51} \ge k_{41}$,则有以下2种情况:

\begin{table}[htbp]
	\centering
	\caption{$k_{52} \ge k_{42}$时的缺陷拼接}
	\begin{tabular}{|cc|c|cc|c|cc|}
		\hline
		\multicolumn{6}{|c|}{$1 \times (3k_{51} - 3k_{41} + 3)$}                & \multicolumn{2}{|c|}{\multirow{4}*{$(3k_{52} - 3k_{42} + 3) \times 2$}}                                                                                        \\
		\cline{1-6}
		\multicolumn{2}{|c|}{\multirow{4}*{$(3k_{52} - 3k_{42} + 3) \times 2$}} & 4                                                                       & \multicolumn{3}{|c|}{$1 \times (3k_{51} - 3k_{41})$}                       &   &     \\
		\cline{3-6}
		                                                                        &                                                                         & \multicolumn{4}{|c|}{$(3k_{52} - 3k_{42}) \times (3k_{51} - 3k_{41} + 1)$} &   &     \\
		\cline{3-6}
		                                                                        &                                                                         & \multicolumn{3}{|c|}{$1 \times (3k_{51} - 3k_{41})$}                       & 5 &   & \\
		\cline{3-8}
		                                                                        &                                                                         & \multicolumn{6}{|c|}{$1 \times (3k_{51} - 3k_{41} + 3)$}                             \\
		\hline
	\end{tabular}
	\label{fig:4-5-painting}
\end{table}

\begin{table}[htbp]
	\centering

	\caption{$k_{52} < k_{42}$的缺陷拼接}
	\begin{tabular}{|ccccc|cc|}
		\hline
		\multicolumn{5}{|c|}{\multirow{2}*{$2 \times (3k_{51} - 3k_{41} + 3)$}} & \multicolumn{2}{|c|}{\multirow{3}*{$3 \times 2$}}                                                                                                                                           \\
		                                                                        &                                                   &                                                                            &   &                                                   &  & \\
		\cline{1-5}
		\multicolumn{2}{|c|}{\multirow{3}*{$2 \times (3k_{42} - 3k_{52} + 3)$}} & 5                                                 & \multicolumn{2}{|c|}{$1 \times (3k_{51} - 3k_{41})$}                       &   &                                                        \\
		\cline{3-7}
		                                                                        &                                                   & \multicolumn{5}{|c|}{$(3k_{42} - 3k_{52}) \times (3k_{51} - 3k_{41} + 3)$}                                                              \\
		\cline{3-7}
		                                                                        &                                                   & \multicolumn{2}{|c|}{$1 \times (3k_{51} - 3k_{41})$}                       & 4 & \multicolumn{2}{|c|}{\multirow{3}*{$3 \times 2$}}      \\
		\cline{1-5}
		\multicolumn{5}{|c|}{\multirow{2}*{$2 \times (3k_{51} - 3k_{41} + 3)$}} &                                                   &                                                                                                                                         \\
		                                                                        &                                                   &                                                                            &   &                                                   &  & \\
		\hline
	\end{tabular}
	\label{fig:5-4-painting}
\end{table}

\begin{itemize}
	\item 在图\ref*{fig:4-5-painting}中,$Corner_C = \left\{(3k_{41} + 1, 3k_{42} - 2), (3k_{51} + 4, 3k_{52} + 2)\right\}$。
	\item 在图\ref*{fig:5-4-painting}中,$Corner_C = \left\{(3k_{51} + 1, 3k_{52} - 2), (3k_{41} + 4, 3k_{42} + 2)\right\}$。
\end{itemize}

由\ref*{fig:4-5-painting}和\ref*{fig:5-4-painting}可得,无论缺陷格子如何落在颜色4和颜色5上,
我们都能找到一种拼接方案,将两个缺陷拼接成满足\ref*{basic-theorem-2}中的条件的矩形缺陷。

\subsection{在染色中分别处于1和2}

同在染色中处于4和5的情况,我们可以写出染色为1和2所对应的格子坐标。

\begin{itemize}
	\item $Color^3_3(i_1, j_1) = Color^3_3(3k_{11} + 1, 3k_{12} + 1) = 1$
	\item $Color^3_3(i_2, j_2) = Color^3_3(3k_{21} + 1, 3k_{22} + 2) = 2$
\end{itemize}


由于对称性,在这里我们不妨假设$k_{21} \ge k_{11}$,则同样有以下两种情况:

\begin{table}[htbp]
	\centering
	\caption{$k_{22} \ge k_{12}$的缺陷拼接}
	\begin{tabular}{|ccc|c|}
		\hline
		1                                                                                    & \multicolumn{2}{|c|}{$1 \times (3k_{22} -3k_{12})$} & \multirow{2}*{$(3k_{21} - 3k_{11}) \times 1$}     \\
		\cline{1-3}
		\multicolumn{3}{|c|}{\multirow{2}*{$(3k_{21} - 3k_{11}) \times (3k_{22} -3k_{12})$}} &                                                                                                         \\
		\cline{4-4}
		                                                                                     &                                                     &                                               & 2 \\
		\hline
	\end{tabular}
	\label{fig:1-2-painting}
\end{table}

\begin{table}[htbp]
	\centering
	\caption{$k_{22} < k_{12}$且$k_{21} > k_{11}$的缺陷拼接}
	\begin{tabular}{|cc|ccc|cc|}
		\hline
		\multicolumn{2}{|c|}{\multirow{3}{*}{$(3k_{21} - 3k_{11}) \times 1$}} & 2 & \multicolumn{4}{|c|}{$1 \times  (3k_{12} - 3k_{22})$}                                                                                 \\
		\cline{3-7}
		                                                                      &   & \multicolumn{3}{c}{\multirow{2}{*}{...}}              & \multicolumn{2}{|c|}{\multirow{3}{*}{$(3k_{21} - 3k_{11}) \times 1$}}         \\
		                                                                      &   &                                                       &                                                                       &  &  & \\
		\cline{1-5}
		\multicolumn{4}{|c|}{$1 \times  (3k_{12} - 3k_{22})$}                 & 1 &                                                       &                                                                               \\
		\hline
		\label{fig:2-1-painting-1}
	\end{tabular}
\end{table}

\begin{itemize}
	\item 若$k_{22} \ge k_{12}$,有缺陷拼接图\ref*{fig:1-2-painting},\\
	      $Corner_C = \left\{(3k_{11} + 1, 3k_{12} + 1), (3k_{21} + 1, 3k_{22} + 2)\right\}$。
	\item 若$k_{22} < k_{12}$,有以下分类:
	      \begin{itemize}
		      \item 若$i_1 \neq i_2$,有缺陷拼接图\ref*{fig:2-1-painting-1},\\
		            $Corner_C = \left\{(3k_{21} + 1, 3k_{22} + 1), (3k_{11} + 1, 3k_{22} + 2)\right\}$。
		      \item 若$i_1 = i_2 \notin \left\{1, m\right\}$,有$m \ge 7$,当$n = 5$时,穷举得到无解;当$n \ge 8$时:
		            \begin{itemize}
			            \item 若$k_{12} - k_{22} = 1$,存在缺陷拼接图见附录,此时\\
			                  $Corner_C = \left\{(3k_{21} - 2, 3k_{22} + 1), (3k_{11} + 1, 3k_{22} + 2)\right\}$。
			            \item 若$k_{12} - k_{22} > 1$,存在缺陷拼接图见附录,此时\\
			                  $Corner_C = \left\{(3k_{21} - 2, 3k_{22} + 1), (3k_{11} + 4, 3k_{22} + 5)\right\}$。
		            \end{itemize}
		      \item 若$i_1 = i_2 \in \left\{1, m\right\}$时,我们称为棘手情况,把它放在下一节讨论。
	      \end{itemize}
\end{itemize}


除了棘手情况,无论缺陷如何分布在颜色1和颜色2上,我们们都能找到一种拼接方案,将两个缺陷拼接成满足\ref*{basic-theorem-2}中的条件的矩形缺陷。

因此由\ref*{basic-theorem-2},除棘手情况外,当缺陷分别落在颜色4和颜色5上,或者分别落在颜色1和颜色2上时,无论缺陷怎么分布,该缺陷棋盘都可被平铺。

\section{骨牌大小k > 3时的情况}

\begin{table}[h]
	\caption{k阶染色}
	\label{fig:k-order-staining}
	\begin{subtable}{.5\linewidth}
		\centering
		\caption{k阶染色-1}
		\begin{tabular}{|c|c|c|c|c|c|c|}
			\hline
			1        & 2        & 3        & $\cdots$ & k        & 1        & $\cdots$ \\
			\hline
			2        & 3        & 4        & $\cdots$ & 1        & 2        & $\cdots$ \\
			\hline
			3        & 4        & 5        & $\cdots$ & 2        & 3        & $\cdots$ \\
			\hline
			$\vdots$ & $\vdots$ & $\vdots$ & $\ddots$ & $\ddots$ & $\ddots$ & $\cdots$ \\
			\hline
			k        & 1        & 2        & $\cdots$ & k-1      & k        & $\cdots$ \\
			\hline
			1        & 2        & 3        & $\cdots$ & k        & 1        & $\cdots$ \\
			\hline
			$\vdots$ & $\vdots$ & $\vdots$ & $\ddots$ & $\vdots$ & $\vdots$ & $\ddots$ \\
			\hline
		\end{tabular}
		\label{fig:k-order-staining-1}
	\end{subtable}%
	\begin{subtable}{.5\linewidth}
		\centering
		\caption{k阶染色-2}
		\begin{tabular}{|c|c|c|c|c|c|c|}
			\hline
			1        & 2        & 3        & $\cdots$ & k        & 1        & $\cdots$ \\
			\hline
			k        & 1        & 2        & $\cdots$ & k-1      & k        & $\cdots$ \\
			\hline
			k-1      & k        & 1        & $\cdots$ & k-2      & k-1      & $\cdots$ \\
			\hline
			$\vdots$ & $\vdots$ & $\vdots$ & $\ddots$ & $\ddots$ & $\ddots$ & $\cdots$ \\
			\hline
			2        & 3        & 4        & $\cdots$ & 1        & 2        & $\cdots$ \\
			\hline
			1        & 2        & 3        & $\cdots$ & k        & 1        & $\cdots$ \\
			\hline
			$\vdots$ & $\vdots$ & $\vdots$ & $\ddots$ & $\vdots$ & $\vdots$ & $\ddots$ \\
			\hline
		\end{tabular}
		\label{fig:k-order-staining-2}
	\end{subtable}
\end{table}

\begin{table}[htbp]
	\centering
	\caption{$k \times k$的区域划分}
	\begin{tabular}{|c|c|}
		\hline
		A: $1 \times 2$       & B: $1  \times (k - 2)$     \\
		\hline
		C: $(k - 1) \times 2$ & D: $(k - 1)\times (k - 2)$ \\
		\hline
	\end{tabular}
	\label{fig:k-division}
\end{table}

在$k \ge 4$的情况下,我们可以给出以下定理。
\begin{theorem}
	\label{basic-theorem-3}
	当$m \times n$的缺陷两格广义棋盘可被$1 \times k$大小的骨牌平铺时,若$k \ge 4$,则缺陷格子分别只可能在坐标$(kx_1 + 1, ky_1 + 1)$和$(kx_2 + 1, ky_2 + 2)$处。
\end{theorem}
\begin{proof}
	我们能够很轻易的得到$f(1) = f(2) = f(x), x \neq 1, x \neq 2$,
	缺陷格子只能落在在所有染色组合中均被染色为1或染色2的格子,
	即\ref{basic-theorem-3-example}
	\begin{equation}
		Color^k_p(i_1, j_1) \in \left\{1, 2\right\}, Color^k_p(i_2, j_2) \in \left\{1, 2\right\} \forall p
		\label{basic-theorem-3-example}
	\end{equation}
	而$Color^k_p(kx_1 + 1, ky_1 + 1) = 1, Color^k_p(kx_2 + 1, ky_2 + 2) = 2, \forall p$,
	因此当$(i_1, j_1) = (kx_1 + 1, ky_1 + 1), (i_2, j_2) = (kx_2 + 1, ky_2 + 2)$时,棋盘可能可以被平铺。

	接下来我们证明,当$(i, j) \neq (kx_1 + 1, ky_1 + 1), (i, j) \neq (kx_2 + 1, ky_2 + 2)$时,$\exists p, s.t. Color^k_p(i, j) \notin \left\{1, 2\right\}$。

	挑选k阶染色中最经典的两种染色方案$Color^k_1$和$Color^k_2$(图\ref{fig:k-order-staining-1}和图\ref{fig:k-order-staining-2}),
	考虑最左上的$k \times k$大小的区域,做出表\ref{fig:k-division}的区域划分,随后分别考虑模块B, C和D:

	\begin{itemize}
		\item[模块B] 由于模块B不会染色为$1$或$2$,所以缺陷格子不可能落在模块B中。
		\item[模块C]
			\begin{itemize}
				\item $Color^k_1(2, 1) = 2, Color^k_2(2, 1) = k$
				\item $Color^k_1(2, 2) = 3, Color^k_2(2, 2) = 1$
				\item $Color^k_1(k, 1) = k, Color^k_2(k, 1) = 2$
				\item $Color^k_1(k, 2) = 1, Color^k_2(k, 2) = 3$
			\end{itemize}
			由于模块C在两种染色下只有共计4个$1$或$2$,故模块C中的所有格子,均不可能满足$Color^k_p(i, j) \in \left\{1, 2\right\}, \forall p$。
		\item[模块D] 对于任意$i$,不妨假定$Color^k_1(i, j_1) = 1, Color^k_1(i, j_2) = 2$,考虑$Color^k_1(i, y) = 1, y \in [2, k], y \notin \left\{j_1, j_2\right\}$,
			由于$k \ge 4$,故$\exists j^{'}, s.t. Color^k_1(i, j^{'}) = t \notin \left\{1, 2\right\}$,此时交换第$j_1$列和第$j^{'}$列,仍然是一个满足条件的染色方案$p$,
			但$Color^k_p(i, j) = Color^k_1(i, j^{'}) = t \notin \left\{1, 2\right\}$,矛盾。
	\end{itemize}
	因此缺陷格子只能落在模块A中,证毕。
\end{proof}

因此类似$k=3$的情况,我们只需考虑像这样的染色

\begin{table}[htbp]
	\centering
	\caption{k阶重染色}
	\begin{tabular}{|c|c|c|c|c|c|c|c|}
		\hline
		1        & 2        & 0        & $\cdots$ & 0        & 1        & 2        & $\cdots$ \\
		\hline
		0        & 0        & 0        & $\cdots$ & 0        & 0        & 0        & $\cdots$ \\
		\hline
		0        & 0        & 0        & $\cdots$ & 0        & 0        & 0        & $\cdots$ \\
		\hline
		$\vdots$ & $\vdots$ & $\vdots$ & $\ddots$ & $\vdots$ & $\vdots$ & $\vdots$ & $\ddots$ \\
		\hline
		0        & 0        & 0        & $\cdots$ & 0        & 0        & 0        & $\cdots$ \\
		\hline
		1        & 2        & 0        & $\cdots$ & 0        & 1        & 2        & $\cdots$ \\
		\hline
		0        & 0        & 0        & $\cdots$ & 0        & 0        & 0        & $\cdots$ \\
		\hline
		$\vdots$ & $\vdots$ & $\vdots$ & $\ddots$ & $\vdots$ & $\vdots$ & $\vdots$ & $\ddots$ \\
		\hline
	\end{tabular}
	\label{fig:k-order-staining-last}
\end{table}

同$k=3$的情况,我们只需要考虑染色分别处于1和2的情况,此时可以完全的套用$k=3$的情况,且棘手情况也一样,为$k>3$时的边界情况,我们在下一节一起讨论。

\clearpage
\newclearpage
\chapter{简单的使用例子}
\label{cha:usage-example}

本部分将会根据毕设论文的写作需要,放置相关的例子和代码段供大家参考,方便大家的论文写作,如果更多有用的Latex使用例子也会欢迎提出PR,贡献更多的例子。

\section{图像的插入}

\subsection{镶嵌在文中的图像}
\begin{wrapfigure}{r}{0.5\linewidth}
    \centering
    \includegraphics[width=0.5\textwidth]{image/chap04/confusion.pdf}
    \caption{镶嵌在文中的图像}
    \label{fig:image-embedding-text}
\end{wrapfigure}
论文主体是毕业论文的主要部分,必须言之成理,论据可靠,严格遵循本学科国际通行的学术规范。在写作上要注意结构合理、层次分明、重点突出,章节标题、公式图表符号必须规范统一。论文主体的内容根据不同学科有不同的特点,一般应包括以下几个方面: (1)毕业论文(设计)总体方案或选题的论证; (2)毕业论文(设计)各部分的设计实现,包括实验数据的获取、数据可行性及有效性的处理与分析、各部分的设计计算等; (3)对研究内容及成果的客观阐述,包括理论依据、创新见解、创造性成果及其改进与实际应用价值等; (4)论文主体的所有数据必须真实可靠,凡引用他人观点、方案、资料、数据等,无论曾否发表,无论是纸质或电子版,均应详加注释。自然科学论文应推理正确、结论清晰;人文和社会学科的论文应把握论点正确、论证充分、论据可靠,恰当运用系统分析和比较研究的方法进行模型或方案设计,注重实证研究和案例分析,根据分析结果提出建议和改进措施等。
论文主体是毕业论文的主要部分,必须言之成理,论据可靠,严格遵循本学科国际通行的学术规范。在写作上要注意结构合理、层次分明、重点突出,章节标题、公式图表符号必须规范统一。论文主体的内容根据不同学科有不同的特点,一般应包括以下几个方面: (1)毕业论文(设计)总体方案或选题的论证; (2)毕业论文(设计)各部分的设计实现,包括实验数据的获取、数据可行性及有效性的处理与分析、各部分的设计计算等; (3)对研究内容及成果的客观阐述,包括理论依据、创新见解、创造性成果及其改进与实际应用价值等; (4)论文主体的所有数据必须真实可靠,凡引用他人观点、方案、资料、数据等,无论曾否发表,无论是纸质或电子版,均应详加注释。自然科学论文应推理正确、结论清晰;人文和社会学科的论文应把握论点正确、论证充分、论据可靠,恰当运用系统分析和比较研究的方法进行模型或方案设计,注重实证研究和案例分析,根据分析结果提出建议和改进措施等。



\subsection{单张图像的插入}

\begin{figure}[ht]
    \centering
    \includegraphics[width=0.5\textwidth]{image/chap04/illustration/hole.pdf}
    \caption{单张图像}
    \label{fig:hole}
\end{figure}


\subsection{多张图像的并排插入}


\begin{figure}[ht]%文中的Grid-LSTM模型做的语义图像分割的例子
    \centering
    \includegraphics[width=.2\textwidth,height=.15\textwidth]{image/chap04/example/2007_000799.jpg}
    \includegraphics[width=.2\textwidth,height=.15\textwidth]{image/chap04/example/2007_002094.jpg}
    \includegraphics[width=.2\textwidth,height=.15\textwidth]{image/chap04/example/2007_004483.jpg}
    \includegraphics[width=.2\textwidth,height=.15\textwidth]{image/chap04/example/2007_003194.jpg}
    \\
    \includegraphics[width=.2\textwidth,height=.15\textwidth]{image/chap04/example/2007_000799.pdf}
    \includegraphics[width=.2\textwidth,height=.15\textwidth]{image/chap04/example/2007_002094.pdf}
    \includegraphics[width=.2\textwidth,height=.15\textwidth]{image/chap04/example/2007_004483.pdf}
    \includegraphics[width=.2\textwidth,height=.15\textwidth]{image/chap04/example/2007_003194.pdf}
    \caption{并排的多张图像}
    \label{fig:multi-image-example1}
\end{figure}


\begin{figure}[ht]
    \centering
    \makebox[0.11\textwidth]{\scriptsize 图像}
    \enspace
    \makebox[0.11\textwidth]{\scriptsize 真值}
    \enspace
    \makebox[0.11\textwidth]{\scriptsize CNN+5LSTM\textbf{1}}
    \enspace\thinspace
    \makebox[0.11\textwidth]{\scriptsize CNN+5LSTM\textbf{2}}
    \enspace\thinspace
    \makebox[0.11\textwidth]{\scriptsize CNN+5LSTM\textbf{3}}
    \enspace\thinspace
    \makebox[0.11\textwidth]{\scriptsize CNN+5LSTM\textbf{4}}
    \enspace\thinspace
    \makebox[0.11\textwidth]{\scriptsize CNN+5LSTM\textbf{5}}\\
    \includegraphics[width=0.11\textwidth]{image/chap04/improvement/2007_000663.jpg}
    \enspace\thinspace %\hfill
    \includegraphics[width=0.11\textwidth]{image/chap04/improvement/2007_000663.png}
    \enspace\thinspace
    \includegraphics[width=0.11\textwidth]{image/chap04/improvement/2007_000663_1.png}
    \enspace\thinspace
    \includegraphics[width=0.11\textwidth]{image/chap04/improvement/2007_000663_2.png}
    \enspace\thinspace
    \includegraphics[width=0.11\textwidth]{image/chap04/improvement/2007_000663_3.png}
    \enspace\thinspace
    \includegraphics[width=0.11\textwidth]{image/chap04/improvement/2007_000663_4.png}
    \enspace\thinspace
    \includegraphics[width=0.11\textwidth]{image/chap04/improvement/2007_000663_5.png}
    \enspace\thinspace
    \caption{并排的多张图像加各自的注解}
    \label{fig:improvement}
\end{figure}


\subsection{两列图像的插入}

\begin{figure}[ht] % image examples & compare
    \begin{subfigure}{0.55\textwidth}
        \makebox[0.18\textwidth]{\scriptsize Grid-5LSTM}
        \makebox[0.18\textwidth]{\scriptsize FCN-8s\cite{long2015fully}}
        \makebox[0.18\textwidth]{\scriptsize SDS\cite{hariharan2014simultaneous}}
        \makebox[0.18\textwidth]{\scriptsize 真值}
        \makebox[0.18\textwidth]{\scriptsize 图像} \\
        \includegraphics[width=0.18\textwidth]{image/chap04/result/compare/my_horse.pdf}
        \includegraphics[width=0.18\textwidth]{image/chap04/result/compare/fcn_horse.png}
        \includegraphics[width=0.18\textwidth]{image/chap04/result/compare/sds_horse.png}
        \includegraphics[width=0.18\textwidth]{image/chap04/result/compare/gt_horse.pdf}
        \includegraphics[width=0.18\textwidth]{image/chap04/result/compare/im_horse.pdf}
        \\
        \includegraphics[width=0.18\textwidth]{image/chap04/result/compare/my_motor.png}
        \includegraphics[width=0.18\textwidth]{image/chap04/result/compare/fcn_motor.png}
        \includegraphics[width=0.18\textwidth]{image/chap04/result/compare/sds_motor.png}
        \includegraphics[width=0.18\textwidth]{image/chap04/result/compare/2007_005173.png}
        \includegraphics[width=0.18\textwidth]{image/chap04/result/compare/2007_005173.jpg}
        \\
        \includegraphics[width=0.18\textwidth]{image/chap04/result/compare/my_sheep.pdf}
        \includegraphics[width=0.18\textwidth]{image/chap04/result/compare/fcn_sheep.png}
        \includegraphics[width=0.18\textwidth]{image/chap04/result/compare/sds_sheep.png}
        \includegraphics[width=0.18\textwidth]{image/chap04/result/compare/gt_sheep.pdf}
        \includegraphics[width=0.18\textwidth]{image/chap04/result/compare/im_sheep.pdf}
        \\
        \includegraphics[width=0.18\textwidth]{image/chap04/result/compare/my_boat.png}
        \includegraphics[width=0.18\textwidth]{image/chap04/result/compare/fcn_boat.png}
        \includegraphics[width=0.18\textwidth]{image/chap04/result/compare/sds_boat.png}
        \includegraphics[width=0.18\textwidth]{image/chap04/result/compare/2007_004241.png}
        \includegraphics[width=0.18\textwidth]{image/chap04/result/compare/2007_004241.jpg}
        \caption{左边的图像}
        \label{fig:compare1}
    \end{subfigure}
    \begin{subfigure}{0.4\textwidth}
        \centering
        %		\makebox[0.3\textwidth]{} \\
        %		\makebox[0.3\textwidth]{} \\
        \includegraphics[width=0.25\textwidth]{image/chap04/result/compare/2010_005284.jpg}
        \includegraphics[width=0.25\textwidth]{image/chap04/result/compare/2007_003349.jpg}
        \includegraphics[width=0.25\textwidth]{image/chap04/result/compare/2009_004507.jpg}
        \\
        \includegraphics[width=0.25\textwidth]{image/chap04/result/compare/2010_005284.png}
        \includegraphics[width=0.25\textwidth]{image/chap04/result/compare/2007_003349.png}
        \includegraphics[width=0.25\textwidth]{image/chap04/result/compare/2009_004507.png} \\
        \includegraphics[width=0.25\textwidth]{image/chap04/result/compare/zoom_bus.png}
        \includegraphics[width=0.25\textwidth]{image/chap04/result/compare/zoom_bird.png}
        \includegraphics[width=0.25\textwidth]{image/chap04/result/compare/zoom_dog.png} \\
        \includegraphics[width=0.25\textwidth]{image/chap04/result/compare/deeplab_bus.png}
        \includegraphics[width=0.25\textwidth]{image/chap04/result/compare/deeplab_bird.png}
        \includegraphics[width=0.25\textwidth]{image/chap04/result/compare/deeplab_dog.png} \\
        \includegraphics[width=0.25\textwidth]{image/chap04/result/compare/my_bus.png}
        \includegraphics[width=0.25\textwidth]{image/chap04/result/compare/my_bird.png}
        \includegraphics[width=0.25\textwidth]{image/chap04/result/compare/my_dog.png}
        \caption{右边的图像}
        \label{fig:compare2}
    \end{subfigure}
    \caption{复杂的两列对象的插入}
    \label{fig:complex}
\end{figure}


\clearpage

\section{表格的插入}

\begin{table}[ht] %voc table result
    \centering
    \caption{典型的实验对比表格}
    \begin{tabular}{*{4}{c}}
        \toprule
        Method                                & Pixel Acc.    & Mean Acc.     & Mean Iu.      \\
        \midrule
        Liu等人\cite{liu2011sift}             & 76.7          & -             & -             \\
        Tighe等人\cite{tighe2013finding}      & 78.6          & 39.2          & -             \\
        FCN-16s\cite{long2015fully}           & 85.2          & \textbf{51.7} & 39.5          \\
        Deeplab-LargeFOV\cite{chen14semantic} & 85.6          & 51.2          & 39.7          \\
        \midrule
        Grid-LSTM5                            & \textbf{86.2} & 51.0          & \textbf{41.2} \\
        \bottomrule
    \end{tabular}
    \label{tab:siftflow}
\end{table}

\begin{table}[h] %voc table result
    \centering
    \caption{复杂一些的表格}
    \resizebox{\textwidth}{!}{
        \begin{tabular}{c|*{20}{c}|c}
            \toprule
            Method                    & aero          & bike          & bird          & boat          & bottle        & bus           & car           & cat           & chair         & cow           & table         & dog           & horse         & mbike         & person        & plant         & shep          & sofa          & train         & tv            & mIoU.         \\
            \midrule
            CNN                       & 72.6          & 29.6          & 70.2          & 53.1          & 65.1          & 81.0          & 74.3          & 79.8          & 25.0          & 64.8          & 47.8          & 69.5          & 66.2          & 65.2          & 74.2          & 42.1          & 69.6          & 38.8          & 74.4          & 58.6          & 62.5          \\
            CNN+\textbf{1}LSTM        & 71.5          & 30.6          & 70.5          & 53.8          & 64.9          & 82.4          & 77.1          & 79.5          & 25.1          & 65.8          & 47.8          & 71.5          & 64.6          & 67.0          & 74.0          & 43.9          & 69.6          & 38.6          & 74.9          & 59.4          & 63.0          \\
            CNN+\textbf{2}LSTM        & 76.1          & 32.6          & 72.1          & 57.0          & 65.3          & 83.6          & 75.4          & 81.7          & 24.7          & 69.3          & 47.5          & 72.3          & 68.9          & 69.5          & 74.7          & 41.5          & 69.8          & 38.3          & 77.8          & 62.1          & 64.3          \\
            CNN+\textbf{3}LSTM        & 77.7          & 32.3          & 72.6          & 60.0          & 68.3          & 85.5          & 78.5          & 82.3          & 25.3          & 71.1          & 49.7          & 71.5          & 69.7          & 70.8          & 75.9          & 47.9          & 71.2          & 38.9          & 80.2          & 61.7          & 65.8          \\
            CNN+\textbf{4}LSTM        & 79.1          & \textbf{33.7} & \textbf{73.6} & \textbf{62.0} & \textbf{70.4} & 85.5          & \textbf{80.9} & 83.7          & \textbf{24.1} & 70.7          & 45.7          & 73.7          & 69.6          & 72.1          & 75.6          & 47.2          & \textbf{76.0} & 37.3          & 80.5          & 62.2          & 66.4          \\
            CNN+\textbf{5}LSTM        & \textbf{79.9} & 33.6          & \textbf{73.6} & 61.7          & 68.0          & \textbf{88.5} & \textbf{80.9} & \textbf{84.0} & 23.6          & \textbf{71.3} & \textbf{49.7} & \textbf{73.1} & \textbf{71.3} & \textbf{72.9} & \textbf{76.4} & \textbf{48.9} & 75.1          & \textbf{38.1} & \textbf{84.5} & \textbf{63.8} & \textbf{67.2} \\
            \midrule
            CNN+\textbf{5}LSTM$^\dag$ & 84.8          & 36.4          & 82.0          & 69.4          & 73.0          & 87.2          & 81.8          & 86.1          & 34.5          & 82.4          & 53.1          & 81.5          & 77.4          & 79.0          & 81.3          & 54.8          & 81.1          & 47.0          & 84.3          & 67.3          & 72.3          \\
            \bottomrule
        \end{tabular}}
    \label{tab:vocval}
\end{table}


\section{公式}
\label{sec:formula}
没有编号的公式
\begin{align*}
    \begin{split}
        \label{eq:feedforward}
        \mybold{z}^{(l)} & = \mybold{W}^{(l)}\mybold{a}^{(l-1)} + \mybold{b}^{(l)} \\
        \mybold{a}^{(l)} & = f(\mybold{z}^{(l)})
    \end{split}
\end{align*}
公式中含有中文
\begin{align}
    \begin{split}
        \mbox{像素准确率} &= \sum_{i=1}^{n_{cl}}n_{ii} / \sum_{i=1}^{n_{cl}}t_i \\
        \mbox{平均像素准确率} &= \frac{1}{n_{cl}} \sum_{i=1}^{n_{cl}}(n_{ii}/ t_i) \\
        \mbox{Mean IU} &= \frac{1}{n_{cl}} \sum_{i=1}^{n_{cl}}\frac{n_{ii}}{t_i + \sum_j^{n_{cl}} n_{ji} - n_{ii}}
    \end{split}
\end{align}
公式中含有矩阵
\begin{equation}
    \textbf{H} = \begin{bmatrix}
        I*\mybold{x}_i \\ \textbf{h}
    \end{bmatrix}
\end{equation}
每行后面都有编号的公式
\begin{align}
    \frac{\partial}{\partial W_{ij}^{(l)}} J(\mybold{W},\mybold{b};\mybold{x},y) & = \frac{\partial J(\mybold{W},\mybold{b};\mybold{x},y)}{\partial z_i^{(l+1)}}\cdot \frac{\partial z_i^{(l+1)}}{\partial W_{ij}^{(l)}} = \delta_i^{(l+1)}a_j^{(l)} \\
    \frac{\partial}{\partial b_i^{(l)}} J(\mybold{W},\mybold{b};\mybold{x},y)    & = \frac{\partial J(\mybold{W},\mybold{b};\mybold{x},y)}{\partial z_i^{(l+1)}}\cdot \frac{\partial z_i^{(l+1)}}{\partial b_i^{(l)}} = \delta_i^{(l+1)}
\end{align}

\section{算法流程图}
\label{sec:algorithm}
\begin{algorithm}[h]
    \KwIn{$m$个训练样本}
    \lFor{$l=1$ \emph{\KwTo} $n_l$}{
        初始化:$\Delta \mybold{W}^{(l)}=0$,$\Delta \mybold{b}^{(l)}=0$}
    \ForEach{训练样本}{
        \lFor{$l=1$ \emph{\KwTo} $n_l-1$}{
            前向传播:$\mybold{z}^{(l+1)}=\mybold{W}^la^l+\mybold{b}^l$,$\mybold{a}^{(l+1)}=f(\mybold{z}^{(l+1)})$}
        输出误差计算:$\delta^{(n_l)} = \frac{\partial}{\partial \mybold{z}^{(n_l)}} J(\mybold{W},\mybold{b};\mybold{x},y)$\;
        \lFor{$l=n_l-1$ \emph{\KwTo} $1$}{
            后向传播:$\delta^{(l)} = \bigl((\mybold{W}^{(l)})^T \delta^{(l+1)}\bigr)f'(\mybold{z}^{(l)})$}
        \ForAll{层l}{
            计算梯度:$\nabla_{\mybold{W}^{(l)}}J(\mybold{W},\mybold{b};\mybold{x},y)=\delta^{(l+1)}(\mybold{a}^{(l)})^T$ \\
            \hspace{60pt}$\nabla_{\mybold{b}^{(l)}}J(\mybold{W},\mybold{b};\mybold{x},y)=\delta^{(l+1)}$\;
            累加梯度:$\Delta \mybold{W}^{(l)} \leftarrow \Delta \mybold{W}^{(l)} + \nabla_{\mybold{W}^{(l)}}J(\mybold{W},\mybold{b};\mybold{x},y)$; \\
            \hspace{60pt}$\Delta \mybold{b}^{(l)} \leftarrow \Delta \mybold{b}^{(l)} + \nabla_{\mybold{b}^{(l)}}J(\mybold{W},\mybold{b};\mybold{x},y)$\;
        }
    }
    \ForAll{层$l$}{
        更新权重:$\mybold{W}^{(l)} \leftarrow \mybold{W}^{(l)} - \alpha \biggl[\frac 1m \Delta \mybold{W}^{(l)}\biggr]$ \\
        \hspace{60pt} $\mybold{b}^{(l)} \leftarrow \mybold{b}^{(l)} - \alpha \biggl[\frac 1m \Delta \mybold{b}^{(l)}\biggr]$
    }
    \caption{梯度下降算法}
    \label{algo:sgd}
\end{algorithm}

\section{例子、定理与证明}

\begin{eg}
    这是一个例子, 用以验证特殊环境的字体成功更改为楷体.
\end{eg}

\begin{theorem}[定理例子]
    \label{the:example-theorem}
    这是一个定理。
\end{theorem}

\begin{corollary}[推论例子]
    \label{the:example-corollary}
    这是一个推论。
\end{corollary}

\begin{lemma}[引理例子]
    \label{the:example-lemma}
    这是一个引理。
\end{lemma}

这里我们先给出\autoref{the:example-theorem-sysu-thesis}

\begin{theorem}[中山大学毕业论文模板定理]
    \label{the:example-theorem-sysu-thesis}
    中山大学 \LaTeX 毕业论文模板\cite{sysu-thesis}可以用于写各种证明。
\end{theorem}

下面我们对\autoref{the:example-theorem-sysu-thesis}进行证明:


\begin{proof}
    
    下面我们开始证明:

    由本定理的证明可见,我可以引用\autoref{the:example-theorem}和引理\ref{the:example-corollary}以及推论\ref{the:example-lemma}来证明我这个 \LaTeX 可以用来写各种证明。

    \autoref{the:example-theorem-sysu-thesis}得证。
\end{proof}

\section{代码}

本模版支持在论文中插入代码片段,或直接从源码文件进行插入。
例如,在论文中插入代码片段的效果为:
\begin{python}
    def func():
    print("hello world")
    with open('./output.txt', 'w') as f:
    L = f.readlines()

    if __name__ == "__main__":
    # this is a comment line
    func()
\end{python}
也可在行内插入代码片段,例如:Python中重载加法运算符的函数为\pyinline{__add__},类的标识符为\pyinline{class}。
此外,还可直接插入代码文件,例如插入\texttt{./code/demo.cpp}的效果为:
\lstinputlisting[style=sysucpp]{code/demo.cpp}


\section{其他的一些用法}
\label{sec:font}
\subsection{子章节编号}
\label{sec:font:subsection}
\subsubsection{更小的章节}
\label{sec:font:subsection:subsub}
更小的章节编号也是支持的。

可以如此引用章节:

\begin{itemize}
    \item \autoref{cha:usage-example}
    \item  \autoref{sec:font}
    \item  \autoref{sec:font:subsection}
    \item  \autoref{sec:font:subsection:subsub}
\end{itemize}


\subsection{列表的使用}
\label{sec:font:list}

这是一个无序列表
\begin{itemize}
    \item 引用文献\cite{long2015fully}
    \item 引用文献作者\citeauthor{long2015fully}
    \item 引用文献年份\citeyear{long2015fully} 
\end{itemize}

这是一个有序列表
\begin{enumerate}
    \item 索引前面的\autoref{sec:formula}、图像\ref{fig:complex}、表格\ref{tab:siftflow}
    \item 加脚注\footnote{测{\zihao{-5}试一下}脚注和URL \url{http://cs231n.github.io/transfer-learning/}}
\end{enumerate}



\newclearpage
%% chapter 5 dataset, network structure, experiment and result
\chapter{结论与展望}
\section{结论}
对于$m \times n$的广义棋盘,骨牌大小$1 \times k$,如果棋盘可以被平铺,棋盘一定满足条件$m \equiv 1 (mod k), n \equiv 1 (mod k)$。
且在此前提下,当$k \ge 3$时,当两个缺陷的坐标$(i_1, j_1)$和$(i_2, j_2)$满足$i_1 \equiv i_2 \equiv 1 (mod k), j_1 + 1 \equiv j_2 \equiv 2 (mod k)$,
且当$i_1 = i_2 = 1$或$i_1 = i_2 = m$时,$j_2 > j_1$,则此时棋盘可被平铺。

当$k = 3$的情况,当两个缺陷的坐标$(i_1, j_1)$和$(i_2, j_2)$满足$i_1 + 1\equiv i_2 \equiv 0 (mod 3), j_1 \equiv j_2 \equiv 0 (mod 3)$,棋盘也可以被平铺。

除此之外,所有的缺陷两格的$m \times n$棋盘都不可被平铺。

\section{展望}
事实上,关于残缺棋盘的完全覆盖问题是一个困难而且复杂的问题,在本论文解决了缺陷两格的完全覆盖问题后,可以沿着本文的解决思路结局缺陷三格甚至缺陷k格的完全覆盖问题,这是下一步的研究方向。

\newclearpage

% 结语

% 附录部分
\backmatter
% 参考文献. 因不需要纳入章节目录, 故放入附录部分
% 实际上参考文献是属于论文主体部分
\makereferences

% 附录
{
    \appendix
    \chapter{区域划分}

\section{\texorpdfstring{$k_{12} - k_{22} = 1$}的区域划分}

\begin{table}[h]
    \begin{tabular}{|ccccclcl|}
    \hline
    \multicolumn{2}{|c|}{\multirow{3}{*}{}}                          & \multicolumn{6}{c|}{}                                                                                                                                 \\ \cline{3-8} 
    \multicolumn{2}{|c|}{}                                           & \multicolumn{1}{c|}{\multirow{3}{*}{}} & \multicolumn{3}{c|}{\multirow{2}{*}{}}                              & \multicolumn{2}{c|}{\multirow{6}{*}{}} \\
    \multicolumn{2}{|c|}{}                                           & \multicolumn{1}{c|}{}                  & \multicolumn{3}{c|}{}                                               & \multicolumn{2}{c|}{}                  \\ \cline{1-2} \cline{4-6}
    \multicolumn{1}{|c|}{\multirow{3}{*}{}} & \multicolumn{1}{c|}{2} & \multicolumn{1}{c|}{}                  & \multicolumn{1}{c|}{1}     & \multicolumn{2}{c|}{\multirow{3}{*}{}} & \multicolumn{2}{c|}{}                  \\ \cline{2-4}
    \multicolumn{1}{|c|}{}                  & \multicolumn{3}{c|}{\multirow{2}{*}{}}                                                       & \multicolumn{2}{c|}{}                  & \multicolumn{2}{c|}{}                  \\
    \multicolumn{1}{|c|}{}                  & \multicolumn{3}{c|}{}                                                                        & \multicolumn{2}{c|}{}                  & \multicolumn{2}{c|}{}                  \\ \cline{1-6}
    \multicolumn{6}{|c|}{}                                                                                                                                                          & \multicolumn{2}{c|}{}                  \\ \hline
    \end{tabular}
    \end{table}


\section{\texorpdfstring{$k_{12} - k_{22} > 1$}的区域划分}

% Please add the following required packages to your document preamble:
% \usepackage{multirow}
% Please add the following required packages to your document preamble:
% \usepackage{multirow}
\begin{table}[h]
    \begin{tabular}{|ccclclcc|}
        \hline
        \multicolumn{2}{|c|}{\multirow{3}{*}{}} & \multicolumn{6}{c|}{}                                                                                                                               \\ \cline{3-8}
        \multicolumn{2}{|c|}{}                  & \multicolumn{2}{c|}{\multirow{3}{*}{}} & \multicolumn{3}{c|}{\multirow{2}{*}{}} & \multirow{3}{*}{}                                                 \\
        \multicolumn{2}{|c|}{}                  & \multicolumn{2}{c|}{}                  & \multicolumn{3}{c|}{}                  &                                                                   \\ \cline{1-2} \cline{5-7}
        \multicolumn{1}{|c|}{\multirow{3}{*}{}} & \multicolumn{1}{c|}{2}                 & \multicolumn{2}{c|}{}                  & \multicolumn{2}{c|}{\multirow{3}{*}{}} & \multicolumn{1}{c|}{1} & \\ \cline{2-4} \cline{7-8}
        \multicolumn{1}{|c|}{}                  & \multicolumn{3}{c|}{\multirow{2}{*}{}} & \multicolumn{2}{c|}{}                  & \multicolumn{2}{c|}{\multirow{3}{*}{}}                            \\
        \multicolumn{1}{|c|}{}                  & \multicolumn{3}{c|}{}                  & \multicolumn{2}{c|}{}                  & \multicolumn{2}{c|}{}                                             \\ \cline{1-6}
        \multicolumn{6}{|c|}{}                  & \multicolumn{2}{c|}{}                                                                                                                               \\ \hline
    \end{tabular}
\end{table}
\endinput

    \newclearpage
}

%%
% 致谢
% 谢辞应以简短的文字对课题研究与论文撰写过程中曾直接给予帮助的人员(例如指导教师、答疑教师及其他人员)表示对自己的谢意,这不仅是一种礼貌,也是对他人劳动的尊重,是治学者应当遵循的学术规范。内容限一页。
% modifier: 黄俊杰
% update date: 2017-04-15
%%

\chapter{致谢}

四年时间转眼即逝,青涩而美好的本科生活快告一段落了。回首这段时间,我不仅学习到了很多知识和技能,而且提高了分析和解决问题的能力与养成了一定的科学素养。虽然走过了一些弯路,但更加坚定我后来选择学术研究的道路,实在是获益良多。这一切与老师的教诲和同学们的帮助是分不开的,在此对他们表达诚挚的谢意。

首先要感谢的是我的指导老师崔潇易教授。我作为一名本科生,缺少学术研究经验,不能很好地弄清所研究问题的重点、难点和热点,也很难分析自己的工作所能够达到的层次。崔老师对整个研究领域有很好的理解,以其渊博的知识和敏锐的洞察力给了我非常有帮助的方向性指导。他严谨的治学态度与辛勤的工作方式也是我学习的榜样,在此向崔老师致以崇高的敬意和衷心的感谢。

最后我要感谢我的家人,正是他们的无私的奉献和支持,我才有了不断拼搏的信心和勇气,才能取得现在的成果。

\vskip 108pt
\begin{flushright}
	卢皓斌\makebox[1cm]{} \\
	\today
\end{flushright}

    % 致谢
\newclearpage

% \makeGrade      % 成绩评定记录表
\end{document}
