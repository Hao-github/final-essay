\documentclass[
    % fontset=ubuntu, % 生僻字可用思源字体,如“昇腾”
    fontset=fandol,
    xcolor=svgnames % SeaGreen
]{ctexbeamer}

\usetheme[
    % compress % 进度条压缩在一行,可选
]{Berlin}

\usecolortheme[named=SeaGreen]{structure}

\title{在缺陷两格的棋盘上的直线骨牌平铺问题}

\subtitle[申请中山大学理学学士学位论文答辩报告]{
    ~\\
    (申请中山大学工学理士学位论文答辩报告)
}

\author[卢皓斌]{
    \texorpdfstring{
        学生:卢皓斌\\
        ~\\
        \href{mailto:luhb5@mail2.sysu.edu.cn}{luhb5@mail2.sysu.edu.cn}
    }{PDF Bookmark Version}
}

\institute[中山大学~数学学院(珠海)]{
    \includegraphics[height=0.1\textheight]{../image/template/logo.png}

}

\logo{\includegraphics[height=0.1\textheight]{../image/template/sysu-logo.pdf}} % 这个是每页都会出现的水印,可能会影响观感,自行决定放不放

\date{
    %\today
    二〇二三年五月
}

\begin{document}

\section{Intro}

\begin{frame}

    \titlepage

\end{frame}

\subsection{选题背景}

\begin{frame}

    \begin{block}{要点一\footnote{引用一}}
        \begin{itemize}
            \item 条目一,每一行字不要太密
            \item 条目二
            \item 条目三
        \end{itemize}
    \end{block}

    \begin{block}{要点二}
        \begin{itemize}
            \item 条目一\footnote{引用二}
            \item 条目二
        \end{itemize}
    \end{block}

\end{frame}

\section{规整棋盘规模}

\subsection{图像搭配单页说明}

\begin{frame}

    \begin{block}{生僻字测试}
        \begin{itemize}
            \item 华为昇腾异构处理器\footnote{\url{https://www.hisilicon.com/cn/products/Ascend}}
            \item 条目二
        \end{itemize}
    \end{block}

    \begin{figure}
        % \includegraphics[width=0.618\textwidth]{../image/chap04/illustration/hole.pdf}
        \caption{单张图像}
        \label{fig:hole}
    \end{figure}

\end{frame}

\begin{frame}

    \begin{block}{要点一}
        \begin{itemize}
            \item 条目一
            \item 条目二
        \end{itemize}
    \end{block}

    \begin{figure} %文中的Grid-LSTM模型做的语义图像分割的例子
        % \includegraphics[width=.2\textwidth,height=.15\textwidth]{../image/chap04/example/2007_000799.jpg}
        % \includegraphics[width=.2\textwidth,height=.15\textwidth]{../image/chap04/example/2007_002094.jpg}
        % \includegraphics[width=.2\textwidth,height=.15\textwidth]{../image/chap04/example/2007_004483.jpg}
        % \includegraphics[width=.2\textwidth,height=.15\textwidth]{../image/chap04/example/2007_003194.jpg}
        % \\
        % \includegraphics[width=.2\textwidth,height=.15\textwidth]{../image/chap04/example/2007_000799.pdf}
        % \includegraphics[width=.2\textwidth,height=.15\textwidth]{../image/chap04/example/2007_002094.pdf}
        % \includegraphics[width=.2\textwidth,height=.15\textwidth]{../image/chap04/example/2007_004483.pdf}
        % \includegraphics[width=.2\textwidth,height=.15\textwidth]{../image/chap04/example/2007_003194.pdf}
        \caption{并排的多张图像}
        \label{fig:multi-image-example1}
    \end{figure}

\end{frame}

\subsection{单页双图}

\begin{frame}

    \begin{block}{要点一}
        \begin{itemize}
            \item 条目一
            \item 条目二
        \end{itemize}
    \end{block}

    \begin{figure}
        \begin{minipage}{0.49\textwidth}
            % \includegraphics[height=0.45\textheight]{../image/chap04/example/2007_004483.pdf}
            \caption{一}
        \end{minipage}
        \begin{minipage}{0.49\textwidth}
            % \includegraphics[height=0.45\textheight]{../image/chap04/example/2007_003194.pdf}
            \caption{二}
        \end{minipage}
    \end{figure}

\end{frame}

\subsection{单页大图}

\begin{frame}

    \begin{figure}
        % \includegraphics[width=0.9\textwidth]{../image/chap03/overleaf-example.jpg}
        \caption{Overleaf使用例子,这里的描述可以长一些}
        \label{fig:overleaf-example}
    \end{figure}

\end{frame}

\section{染色法处理}

\subsection{图像搭配单页说明}

\begin{frame}

    \begin{block}{再试试公式}
        \begin{itemize}
            \item 看下面!
        \end{itemize}
    \end{block}

    \begin{equation}
        e^{i\pi}+1=0
    \end{equation}

\end{frame}

\section{无穷递降/分治法处理}

\begin{frame}

    \begin{block}{最后还有表格}
        \begin{itemize}
            \item 往下看!
        \end{itemize}
    \end{block}

    \begin{table}
        \begin{tabular}{ccc}
            \hline
            姓名         & 学号  & 性别     \\
            \hline
            Steve Jobs & 001 & Male   \\
            Bill Gates & 002 & Female \\
            \hline
        \end{tabular}
        \caption{表格示例,乱写的}
        \label{fig:table-example}
    \end{table}

\end{frame}

\section{总结与展望}

\subsection{}

\begin{frame}
    \begin{block}{问题的解}
        \begin{enumerate}
            \item 对于$m \times n$的缺陷两格广义棋盘,骨牌大小$1 \times k$,其中$k \ge 3$。满足$m \equiv 2 (mod k), n \equiv 1 (mod k)$。
            \item 当$i_1 = i_2 \notin \left\{1, m\right\}$,缺陷格子坐标$(i_1, j_1)$和$(i_2, j_2)$满足$i_1 \equiv i_2 \equiv 1 (mod k), j_1 + 1 \equiv j_2 \equiv 2 (mod k)$时,可完全覆盖。
            \item 当$i_1 = i_2 \in \left\{1, m\right\}$,满足$j_2 > j_1, m \ge 2k + 2, n \ge 2k + 1$的条件时,棋盘可完全覆盖。
            \item 当$k = 3$时,缺陷格子坐标$(i_1, j_1)$和$(i_2, j_2)$满足$i_1 + 1\equiv i_2 \equiv 0 (mod 3), j_1 \equiv j_2 \equiv 0 (mod 3)$,棋盘也可以被平铺。
            \item 对于其余情况,棋盘都不可被平铺。
        \end{enumerate}
    \end{block}
\end{frame}
% \begin{alertblock}{重要理论}
%     \begin{itemize}
%         \item 条目一\footnote{引用二}和\alert{高亮字色}
%     \end{itemize}
% \end{alertblock}

\begin{frame}
    \begin{block}{展望与其他研究方向}
        \begin{itemize}
            \item 对于$m \times n$的缺陷$z$格($z > 2$)的广义棋盘的直线骨牌的平铺问题。
            \item 对于$m \times n$的缺陷两格的广义棋盘的各类L型骨牌的平铺问题。
            \item 对于$m \times n \times k$的立体棋盘中,直线骨牌的平铺问题。
        \end{itemize}
    \end{block}
\end{frame}

\section{Q \& A}

\begin{frame}

    \begin{block}{Questions?}
        ~\\
        ~\\
        \center{\Large{Thank you!}}\\
        ~\\
        ~\\
        ~\\
        ~\\
    \end{block}

\end{frame}
\end{document}